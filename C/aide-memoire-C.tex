\documentclass[a4paper,twoside]{article}
\usepackage{autiwa}
\usepackage{listings}



\title{Aide mémoire C}
\author{Autiwa}

\newcommand{\raccourci}[1]{{\bfseries #1}}


\makeindex
\begin{document}

\tableofcontents

\clearpage


\section{Les pointeurs}\label{sec:pointeurs}
Pour déclarer un pointeur, il faut en même temps déclarer le type de ce vers quoi on pointe : 
\begin{lstlisting}[language=C++]
int ageUtilisateur(16);  //declaration
int& maVariable(ageUtilisateur); //pointeur toward ageUtilisateur
\end{lstlisting}

Pour une variable normale, on peut afficher l'adresse correspondante via :
\begin{lstlisting}[language=C]
printf("L'adresse de la variable age est : %p", &age);
\end{lstlisting}

\textbf{age} désigne la valeur de la variable ; \textbf{&age} désigne l'adresse de la variable.

\subsection{Créer un pointeur}

Pour créer une variable de type pointeur, on doit rajouter le symbole * devant le nom de la variable.
\begin{lstlisting}[language=C]
int age = 10;
int *monPointeur=NULL;
pointeurSurAge = &age;
\end{lstlisting}




\end{document}
