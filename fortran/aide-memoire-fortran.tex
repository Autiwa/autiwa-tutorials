\documentclass[a4paper,twoside]{article}
\usepackage{autiwa}
\usepackage{listings}
\lstset{language=IDL,
basicstyle=\ttfamily\small,
columns=flexible,
escapechar=+}
\lstset{keywordstyle=\color{blue}\bfseries}

\title{Aide mémoire Fortran 90}
\author{Autiwa}

\newcommand{\raccourci}[1]{{\bfseries #1}}

\makeindex
\begin{document}

\tableofcontents

\clearpage

\section{Préambule}
Ceci est un tutoriel fortran 90, il a pour but de donner des astuces de programmations, des bonnes pratiques, présenter ce qui se faisait en fortran 77 et qu'il ne faut plus faire. 

\section{Transition fortran 77/fortran 90}
\subsection{Instructions obsolètes ou dépréciées}

\begin{center}
\begin{tabular}{ll}
Obsolètes & Déprécié\\
IF arithmétique & format fixe\\
GO TO assigné & COMMON\\
RETURN multiple & DATA au milieu des inst.\\
FORMAT assigné & BLOCK DATA\\
DO sur une même instruc. & EQUIVALENCE\\
Index réel de boucle DO & GO TO calculé\\
branchement sur END IF & INCLUDE\\
PAUSE & ENTRY\\
descripteur H & DOUBLE PRECISION\\
 & Instructions Fonction\\
 & SEQUENCE\\
 & DO WHILE
\end{tabular}
\end{center}


\end{document}