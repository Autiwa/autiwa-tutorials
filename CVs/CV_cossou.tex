\documentclass[helvetica,openbib,totpages,french]{europecv}
\usepackage[T1]{fontenc}
\usepackage{graphicx}
\usepackage[a4paper,top=1.27cm,left=1cm,right=1cm,bottom=2cm]{geometry}
\usepackage[french]{babel}
\usepackage{bibentry}
\usepackage{url}
\newcommand{\tabitem}{~~\llap{\textbullet}~~}

\renewcommand{\ttdefault}{phv} % Uses Helvetica instead of fixed width font

\ecvname{Cossou, Christophe}
%\ecvfootername{Name Surname}
\ecvaddress{69, rue anatole france ; 33150 CENON}
\ecvtelephone[06 02 26 97 63]{05 33 51 98 13}
%\ecvtelephone[portable (Remove if not relevant)]{fixe (Remove if not relevant)}

%\ecvfax{(Remove if not relevant)}
\ecvemail{\url{ccossou@gmail.com}}
\ecvnationality{Française}
\ecvdateofbirth{28/06/1986}
\ecvgender{Masculin}
%\ecvpicture[width=2cm]{mypicture}
\ecvfootnote{\textcopyright~Union européenne, 2002--\the\year{} | \url{http://europass.cedefop.eu.int}}

\begin{document}
\selectlanguage{french}


\begin{europecv}
\ecvpersonalinfo%[5pt]

\ecvitem[.5em]{\ecvSectionStyle{Emploi recherché}}{\large\textbf{Ingénieur analyste}}

\ecvsection{Expérience professionnelle}
%\ecvitem{Dates}{Add separate entries for each relevant post occupied, starting from the most recent. (Remove if not relevant).}
%\ecvitem{Occupation or position held}{\ldots}
%\ecvitem{Main activities and responsibilities}{\ldots}
%\ecvitem{Name and address of employer}{\ldots}
%\bigskip
%\ecvitem{Type of business or sector}{\ldots}
\ecvitem{2014}{Audit de programme informatique}
\ecvitem{Poste occupé}{Ingénieur de recherche}
\ecvitem{Activités principales}{\begin{itemize}
\item Analyse fonctionnelle des besoins utilisateurs
\item Amélioration ergonomique et algorithmique d'un code informatique de simulation déjà existant
\item Définir les protocoles et les scenarii de tests
\item Documenter les applications pour les développements ultérieurs
\item Rédaction d'une documentation utilisateur, en parallèle de leur formation
\end{itemize}}
\ecvitem{Employeur}{CNRS/Laboratoire d'Astrophysique de Bordeaux}
\bigskip
\ecvitem{secteur}{Modélisation numérique de la chimie dans les nuages interstellaires.}

\ecvitem{2011--2013}{Thèse de doctorat}
\ecvitem{Poste occupé}{Chercheur}
\ecvitem{Activités principales}{\begin{itemize}
\item Expert des problématiques scientifiques
\item Veille scientifique
\item Définir des méthodologies, objectifs et ressources nécessaires au bon déroulement des travaux de recherche.
\item Confronter l'ensemble des résultats obtenus aux hypothèses de départ afin de les valider, corriger ou infirmer
\item Rédaction d'article scientifiques, et présentation dans des colloques internationaux.
\end{itemize}}
\ecvitem{Employeur}{CNRS/Laboratoire d'Astrophysique de Bordeaux}
\ecvitem{secteur}{Modélisation numérique de la formation des systèmes planétaires}


\ecvsection{Éducation et Formation}

\ecvitem{2005--2010}{Master de physique fondamentale, Université Bordeaux 1, Talence (33)}

%\ecvitem{Title of qualification awarded}{\ldots}
\ecvitem{Principaux thèmes}{Mathématiques, Physique (optique, électromagnétisme, physique quantique, physique nucléaire, astrophysique), Informatique}
%\ecvitem{Name and type of organization providing education and training}{\ldots}
%\ecvitem{Level in national or international classification\footnote{If appropriate.}}{\ldots}

\ecvitem{Juin 2004}{Baccalauréat scientifique, Lycée Val de Garonne, Marmande (47), Mention bien}

\ecvsection{Compétences personnelles}

\ecvmothertongue[5pt]{Français}
\ecvitem{\large Autre(s) langue(s)}{}
\ecvlanguageheader{(*)}
\ecvlanguage{Anglais}{\ecvCOne}{\ecvCOne}{\ecvCOne}{\ecvCOne}{\ecvBTwo}
\ecvlanguage{Espagnol}{\ecvATwo}{\ecvATwo}{\ecvAOne}{\ecvAOne}{\ecvAOne}
\ecvlanguagefooter[10pt]{(*)}

\ecvitem[10pt]{\large Compétences en communication}{Vulgarisation et formation (pour les utilisateurs d'un programme, pour de nouveaux outils), rédaction de comptes rendus (articles scientifiques), communication en anglais dans des colloques internationaux}
\ecvitem[10pt]{\large Compétences organisationnelles/managériales}{Gestion de projet (Définir un cadre et des méthodes pour résoudre un problème)}
%\ecvitem[10pt]{\large Compétences liées à l'emploi}{Replace this text by a description of these competences and indicate where they were acquired (remove if not relevant).}
\ecvitem[10pt]{\large Compétences informatiques}{}
\ecvitem[10pt]{Outils}{IDL, Matlab, Excel, Word, Powerpoint, Gnuplot, \LaTeX, Bash avancé, Eclipse, Vim, Gimp, Inkscape}
\ecvitem[10pt]{Langages}{Python, Java, Fortran (77,90), C (notions)}
\ecvitem[10pt]{Systèmes}{GNU/Linux, Mac OS X, Windows}
\ecvitem[10pt]{Suivi de projet}{Git, SVN, Doxygen}

%\ecvitem[10pt]{\large Autres compétences}{Replace this text by a description of these competences and indicate where they were acquired (remove if not relevant).}
%\ecvitem{\large Permis de conduire}{State here whether you hold a driving licence and if so for which categories of vehicle. (Remove if not relevant).}

\ecvsection{Information complémentaire}
%\ecvitem[10pt]{}{Include here any other information that may be relevant, for example contact persons, references, etc. (Remove heading if not relevant).}
%\bibliographystyle{plain}
%\nobibliography{publications}
%\ecvitem{}{\textbf{Publications}}
%\ecvitem{}{\bibentry{pub1}}
%\ecvitem[10pt]{}{\bibentry{pub2}}
\ecvitem{}{\textbf{Passion}}
\ecvitem{}{Photographie (Paysage, macrophotographie, animaux, portraits)}
%
%\ecvsection{Annexes}
%\ecvitem{}{List any item attached to the CV}
\end{europecv}


\end{document} 