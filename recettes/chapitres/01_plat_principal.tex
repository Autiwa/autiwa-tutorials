\begin{recette}{Bœuf Bourguignon}{4}{15 min.+12h+45min.}{6h}\index{bœuf bourguignon}\index{bœuf}\index{daube}
\begin{ingredients}
\ingredient 1.5kg de viande de bœuf
\ingredient 100g de lardons fumés (plutôt lardon qu'allumette vu que ça va cuire longtemps)
\ingredient 75cl de vin rouge corsé
\ingredient 5cl de cognac
\ingredient 3 gousses d'ail
\ingredient 2 oignons
\ingredient 4 échalottes
\ingredient 4 carottes
\ingredient 100g d'olives noires
\ingredient 20cl de bouillon de volaille (adapter pour que la viande soit recouverte quand ça mijote)
\ingredient 1 cuillère à soupe de coulis de tomate
\ingredient une cuillère à soupe de farine
\ingredient huile d'olive, sel, sucre, poivre, un clou de girofle, céleri, herbe de provence, bouquet garni
\end{ingredients}

\begin{preparation}
\etape Émincez oignons et échalottes. Coupez les carottes en petits cubes (de 0.5cm de coté environ). Hachez l'ail. 
\etape Dans un grand récipient, ajoutez oignons, échalottes, carotte, ail, épices (clou de girofle, céleri, herbe de provence,
bouquet garni). Salez et poivrez. Arrosez avec le vin et le cognac. Mélangez puis ajoutez la viande (et arrangez vous pour
qu'elle soit recouverte par la marinade).
\etape Faites ensuite mariner cette préparation environ 12h au réfrigérateur (en couvrant le récipient par un couvercle ou du
film étirable). 
\etape Sortez la marinade du frigo. Égouttez la viande puis les légumes de la marinade dans des récipients séparés
\etape Pendant ce temps, faites revenir les lardons dans le récipient qui servira à mijoter (une cocotte ou marmite).
Réservez-les dès qu'ils sont un peu blanchis, il ne faut pas qu'ils soient cuits, juste qu'il rendent le gras
\etape Complétez le gras des lardons avec de l'huile d'olive si besoin et saisissez la viande à feu vif, juste pour la colorer
un peu, très légèrement. Réservez la viande. 
\etape Faites alors revenir les légumes de la marinade, afin qu'ils rendent l'alcool qu'ils ont absorbé, et qu'ils rissolent un
peu. 
\etape Saupoudrez une cuillère à soupe de farine sur les légumes, puis mélangez la bien. Ajoutez alors le bouillon de bœuf,
mélangez. 
\etape Incorporez ensuite la cuillère à soupe de tomate, une pincée de sucre, le vin de la marinade et les olives. 
\etape Mélangez la préparation et ajoutez enfin les lardons et la viande. 
\end{preparation}

\begin{cuisson}
Couvrez la marmite et faites mijoter au minimum pendant environ 6h (ça dépend du feeling, mais j'ai fait mijoter très longtemps,
pas loin de 7h)
\end{cuisson}

\end{recette}

\begin{recette}{Blanquette de Veau}{3}{20 min}{2h}\index{blanquette de veau}\index{veau}
\begin{ingredients}
\ingredient 1,2 kg d'épaule ou de tendron de veau coupé en morceaux
\ingredient 1 carotte
\ingredient 2 blancs de poireaux
\ingredient 1 gros oignon
\ingredient 1 gousse d'ail
\ingredient 2 échalotes
\ingredient 1 brin de céleri
\ingredient 300 g de champignons de Paris
\ingredient 3 cuillères à soupe de vin blanc sec
\ingredient 2 jaunes d'oeufs
\ingredient 1 dl de crème fraîche
\ingredient 70 g de beurre + 70g de farine (roux blond)
\ingredient 1 bouquet de persil, bouquet garni, quelques gouttes de citron
\end{ingredients}

\begin{preparation}
\etape Pelez carotte, ail, échalotes et oignon. Hachez ce dernier ainsi que les blancs de poireaux. Coupez les échalotes et la carotte en deux.

\etape Portez à ébullition 2 litres d'eau dans un grand faitout, plongez-y les morceaux de viande pendant environ une minute pour les blanchir (blanchir la viande permet d'éliminer les éventuelles impuretés tout en la rendant plus ferme). Egouttez la viande, rincez-la sous l'eau froide et jetez l'eau de cuisson.

\etape Replacez la viande dans le faitout rincé. Ajoutez oignon et poireaux hachés, carottes, échalotes, ail, céleri et bouquet garni. Salez, poivrez et mouillez avec le vin. Ajoutez de l'eau pour que la viande et les légumes soient immergés.

\etape Couvrez. Portez à ébullition et laissez cuire 1 h 30. Faites cuire dans une poêle les champignons coupés 10 minutes environ.

\etape Préparez un roux blond : faites fondre le beurre dans une casserole, versez d'un coup la farine et mélangez vivement au fouet jusqu'à ce qu'une couleur blonde et une odeur de noisette se dégage, puis laissez refroidir. Quand la viande est cuite, mettez-la dans une passoire avec les légumes et récupérez le bouillon de cuisson. Délayez le roux avec ce bouillon et amenez à ébullition en fouettant.

\etape Remettez la viande et tous les légumes dans le faitout après avoir retiré bouquet garni, ail, céleri et carotte. Ajoutez les champignons, versez la sauce et réchauffez le tout 10 à 15 mn.

\etape Juste avant de servir mélangez la crème et les jaunes d'oeufs, incorporez-les à la sauce en tournant sans laisser bouillir. Ajoutez quelques gouttes de jus de citron. Servez dans un plat creux avec du persil.
\end{preparation}



\end{recette}

\begin{recette}{Brandade de poisson}{2}{1h30}{}\index{brandade de poisson}\index{morue}
\begin{ingredients}
\ingredient 1kg de poisson surgelés (filets de colin ou autre, le moins cher)
\ingredient 2 gousses d'ail
\ingredient 50cl de crème fraiche épaisse
\ingredient 2.5kg de pomme terre
\ingredient laurier, persil, beurre, huile d'olive, jus de citron
\end{ingredients}

\begin{remarque}
C'est bien sur la brandade de morue à la base, mais la lotte coûtant moins cher que la morue, j'ai adapté la recette. Le principe est de faire avec n'importe quel poisson.
\end{remarque}


\begin{preparation}
\etape Émincez l'ail très finement, 
\etape Faire cuire la lotte dans une casserole d'eau froide au départ, avec deux feuilles de laurier. Salez l'eau abondamment (c'est pour récupérer le caractère salé de la morue qui est d'ordinaire le poisson pour ce plat). Portez à ébullition et laisser cuire à petite ébulition pendant 10 minutes.
\etape En parallèle, faites cuire les pommes de terre dans une casserole d'eau pendant 20 minutes (ou à l'auto-cuiseur).
\etape Égouttez le poisson, enlevez l'arrête centrale et émiettez-le dans une casserole contenant l'ail et le persil. Mettez un peu de jus de citron. Augmentez un peu le feu et rajoutez de l'huile progressivement. Laissez alors rissolez à feu moyen en remuant de temps en temps. Salez.
\etape Écrasez les pommes de terre à la fourchette et mettre la purée ainsi formée dans la casserole.
\etape Ajoutez la crème fraîche et tourner à nouveau pour bien mélanger les ingrédients. Si le mélange est trop sec et ne forme pas une purée, ajouter un peu d'eau de cuisson des pommes de terre (ou du poisson si vous faites à l'auto-cuiseur).
\etape Placez dans un plat à gratin, ajouter quelques fines lamelles de beurre dessus et faire gratiner pendant 5 minutes à 200°C
\end{preparation}

\end{recette}

\begin{recette}{Calamars à l'armoricaine}{2}{}{}\index{calamar à l'armoricaine}\index{calamar}
\begin{ingredients}
\ingredient 1 grosse boite de tomates en dés
\ingredient 500g de calamars
\ingredient 4 échalotes
\ingredient 3 ou 4 oignons
\ingredient 1 gousse d'ail
\ingredient 25cl de vin blanc
\ingredient 5 cl de cognac
\ingredient 20g de beurre
\ingredient sel, poivre, huile d'olive, piment ou sauce piquante, sucre
\end{ingredients}

\begin{remarque}
J'ai mis 750g de calamars (une poche et demie) pour 4. Ça diminue beaucoup durant la cuisson. Donc en gros, on peut mettre une poche, suivant le poids de la poche, c'est pas très grave qu'il y en ait un peu moins ou un peu plus.
\end{remarque}


\begin{preparation}
\etape Pelez et hachez lail, l'oignon et l'échalote (avec un robot, pas besoin de s'embêter).
\etape Faites revenir les ronds de calamars (sans les décongeler s'ils le sont) dans le beurre et l'huile pendant 2 minutes environ. (dans la pratique, s'ils sont surgelés, faut au moins qu'ils soient décongelés). Une fois fait, réservez les calamars et le jus rendu dans un récipient.
\etape Faites revenir à feu doux la mixture oignon+échalote+ail. 
\etape Une fois légèrement transparent, à peine doré, rajoutez les calamars, laissez un peu réchauffer, puis rajouter le cognac et faites flamber.
\etape Ajoutez les tomates en dés, le vin blanc, salez, poivrez et laissez mijoter à couvert pendant une heure environ.
\etape Enfin, durant la cuisson, une fois que tout est un peu mélangé, goutez. Compensez les gouts avec sel poivre et piment fort. Et s'il y a une sorte d'aigreur, rajoutez un peu de sucre afin de l'éliminer. Bien entendu, goutez jusqu'à ce que ça vous convienne.
\end{preparation}

\end{recette}

\begin{recette}{Canard entier au four}{4}{20 minutes}{1h30}\index{canard au four}\index{canard}
\begin{ingredients}
\ingredient un canard entier
\ingredient pain dur
\ingredient sel, poivre du moulin
\end{ingredients}

\begin{preparation}
\etape Enlevez ce que vous pouvez de l'intérieur du canard. Si l'estomac est explosé, vous pouvez passer un peu d'eau pour éliminer les morceaux.
\etape Garnissez de pain l'intérieur du canard, puis fermez avec la peau qui dépasse peut-être, en la coinçant avec des piques en bois.
\etape Salez et poivrez le canard.
\end{preparation}

\begin{cuisson}
Faites cuire à 220°C pendant 1h30 environ (20 minutes par 500g)

Dans mon cas, j'ai fait avec un canard gras de 3.3kg, comme il avait beaucoup de gras, j'ai diminué un peu le poids du canard pour le calcul du temps de cuisson, puisque ce gras fond rapidement. In fine, il a cuit 1h30 à peu près.
\end{cuisson}
\end{recette}

\begin{recette}{Canard madère}{4}{45min}{1h}\index{canard madère}\index{madère}\index{canard}
% (Excellent)

\begin{ingredients}
\ingredient 4 cuisses de canard
\ingredient 20g de beurre
\ingredient $2$ oignons
\ingredient $1$ carotte
\ingredient $2$ gousses d'ail
\ingredient $15$ cl de madère
\ingredient 15cl de bouillon de volaille
\ingredient $50$ g d'olives dénoyautées
\ingredient 1 cuillère à soupe rase de farine
\ingredient sel, poivre du moulin
\end{ingredients}

\begin{preparation}
\etape Écrasez les gousses d'ails, émincez l'oignon et coupez la carotte en petits cubes (en fine lamelle que vous coupez en tranche)
\etape Faites fondre le beurre dans une sauteuse et faites revenir les cuisses à feu vif. Pas besoin que la viande soit cuite à l'intérieur, c'est juste pour faire dorer.
\etape Réservez les morceaux puis faites revenir l'ail, l'oignon et les dés de carotte.
\etape Ajoutez alors la farine, remuez jusqu'à l'incorporer autour des légumes. Ajoutez enfin le madère, le bouillon et les olives vertes. Ajoutez un peu de poivre, mélangez et rajoutez enfin les morceaux de canard.
\end{preparation}

\begin{cuisson}
Faites cuire pendant une heure environ à feu doux.
\end{cuisson}


\begin{remarque}
Mon avis est que cette sauce va très bien avec du riz.
\end{remarque}

\end{recette}

\begin{recette}{Canard aux pruneaux}{3}{}{}\index{canard aux pruneaux}\index{pruneaux}\index{canard}
% (recette que j'ai inventé)
\begin{ingredients}
\ingredient morceaux de canards (8 manchons par exemple)
\ingredient 3 échalottes
\ingredient 150g de champignons
\ingredient 25cl de bouillon de volaille
\ingredient une cuillère à café de fond de veau
\ingredient 10cl de cognac
\ingredient 20cl de vin blanc
\ingredient pruneaux
\ingredient huile, beurre, sel, poivre
\end{ingredients}

\begin{preparation}
\etape Faites revenir les morceaux de canard à feu vif dans une sauteuse avec moitié beurre moitié huile d'olive. Une fois bien doré, réservez les.
\etape déglacez avec le cognac, et mettez les échalottes et les champignons dans la sauteuse. Couvrez et laissez mijoter jusqu'à ce que ce soit cuit (en remuant de temps en temps)
\etape rajoutez le bouillon de volaille, le vin blanc, le fond de veau et les pruneaux. Remuez, puis rajoutez les morceaux de canard.
\etape Laissez mijoter 30 minutes environ (ou plus longtemps si les morceaux sont plus gros et plus longs à cuire).
\end{preparation}

\end{recette}

\begin{recette}{Canard laqué}{3}{10 min + 6h}{2h}\index{canard laqué}\index{canard}\index{sauce soja}\index{miel}

\begin{ingredients}
\ingredient 4 cuisses de canard (ou morceaux de canard)
\ingredient 1 cuillère à café de sel
\ingredient 1 cuillère à soupe de poudre aux cinq-épices (ou 5 baies)
\ingredient 6 cuillères à soupe de miel liquide (ou mélasse ou sucre roux)
\ingredient 2 cuillères à soupe de sauce de soja
\ingredient 2 cuillère à soupe de vinaigre blanc
\ingredient 2 cuillères à soupe de vermouth (ou porto) blanc
\ingredient 2 cuillères à soupe de fécule
\ingredient 2 gousses d'ail écrasées et finement hachées
\ingredient 10 g de levure vivante (ou levure chimique)
\end{ingredients}

\begin{preparation}
\etape Plongez les morceaux de canard dans de l'eau bouillante 30 secondes puis lavez et essuyez l'intérieur et l'extérieur avec des serviettes en papier.
\etape En utilisant un poinçon, faites de multiples trous dans la peau et les muscles des morceaux.
\etape Dans un bol, déposez un sac de congélation de taille moyenne. Versez tous les ingrédients de la laque, puis entortillez la poche pour la fermer et remuez jusqu'à obtenir un mélange homogène (c'est le miel le plus difficile à mélanger)
\etape Ajoutez alors les morceaux de canard, et fermez le sac de congélation pour de bon.
\etape Laissez mariner le canard au moins 6 heures, au réfrigérateur de préférence, en le retournant et l'arrosant de laque de temps en temps.
\end{preparation}

\begin{cuisson}
Faites ensuite griller les morceaux de canard au barbecues ou à la rotissoire. En cours de cuisson, arrosez du reste de laque de temps en temps, quand la viande a un peu perdu son enrobage. Au besoin, ajoutez un peu de miel dans la préparation afin de l'épaissir.

Servez chaud ou froid.
\end{cuisson}
\end{recette}

\begin{recette}{Canard à l'orange}{3}{30 min}{}\index{canard à l'orange}\index{orange}\index{vinaigre}

\begin{ingredients}
\ingredient 40 à 50cl de jus d'orange
\ingredient 2 cuillères à soupe de gelée de groseille
\ingredient 2 cuillères à soupe de fécule de pomme de terre
\ingredient 15cl de bouillon de volaille
\ingredient 30g de sucre
\ingredient 15cl de vinaigre de xeres
\end{ingredients}

\begin{preparation}
\etape Préparez le jus d'orange
\etape Faites chauffer le sucre et le vinaigre et laissez réduire jusqu'à la formation d'un caramel et la dissipation des odeurs de vinaigre.
\begin{attention}
Lors de la disparition complète des odeurs de vinaigre, le caramel va commencer à prendre, il faut donc que ça aille vite à ce moment là, afin de ne pas se retrouver avec un vrai caramel très épais. 
\end{attention}

\etape Une fois pris, rajoutez de suite le jus d'orange, les morceaux de carotte et le citron. Laissez cuire 10 à 15 minutes
\etape Pendant ce temps, préparez à peu près 3 cuillères à soupe de fécule de pomme de terre dans 15cl de bouillon.
\etape Mélangez les deux et laissez cuire environ 10 minutes en remuant tout le temps. 
\etape Ajoutez 2 cuillères à soupe de gelée de groseille, et en rajouter si c'est trop acide.
\end{preparation}
\end{recette}

\begin{recette}{Canard aux pêches en cocotte}{4}{30 min}{1h30}\index{canard aux pêches}\index{canard}\index{pêches}

\begin{ingredients}
\ingredient 1 canard prêt à cuire, avec ses abattis à part (cou, ailerons, foie, cœur et gésier)
\ingredient 1 oignon
\ingredient 1 carotte
\ingredient 2 échalotes
\ingredient 2 gousses d'ail
\ingredient 10 cl de bouillon de volaille
\ingredient 100g de champignons de paris
\ingredient 4 pêches au sirop
\ingredient 10 cl du sirop des pêches
\ingredient 3 cuil. à soupe de sucre
\ingredient 1 cuil. à soupe de vinaigre
\ingredient 5cl de Cognac
\ingredient sel et poivre, huile d'olive
\end{ingredients}

\begin{preparation}
\etape Faites chauffer le four à 180°C, th. 6.
\etape Hachez le foie, le cœur et le gésier.
\etape Pelez l'oignon, la carotte et les échalotes, coupez l'oignon et la carotte en tout petits dés et hachez les échalotes. Préparez la gousse d'ail pour la réduire en bouillie (avec une fourchette).
\etape Salez et poivrez le canard à l'intérieur et à l'extérieur, puis piquez la peau régulièrement avec une fourchette.
\etape Mettez l'huile à chauffer dans une cocotte et faites-y revenir le canard 10 min environ, de tous les côtés, puis réservez-le et jetez le surplus de graisse de cuisson.
\etape Faites revenir les champignons dans la cocotte où vous avez fait revenir le canard.
\etape Mettez alors carotte, échalotte, oignon, ail, bouillon de volaille et haché des abats dans la cocotte hors du feu. Mélangez.
\etape Remettez le canard dans la cocotte, les ailerons et le cou. Salez, poivrez.
\etape Mettez la cocotte au four, sans couvercle, et laissez cuire une heure. Ensuite, égouttez le canard, découpez-le en morceaux, disposez ces derniers dans un plat de service et tenez-les au chaud dans le four éteint.
\etape Versez le sucre dans une casserole. Ajoutez une cuillerée à soupe d'eau et faites caraméliser. Dès que c'est un peu épaissi, pas forcément coloré, et que vous ne voyez plus de vapeur d'eau, retirez du feu et arrosez avec le vinaigre.
\etape Posez la cocotte sur feu vif, versez-y le sirop des pêches, les pêches et le caramel vinaigré. Faites bouillir en grattant le fond du récipient pour dissoudre les sucs de viande et rectifiez l'assaisonnement au besoin.
\etape Retirez du feu, ajoutez le Cognac et mélangez.
\etape Disposez les pêches autour du canard, arrosez avec la sauce et servez aussitôt.
\end{preparation}
\end{recette}


\begin{recette}{Cassoulet}{0}{10h}{1h30}\index{cassoulet}
\begin{ingredients}[7 pers.]
\ingredient $750$ g de haricots secs (lingots ou tarbais)
\ingredient $400$ g de saucisse de toulouse
\ingredient $300$ g d'échine de porc
\ingredient 1 ou 2 gésiers confits
\ingredient 1 morceaux de vieux jambon
\ingredient 4 ou 5 morceaux de confit de canard
\ingredient 1 tête d'ail d'entière
\ingredient 1 grosse pomme de terre
\ingredient 3 ou 4 cuillères de graisse d'oie ou 2 ou 3 morceaux de couenne
\ingredient 3 cuillères de chapelure
\ingredient pour préparer un cassoulet toulousain, il suffit de rajouter du mouton
\end{ingredients}

\begin{preparation}
\etape Mettre les haricots à tremper dans de l'eau froide pendant 5 ou 6 heures
\etape Les égoutters, les couvrir largement d'eau froide non salée et les faire blanchir $30$ minutes à feu moyen. Eteindre le feu et laisser gonfler un moment.
\etape Faire blanchir la couenne de porc dans de l'eau non salée pendant $15$ minutes.
\etape Pendant ce temps, dans une sauteuse, faire dorer dans de la graisse d'oie, les saucisses et la viande de porc coupées en morceaux. Ajouter l'oignon émincé et laisser dorer.
\etape Ajouter les gésiers confits, le jambon et la couenne de porc coupée en morceaux puis enfin l'ail.
\etape Mouiller avec de l'eau chaude. Amener à l'ébullition, puis ajouter les haricots blanchis. Saler et poivrer. Les haricots doivent être largement couverts d'eau. Couvrir et laisser cuire à feu assez vif au début puis très doux à la fin pendant 3 heures environ. secouer la cocotte de temps en temps, mais ne remuez jamais en cours de cuisson, surtout à la fin. Le temps de cuisson peut varier selon la variété des haricots. Bien surveiller et ajouter éventuellement de l'eau bouillante si l'évaporation vous paraît trop importante. les haricots ne doivent pas "nager" mais il ne doivent pas être trop secs non plus. En fin de cuisson, le jus de cuisson doit devenir crémeux. Pour qu'il soit très crémeux, on peut ajouter une pomme de terre grossièrement rapée en même temps que les haricots. En fin de cuisson, rectifier l'assaisonnement.
\etape Frotter la casserole avec un grain d'ail. Y verser délicatement la moitié des haricots. ajouter les morceaux de viance, intercaler quelques morceaux de confit et verser le reste des haricots, en enfouissant la viande au maximum. Ajouer une bonne cuillère de graisse d'oie fondue et saupoudrer de chapelure.
\end{preparation}

\begin{cuisson}
Mettre à gratiner au four préchauffé à thermostat 4 sur la grille du milieu pendant 1h30 environ. La croute qui se forme en surface peut être enfoncée délicatemen. en fin de cuisson, le jus doit avoir la consistance d'une crème épaisse.
\end{cuisson}
\end{recette}

\begin{recette}{Choucroute}{3}{30 min}{2h}\index{choucroute}
\begin{ingredients}[6 pers.]
\ingredient 2.5 kg de choucroute
\ingredient 2 oignons
\ingredient 1 ou 2 échalotes
\ingredient 2 gousses d'ail
\ingredient 1 palette fumée
\ingredient 200g de lard fumé (ou lardons)
\ingredient 6 saucisses de Strasbourg
\ingredient 2 saucisses de Montbéliard
\ingredient (700g de pommes de terre)
\ingredient 25cl de vin blanc
\ingredient 50cl de bouillon
\ingredient 1 dizaine de baies de genièvre
\ingredient 2 cuillères à café de baies de coriandre
\ingredient 1 cuillère à café de cumin (ou carvi)
\ingredient 2 clous de girofle
\ingredient sel, poivre
\begin{remarque}
Il est possible de mettre une saucisse de Morteau, mais je trouve que le gout est similaire aux saucisses de Montbéliard en beaucoup plus cher.
\end{remarque}

\end{ingredients}
\begin{remarque}
Il est possible de mettre votre petit salé dans un gros volume d'eau la veille.
\end{remarque}
\begin{preparation}
\etape Lavez la choucroute plusieurs fois jusqu'à ce que l'eau de trempage soit claire. Égouttez-la et essorez-la bien en la pressant entre les mains puis démêlez-la avec les doigts.
\etape Émincez les oignons et les échalotes, hachez l'ail finement. 
\etape Faites chauffer l'huile dans une marmite chaude. Faites blondir oignons, ail et échalottes.
\etape Dans un grand récipient, mélangez cette préparation à la choucroute, Ajoutez les baies et épices (cumin, clous de girofle, poivre, sel). 
\etape Réduisez le feu et étalez au fond le lard. Recouvrez le lard avec la moitié de la choucroute. Rajoutez la palette au milieu de ce nid de choucroute. Recouvrez enfin du reste de chou, baies et épices. Ajoutez enfin les saucisses de Montbéliard et les pommes de terres lavées sous l'eau (coupez-les en deux ou quatre si elles sont grosses).
\etape Ajoutez alors le vin blanc et le bouillon et laissez mijoter à feu doux pendant deux heures à couvert. 
\etape 10 minutes avant la fin de la cuisson rajoutez les saucisses de Strasbourg.
\end{preparation}
\end{recette}

\begin{recette}{Couscous}{4}{}{}\index{couscous}\index{mouton}
\begin{ingredients}[8 pers.]
\ingredient $1\unit{kg}$ de semoule moyenne
\ingredient $1\unit{kg}$ de mouton (collier ou plat de côtes)
\ingredient $4$ oignons
\ingredient $70\unit{g}$ de concentré de tomate en boite
\ingredient $2$ gousses d'ail
\ingredient $100\unit{g}$ de raisins de smyrne
\ingredient $1$ tasse d'huile d'olive
\ingredient $1\unit{kg} 200$ de poulet (un peu ferme)\footnote{Choisissez un poulet pas trop tendre sinon il se déferait dans le bouillon. Une petite poule peut faire l'affaire.}
\ingredient $2$ carottes
\ingredient $2$ navets
\ingredient $4$ courgettes
\ingredient $2$ tomates
\ingredient $1$ petite boite de pois chiches en conserve
\ingredient $1$ boite de piments doux (morones)
\ingredient épices : \begin{itemize}
		\item une ou deux cuillères à café de Ras-el-hanout
		\item une cuillerée à café de Cumin arabe (kamoun)
		\item une petite boite de Arissa (sauce forte)
		\item $\sfrac{1}{2}\unit{g}$ de safran
		\end{itemize}
\ingredient $125\unit{g}$ de beurre
\ingredient $\sfrac{1}{2}$ boite de petits pois en conserve
\ingredient sel, poivre
\ingredient un couscoussier
\end{ingredients}

\begin{preparation}
\etape Bouillon : Dans la marmite à couscous, mettez $2$ litres d'eau environ avec la viande de mouton, oignons, safran, sel, poivre, huile d'olive, concentré de tomate, une cuillerée à café d'harissa, ail. Couvrez. Laissez cuire ce bouillon en tout $2$ heures.
\etape Versez la semoule à couscous dans une grande bassine. Humectez-la, en plusieurs fois, avec de l'eau froide salée jusqu'à ce qu'elle en soit saturée ($\sfrac{2}{3}$ de litre environ). Aspergez aussi d'un peu d'huile. Égrenez avec une frouchette. Laissez gonfler le temps indiqué sur le paquet.
\etape Quand le bouillon à déjà cuit $\sfrac{1}{2}$ heure, ajoutez-y les carottes et navets fendus en deux, ainsi que le poulet.
\etape Versez la semoule dans la passoire du couscoussier. Posez-la au-dessus du bouillon. Couvrez d'un torchon seulement. Laissez cuire $30$ minutes environ.
\etape Au bout de ce temps, reversez la semoule dans un torchon, aspergez-la abondamment d'eau froide salée. Aérez-la avec une fourchette.
\etape Ajoutez au bouillon resté sur le feu : courgette non épluchées et tomates coupées, une ou deux cuillerées à café de ras-el-hanout, le kamoun. Remettez la semoule dans la passoire, au-dessus du bouillon. laissez cuire à nouveau, couvert d'un torchon, pendant $\sfrac{1}{2}$ heure
\etape Versez un peu de bouillon dans une casserole. Mettez-y les petits pois, les raisins préalablement lavés, les pois chiches égouttés, les piments doux et plus ou moins d'arissa pour pimenter. Mettez sur feu doux jusqu'à frémissement.
\etape Versez enfin le couscous dans un très grand plat creux. Incorporez-y $125\unit{g}$ de beurre ou de margarine par petits morceaux. Mettez le couscous en dôme. Formez un creux au centre pour y verser la viande coupée en morceaux et les légumes. À part, présentez le bouillon et un petit récipient d'arissa (sauce forte). Chaque convive arrosera son couscous de bouillon et l'épicera à son gré.
\end{preparation}

\end{recette}

\begin{recette}{Crépinettes de canard aux raisins}{0}{}{}\index{crépinettes de canard aux 
raisins}\index{raisins}\index{canard}\index{crépinettes}\index{paupiettes}
\begin{ingredients}
\ingredient 6 crépinettes
\ingredient 200g de champignons
\ingredient 100g de lardons
\ingredient 100g de raisins
\ingredient 1 cuillère à soupe rase de farine
\ingredient 10cl de vin blanc
\ingredient 5cl de calvados
\ingredient 20cl de bouillon de volaille
\ingredient sel, poivre, beurre
\end{ingredients}

\begin{preparation}
\etape Mettre les raisins à tremper dans le bouillon de volaille et 5cl de calvados puis commencez la recette.
\etape Faites revenir les lardons puis réservez les.
\etape Ajoutez un peu de beurre et faites saisir les crépinettes sur toutes les faces (pas besoin qu'elles soient cuites).
\etape Réservez les crépinettes et ajoutez les chamignons finement émincés, puis poivrez.
\etape Une fois bien revenus, ajoutez la farine et mélangez bien. Ajoutez le calvados, le vin blanc, le fond de veau et les raisins. Puis une fois mélangé, rajoutez les lardons et les crépinettes.
\etape Laissez mijoter à feu doux à couvert pendant une heure environ.
\end{preparation}
\end{recette}

\begin{recette}{Crépinettes en sauce}{3}{}{}\index{crépinettes en sauce}\index{crépinettes}\index{paupiettes}

\begin{ingredients}
\ingredient 1 gousse d'ail
\ingredient 1 oignon
\ingredient 100g de champignons
\ingredient 4 crépinettes
\ingredient 1 verre de vin blanc sec
\ingredient 1 cuillère à café de fond de veau
\ingredient huile, herbes de provence, cognac
\end{ingredients}

\begin{preparation}
\etape Faire revenir les crépinettes dans l'huile chaude pour qu'elles soient dorées, puis les sortir.
\etape Déglacez les sucs avec un peu de cognac, puis faites revenir les champignons.
\etape Une fois fait, réservez les avec les crépinettes et faites cuire les oignons et l'ail jusqu'à ce qu'ils soient bien dorés ; au besoin, rajoutez un peu d'huile.
\etape Remettre les crépinettes, les champignons et ajouter le vin blanc, un verre d'eau et le fond de veau. Salez, poivrez et mettez les herbes.
\etape Couvrez et laissez cuire à feu doux pendant une heure. Remuez de temps en temps.
\end{preparation}

\begin{remarque}
Cette recette marche très bien avec des paupiettes. Elle est d'ailleurs relativement proche de la recette du lapin en gibelote (qui s'adapte lui aussi pour les paupiettes)
\end{remarque}

\end{recette}

\begin{recette}{Croque monsieur}{3}{}{}\index{croque monsieur}
\begin{ingredients}
\ingredient 24 tranches de pain de mie
\ingredient fromage en tranche
\begin{remarque}
J'achète un morceau d'emmental de 500g, et je fais des tranches. Avec deux tranches sur la largeur je fais une surface de pain de mie.
\end{remarque}

\ingredient 4 tranches de jambon blanc
\begin{remarque}
C'est aussi très bon si on remplace le jambon par du saumon ou de la charcuterie diverse.
\end{remarque}

\ingredient beurre (comptez 5g par tranche si vous comptez le faire fondre, donc 120 grammes pour les 24 tranches)
\begin{attention}
Utilisez de préférence du beurre classique (à 80\% de matière grasse), et non du beurre allégé. Je l'ai fait avec du beurre à 40\% et les croques monsieurs accrochaient.
\end{attention}

\ingredient poivre
\end{ingredients}

\begin{preparation}
\etape Beurrez un coté du pain de mie
\etape disposez le coté beurré à l'extérieur (il sera en contact avec la partie chaude)
\etape disposez une couche de fromage, une couche de jambon, une pincée de poivre, puis une autre couche de fromage et enfin une tranche de pain de mie, coté beurré à l'extérieur
\etape faites cuire dans un appareil pour les croque-monsieurs (ou au four le cas échéant)
\end{preparation}

\end{recette}

\begin{recette}{Escalopes à la milanaise}{3}{}{}\index{escalopes à la milanaises}\index{escalopes}\index{veau}
\begin{ingredients}
\ingredient 2 escalopes de veau
\ingredient 30g de parmesan rapé
\ingredient 30g de chapelure
\ingredient 1 œuf
\end{ingredients}

\begin{preparation}
\etape Prenez deux assiettes. Dans l'une d'elle, on mélange chapelure et parmesan. Dans l'autre on bat l'œuf
\etape On trempe les deux faces des escalopes d'abord dans l'œuf puis dans le mélange chapelure/parmesan
\etape Faites ensuite cuire les escalopes dans un peu de matière grasse.
\end{preparation}
\end{recette}

\begin{recette}{Filet mignon de porc aux champignons}{4}{}{}\index{filet mignon de porc aux champignons}\index{filet 
mignon}\index{porc}
\begin{ingredients}
\ingredient 2 filets mignons de porc
\ingredient 100g de champignons
\ingredient 20cl de bouillon (eau + bouillon-cube par défaut)
\ingredient 20cl de fond de veau (2 cuillères à café de fond de veau dans de l'eau)
\ingredient 5cl de porto
\ingredient 20cl de crème fraîche
\ingredient persil
\ingredient sel, poivre, beurre
\end{ingredients}

\begin{preparation}
\etape Faites chauffer la sauteuse puis saisissez les 4 faces des filets mignons à feu vif. Ajoutez ensuite le bouillon et laissez cuire à couvert pendant 20 minutes à feu moyen.
\etape Pendant ce temps, émincez les champignons et le persil.
\etape Au bout des 20 minutes, retirez les filets mignons de la sauteuse et réservez-les au chaud (papier d'alu + papier journal autour). Dans la sauteuse, déglacez les sucs de cuisson avec le porto et laissez réduire de moitié.
\etape Ajoutez ensuite les champignons, le persil et le fond de veau et laissez à nouveau réduire de moitié.
\etape Ajoutez enfin la crème fraîche et laissez réduire jusqu'à obtenir la consistance que vous souhaitez (quand même un peu épais).
\etape À la toute fin, juste avant de servir, ajoutez à la sauce le jus qu'auront rendu les filets mignons, laissez mijoter quelques instants en remuant pour que la sauce soit homogène et à votre convenance.
\end{preparation}

\end{recette}

\begin{recette}{Filet mignon de porc au bleu}{2}{}{}\index{filet mignon de porc au bleu}\index{filet 
mignon}\index{porc}\index{roquefort}\index{bleu}
\begin{ingredients}
\ingredient 2 filets mignons de porc
\ingredient 3 petites echalotte emincées
\ingredient 10 cl de porto (ou un autre vin cuit)
\ingredient 25 cl de bouillon de boeuf
\ingredient 100g de bleu de bresse.
\ingredient 4 cuillères à café de poudre d'amande (à garder ? ; c'est à mettre avec la crème)
\end{ingredients}

\begin{preparation}
\etape Salez les filets mignons. Dans une poêle, faites fondre le beurre et colorez les filets mignons sur toutes les faces à feux vif. Sortez les filets et gardez-les dans un endroit tiède.
\etape Faites dorer les échalotes à feu doux. Pendant ce temps, dans un mixeur, mélangez la crème, le fromage, et éventuellement la poudre d'amande.
\etape Déglacez avec le porto et laisser réduire de moitié. Ajouter ensuite le bouillon et le mélange du mixeur. Remuez doucement.
\etape Laissez mijoter pour faire réduire et homogénéiser la sauce, ne pas couvrir totalement. 
\etape Une fois la consistance plus agréable, rajoutez les filets mignons et laissez mijoter à feu doux pendant 10 minutes environ.
\end{preparation}

\begin{remarque}
En accompagnement, des tagliatelles vont très bien.
\end{remarque}
\end{recette}

\begin{recette}{Gigot d'agneau fondant en cocotte}{3}{30 min}{5 heures}\index{gigot d'agneau fondant en 
cocotte}\index{gigot}\index{agneau}
\begin{ingredients}
\ingredient un gigot ou épaule d'agneau
\ingredient 4 ou 5 gousses d'ail
\ingredient 2 gros oignons
\ingredient 1 carotte
\ingredient 15cl de vin blanc sec
\ingredient 15cl de bouillon
\ingredient 1 cuillère à soupe de miel
\ingredient sel, poivre, thym, romarin
\end{ingredients}


\begin{preparation}
\etape Coupez la carotte en lamelle puis cubes grossiers et les oignons en quatre.
\etape Dans la cocotte préalablement huilée, saisissez le gigot de tous les cotés jusqu'à avoir une jolie croûte dorée.
\begin{remarque}
Suivant la taille de votre cocotte, il vous faudra peut-être couper l'os situé à l'extrémité, vers la souris du gigot, ne jetez pas ce bout d'os, mettez-le au fond de la cocotte, cela apportera encore plus de goût.
\end{remarque}
\etape Réservez le gigot puis faites fondre l'oignon et la carotte dans les sucs pendant 5 minute environ
\etape Déglacez avec le vin blanc. Ajoutez le bouillon, le miel et les herbes. Faites chauffer un peu afin que ce soit homogène, puis ajoutez le gigot préalablement salé et poivré. 
\end{preparation}

\begin{cuisson}
Faites cuire le gigot dans la cocotte fermée pendant 5 heures à 120°C. Toutes les heures environ, sortez la cocotte pour retourner le gigot afin qu'il baigne uniformément dans la sauce.

Je conseille de servir ce plat avec des pommes de terre au four (\refsec{sec:pomme-de-terre-four}). Il est aussi possible d'épaissir un peu la sauce à la fin de la cuisson du gigot. 
\end{cuisson}

\end{recette}

\begin{recette}{Grattin Dauphinois}{3}{}{}\index{gratin dauphinois}\index{pomme de terre}
\begin{ingredients}
\ingredient $800$ g de pommes de terre
\ingredient $25$ cl de lait entier
\ingredient $30$ cl de crème fraîche
\ingredient sel
\ingredient poivre
\ingredient noix de muscade
\ingredient $1$ grosse noix de beurre
\ingredient $3$ gousses d'ail
\end{ingredients}

\begin{preparation}
\etape Laver, éplucher et émincer les pommes de terre en tranches de $3$ mm environ.\footnote{Ne pas les laver après la coupe.}
\etape Les disposer dans une casserole avec $25$ cl de lait (entier si possible), une grosse noix de beurre, sel, poivre et muscade.
\etape Porter à ébullition puis baisser le feu légèrement et poursuivre la cuisson une dizaine de minutes.\footnote{Remuer de temps en temps avec une spatule pour éviter que la préparation attache.}
\etape Quand les pommes de terres s'enrobent d'une sorte de crème, verser à ce moment $30$ cl de crème.
\etape Laisser cuire à petit feu pendant une dizaine de minutes environ.
\etape Retirer du feu, ajouter l'ail.
\etape Disposer délicatement les pommes de terre dans un plat à gratin.
\etape Aplanir la surface et laisser refroidir pour que les goûts se mélangent.
\end{preparation}

\begin{cuisson}
Enfourner à $180\degres$ et laisser cuire entre $20$ et $30$ minutes. Servir dans le plat de cuisson.
\end{cuisson}
\end{recette}

\begin{recette}{Lapin à la moutarde}{5}{30 min}{50 min}\index{lapin à la moutarde}\index{moutarde}\index{lapin}\index{poulet}
\begin{ingredients}[6 pers.]
\ingredient 1 lapin coupé en morceaux
\ingredient 4 échalotes
\ingredient 20cl de bouillon de volaille
\ingredient 3 cuillères à soupe rase de moutarde à l'ancienne (il n'en faut pas moins, sinon il y a trop de crème ; si la cuillère est petite, ne pas hésiter à en mettre 4, ou 3 non rase)
\ingredient 50 cl de crème fraiche épaisse
\ingredient sel, poivre, romarin
\end{ingredients}

\begin{preparation}
\etape Verser l'huile dans une cocotte et y faire fondre le beurre, puis saisir à feu vif les morceaux de lapin des deux côtés avant de les réserver.
\etape Faites revenir les échalotes pelées et émincées, en remuant jusqu'à ce qu'elles soient dorées.
\etape Mouillez avec le bouillon, saler, poivrer. Remettre les morceaux de viande, couvrir et laissez mijoter pendant 30 min.
\etape Enlevez les morceaux de viande afin de pouvoir bien remuer puis ajoutez la moutarde, la crème et le romarin. Mélangez jusqu'à ce que la crème et la moutarde fassent une mixture homogène (plus de grumeau de moutarde ou de crème). 
\etape Rectifiez l'assaisonnement, remettez les morceaux de viande et laissez mijoter à couvert et à feu doux pendant 20 minutes environ.
\end{preparation}
\end{recette}

\begin{recette}{Lapin à la tomate}{5}{1h}{1h}\index{lapin à la tomate}\index{lapin}\index{poulet}
\begin{ingredients}[4 pers.]
\ingredient un lapin
\ingredient un oignon
\ingredient 1 ou 2 carottes
\ingredient 100 ou 200g de lardons fumés
\ingredient 150 à 200g de champignons
\ingredient un cube de volaille et 20cl d'eau
\ingredient 20cl de vin blanc (un verre)
\ingredient une cuillère à soupe rase de farine
\ingredient une boîte de coulis de tomate (entre 200 et 500g, la quantité exacte importe peu)
\ingredient sel, poivre, herbes de provence
\end{ingredients}

\begin{preparation}
\etape Faites bien dorer les morceaux de lapin dans du beurre (et un peu d'huile) ; en plusieurs fois s'il n'y a pas de place dans la cocotte (attention, ça éclabousse!).
\etape Réservez les morceaux de lapin dans une assiette.
\etape Faites revenir l'oignon émincé et les carottes coupés en petits morceaux (rajoutez un peu d'huile si besoin). Réservez.
\begin{remarque}
Pendant ce temps, je met le bouillon cube et l'eau dans un bol que je fais chauffer au micro-onde, puis je mélange avec une fourchette quand c'est chaud.
\end{remarque}
\etape Rajoutez ensuite les lardons, faites revenir,. Réservez les lardons et conservez la graisse. 
\etape Ajoutez alors les champignons, faites les revenir, puis ajoutez les lardons, oignon et carotte.
\etape Saupoudrez alors le tout avec la farine et mélangez. 
\etape Mouillez ensuite avec un verre de vin blanc, le coulis de tomate et le bouillon préalablement préparé. Ajoutez les herbes de provence et mélangez.
\etape Rajoutez les morceaux de lapin dans la cocotte et remuez-les un peu dans la sauce.
\end{preparation}

\begin{cuisson}
Couvrez et laissez cuire à feu très doux 1h en mélangeant de temps en temps. Ajoutez sel et poivre en fin de cuisson.
\begin{remarque}
Les lardons salent déjà pas mal la sauce, je ne la resale quasiment jamais. Par contre je poivre avant de faire mijoter une heure.
\end{remarque}
\end{cuisson}
\end{recette}

\begin{recette}{Lapin en gibelote}{4}{}{}\index{paupiettes}\index{lapin}\index{gibelote}\index{poulet}
\begin{ingredients}
\ingredient un lapin
\ingredient 100 g de champignons
\ingredient deux ou trois oignons
\ingredient 25 cl de vin blanc sec
\ingredient 25 cl de bouillon (1 bouillon cube de volaille)
\ingredient $100\unit{g}$ de lardons
\ingredient 1 cuillère à soupe rase de farine
\ingredient 2 cuillères à café de fond de veau
\ingredient sel, poivre (un sachet d'arômes).
\end{ingredients}

\begin{preparation}
\etape Faire revenir les lardons (réservez), puis les champignons (réservez), et enfin les oignons (réservez).
\etape Découper le lapin et faire dorer les morceaux dans de l'huile d'olive (penser à laisser un peu plus longtemps les cuisses qui ont plus de viande)
\etape Réserver les morceaux
\etape Dans les sucs, mettez une cuillère à soupe rase de farine. Laissez roussir, puis diluez avec un peu du vin blanc.
\etape Ajoutez alors le reste de vin blanc, le bouillon, 2 cuillères à soupe de fond de veau, les lardons, oignons et champignons. Remuez pour diluer le fond de veau.
\etape Arômatisez selon votre gout.
\end{preparation}

\begin{cuisson}
Faire cuire à feu doux pendant 1h30.

\begin{remarque}
C'est aussi excellent avec des paupiettes de veau.

Dans ce cas, à la fin de la cuisson, stockez séparement les paupiettes et la sauce, pour pouvoir dégraisser la sauce une fois froide.
\end{remarque}
\end{cuisson}
\end{recette}

\begin{recette}{Lentilles}{3}{}{}\index{lentilles}
\begin{ingredients}
\ingredient 500g de lentilles
\ingredient un oignon
\ingredient une demi carotte
\ingredient deux gousses d'ail
\ingredient 6 saucisses
\ingredient poitrine demi-sel
\ingredient cube de bouillon de volaille
\ingredient laurier sauce, herbes de provence, sel, poivre
\end{ingredients}

\begin{preparation}
\etape Découpez finement la carotte et l'ail.
\etape Ajoutez l'huile dans une sauteuse et passez brièvement le petit salé du coté de la couenne.
\etape Réservez le petit salé et  et colorer les saucisses sur toutes les faces.
\etape Retirez les saucisses de la cocotte.
\etape Ajoutez les légumes dans la cocotte et les faire revenir doucement à l'huile.
\etape Ajoutez les lentilles. Mouillez à hauteur avec de l'eau et le cube de bouillon de volaille, rajoutez le petit salé et laissez cuire 30 minutes à feu doux.
\etape Ajoutez les saucisses et les faire mijoter une dizaine de minutes dans les lentilles cuites.
\end{preparation}

\end{recette}

\begin{recette}{Magret de canard aux poires}{4}{}{}\index{magret de canard aux poires}\index{magret}\index{canard}\index{poires}
\begin{ingredients}
\ingredient $2$ magrets de canard
\ingredient $2$ échalotes
\ingredient 20g sucre en poudre
\ingredient  $10 \unit{cl}$ de vin rouge
\ingredient  $10 \unit{cl}$ de floc de Gascogne
\ingredient  $15 \unit{cl}$ de bouillon de volaille ou de bœuf
\ingredient  une grosse poire comice ou deux petites.
\ingredient une cuillère à soupe de farine
\ingredient  Sel, poivre, jus de citron

\end{ingredients}

\begin{preparation}
\etape Poser les magrets côté peau dans une sauteuse chauffée à vif afin de graisser un peu celle-ci. Retirez-les puis faites les cuires au grill.
\etape Faites suer les échalotes ciselées mais ne les laissez pas cuire trop longtemps, il ne faut pas qu'elles soient dorées. 
\etape Saupoudrer du sucre sur les échalottes une fois qu'elle sont revenues et laisser caraméliser. 
\etape Ajouter ensuite la farine puis remuez afin que ça soit homogène.
\etape Déglacer ensuite avec le Floc et le vin rouge. 
\etape Faire réduire à découvert quelques minutes. Préparez pendant ce temps la poire, que vous coupez en petit cubes d'un centimètre environ.
\etape Ajouter le bouillon et un tout petit peu de jus de citron. Remuez pour que ce soit homogène. Ajoutez enfin les cubes de poire et laissez mijoter 5 minutes à couvert.
\begin{remarque}
Faites attention à remuer doucement la sauce une fois les poires ajoutées, afin de ne pas trop exploser les morceaux.
\end{remarque}
\etape Dresser les magrets escalopes en éventail et napper les pointes en sauce.
\etape Le magret doit être servi rosé.
\end{preparation}

\end{recette}

\begin{recette}{Paëlla (à la moi)}{3}{45 min}{30 min}\index{paëlla}\index{fruits de mer}\index{chorizo}
\begin{ingredients}
\ingredient 400g de riz (qui se cuit en 20 minutes ou plus, ne pas utiliser du riz cuisson rapide)
\ingredient 200g de chorizo
\ingredient 200g de rondelles de calamar
\ingredient 8 morceaux de poulets (4 cuisse + contre cuisse par exemple)
\ingredient 500g de fruits de mers
\ingredient 1 gros oignon
\ingredient 1 poivron
\ingredient 1 gousse d'ail
\ingredient 1 grosse boite de tomate en dés
\ingredient 100g de petits pois
\ingredient 1 dl de vin blanc sec
\ingredient un bol de bouillon (dépend)
\ingredient 1 dose de safran (je crois que c'est 1g)
\ingredient huile d'olive, beurre, sel, poivre, thym, laurier.
\end{ingredients}

\begin{preparation*}
Pensez à lire la recette en entier. Il faudra notamment couper les poivrons, l'oignon, le chorizo. Faites ça pendant la cuisson des autres éléments. N'utilisez pas plus de 400g de riz si vous utilisez une sauteuse conventionnelle, sinon ça ne rentrera pas dedans.

\begin{tabular}{p{0.45\textwidth}|p{0.45\textwidth}}
Partie viande (la sauteuse)& Partie poisson (la casserole)\\\hline
\begin{enumerate}
\item Faites revenir dans une grande sauteuse les morceaux de viande dans un peu d'huile à feu vif afin de les faire dorer (mais pas cuits). Une fois prêts réservez-les.
\item Faites fondre doucement l'oignon dans la sauteuse (rajoutez de l'huile si nécessaire). Une fois qu'ils deviennent translucides, rajoutez le poivron et laissez revenir à couvert (pour ne pas perdre trop d'eau et que ça n'accroche pas ; sinon, rajoutez de l'eau si besoin)
\item Une fois que poivrons et oignons sont prêts, réservez les avec la viande.
\end{enumerate}&\begin{enumerate}
\item Pendant ce temps, dans une casserole (ou autre) mettez le beurre et le vin blanc, puis mettez les fruits de mers à cuire là dedans.
\item Filtrez le jus des fruits de mer, réservez les fruits de mer avec la viande. 
\item Dans la casserole, ajoutez les tomates, thym, laurier, une gousse d'ail coupée fin. Ajoutez le jus des fruits de mer et le bouillon cube (sans eau supplémentaire). Laissez mijoter quelques minutes le temps que les saveurs se mélangent.
\end{enumerate}
\end{tabular}
\begin{enumerate}
\item Versez les 400g de riz dans un récipient (saladier) où vous pourrez facilement repérer le volume que ça représente. 
\item Faites revenir brièvement le chorizo et réservez-le avec la viande avant qu'il ne soit trop cuit. 
\item Dans la graisse rendue par le chorizo faites rissoler le riz jusqu'à ce qu'il devienne translucide (entre 2 et 5 minutes). Rajoutez de l'huile si besoin.
\item Dans le saladier où vous avez repéré le volume du riz, versez la préparation de la casserole, et en complétant avec de l'eau afin de préparer l'équivalent en liquide de deux fois le volume du riz
\item Ajoutez ensuite le liquide ainsi mesuré dans le récipient du riz. Ajoutez de même \textbf{le safran}, oignon/poivron, le chorizo, la viande, le poisson \textbf{et les petits pois}.
\end{enumerate}
\end{preparation*}

\begin{cuisson}
Remuez et laissez cuire 20 à 25 minutes à couvert à feu moyen, le temps que le riz soit cuit (il doit absorber tout le liquide). C'est à peu près le temps qu'il faut pour que le poulet soit cuit. 
\end{cuisson}


\end{recette}

\begin{recette}{Pâtes à la bolognaise (à la moi)}{4}{1h}{1h}\index{pâtes}\index{pâtes à la bolognaise}\index{bolognaise}
\begin{ingredients}
\ingredient 400g de viande hachée
\ingredient 100g de lardons
\ingredient deux oignons
\ingredient une gousse d'ail
\ingredient 1 ou 2 carottes
\ingredient un cube de bœuf (ou de volaille) et 20cl d'eau
\ingredient 20cl de vin blanc (un verre)
\ingredient une cuillère à soupe (rase) de farine
\ingredient 500g de purée de tomate
\ingredient sel, poivre, herbes de provence, sucre
\end{ingredients}

\begin{preparation}
\etape Faites revenir à feu vif les lardons puis réservez-les.
\etape Faites revenir la viande hachée émiettée et rajoutez un peu d'huile au besoin (en plus du gras des lardons). Pas besoin que la viande soit parfaitement cuite, c'est pour faire dorer la viande un peu.
\etape Réservez la viande.
\etape Faites revenir l'oignon émincé, l'ail et les carottes en petits morceaux (plutôt que des tranches ; typiquement des $\sfrac{1}{4}$ de tranche).
\begin{remarque}
Pendant ce temps, je met le bouillon cube et l'eau dans un bol que je fais chauffer au micro-onde, puis je mélange avec une fourchette quand c'est chaud.
\end{remarque}
\etape Une fois fait, saupoudrez le tout de farine et mélangez. Mouillez ensuite avec un verre de vin blanc et le bouillon préalablement préparé. Mélangez et ajoutez les herbes de provence.
\etape Ajoutez la boîte de coulis de tomate et une pincée de sucre puis mélangez.
\etape Rajoutez enfin la viande et mélangez.
\end{preparation}

\begin{cuisson}
Couvrir et laisser cuire à feu très doux 1h. Ajoutez sel et poivre en fin de cuisson.
\begin{remarque}
Le vin blanc (surtout bas de gamme) et la tomate peuvent acidifier la sauce. Afin de corriger ça, il faut rajouter un peu de sucre.
\end{remarque}
\end{cuisson}
\end{recette}

\begin{recette}{Pâtes à la carbonara}{4}{}{}\index{pâtes}\index{pâtes à la carbonara}\index{carbonara}
\begin{ingredients}[Pour 500g de pâtes]
\ingredient 200g de lardons
\ingredient 2 oignons
\ingredient 50 à 100g de parmesan
\ingredient 10 à 15 cl de crème liquide légère (je met une petite brique généralement)
\ingredient un jaune d'œuf (il m'arrive soit de mettre l'œuf entier, soit de ne pas en mettre)
\ingredient herbes aromatiques (herbes de provence, romarin,\dots)
\ingredient sel, poivre
\end{ingredients}

\begin{preparation}
\etape Faites cuire les lardons
\etape Une fois cuits, réservez-les puis faites rissoler l'oignon afin qu'il soit bien doré.
\etape Dans un bol, mélangez l'œuf, le parmesan, la crème liquide et les herbes aromatiques. Salez et poivrez. Pas besoin de saler beaucoup, les lardons l'étant déjà, mais pour le poivre, vous pouvez en mettre à votre convenance.
\etape Une fois les pâtes cuites, rajoutez dans le plat le contenu du bol ainsi que les lardons. Remuez ensuite abondament jusqu'à ce que ça ait la consistance qui vous convienne (avec la chaleur des pâtes, l'œuf, la crème liquide, le parmesan et l'éventuel reste d'eau de cuisson vont faire une sauce onctueuse).
\end{preparation}

\end{recette}

\begin{recette}{Pâtes au boudin}{2}{10 min}{}\index{pâtes}\index{boudin}
\begin{ingredients}
\ingredient 200 g Boudin à l'oignon
\ingredient 500g de pâtes
\end{ingredients}

\begin{preparation}
\etape Enlevez la peau au boudin, et coupez le en morceaux grossiers
\etape Mettez ces morceaux dans une poële à feu moyen
\etape Avec une spatule en bois, remuez et écrasez un peu les morceaux pour qu'ils se désolidarisent et finissent par former une sorte de bouilli peu agréable à l'œil.
\etape Ajoutez ça aux pâtes et remuez pour que ça soit homogène.
\end{preparation}

\end{recette}

\begin{recette}{Pâtes au saumon (à la moi)}{2}{}{}\index{pâtes}\index{saumon}
\begin{ingredients}
\ingredient 25cl de crème fraîche (liquide ou épaisse)
\ingredient 1 fenouil
\ingredient 200g de saumon fumé
\ingredient sel, poivre, aneth
\end{ingredients}

\begin{preparation}
\etape Émincez le fenouil (comme un oignon). Coupez le saumon fumé en lamelles puis en morceaux
\etape Faites revenir le fenouil émincé dans un peu d'huile pendant 10 minutes environ puis sortez du feu en
laissant couvert après avoir remué.
\etape Pendant ce temps, faites bouillir l'eau des pâtes. 
\etape au moment de lancer la cuisson des pâtes, ajoutez la crème fraiche liquide et l'aneth. Laissez mijoter à couvert et à
feu doux.
\etape Une fois les pâtes cuites, éteignez la sauce, sortez le couvercle et remuez. 
\etape Égoutez les pâtes puis servez les. 
\etape Ajoutez alors le saumon dans la sauce, puis versez la sauce sur les pâtes
\end{preparation}

\end{recette}

\begin{recette}{Paupiettes de veau à la Normande}{4}{1h30}{}\index{paupiettes}\index{veau}\index{pommeau de normandie}
\begin{ingredients}
\ingredient 8 paupiettes de veau
\ingredient 200g de champignons
\ingredient 4 oignons
\ingredient 1 pomme assez ferme
\ingredient 200g de lardons
\ingredient 25cl de pommeau de Normandie
\ingredient 25cl de bouillon de Volaille
\ingredient sel, poivre, sucre, fond de veau, farine
\end{ingredients}

\begin{preparation}
\etape Faites revenir les paupiettes de veau à feu vif, d'abord sur la partie gras, puis sur les autres faces. 
\etape Réservez les paupiettes et mettez les champignons.
\etape Une fois les champignons revenus, réservez les dans un récipient différent des paupiettes
\etape Faites cuire les lardons pour rajouter un peu de gras, réservez-les avec les champignons
\etape mettez les oignons à revenir avec une pincée de sucre. 
\etape Pendant ce temps, coupez les pommes en petits cubes de $5\unit{mm^3}$ environ. 
\etape Une fois les oignons cuits, rajoutez les champignons et les lardons, puis ajouter en saupoudrant, une cuillère à soupe rase de farine. Remuez jusqu'à ce que ce soit homogène puis ajoutez le bouillon, une cuillère à soupe de fond de veau et le pommeau de Normandie. 
\etape Rajoutez alors les pommes et les paupiettes.
\end{preparation}

\begin{cuisson}
Faites alors cuire pendant une heure environ à feu doux et à couvert en remuant de temps en temps.
\end{cuisson}


\end{recette}

\begin{recette}{Petits pois}{3}{30 minutes}{30 minutes}\index{petits pois}
\begin{ingredients}
\ingredient $2$ grosses boites de Petits pois extra fins sans carottes. 
\ingredient $4$ ou $5$ oignons
\ingredient 2 carottes
\ingredient 250g de lardons fumés
\ingredient 6 saucisses
\ingredient 10g de beurre
\ingredient une cuillère à soupe de farine
\ingredient 30cl de bouillon de volaille
\end{ingredients}

\begin{preparation}
\etape Émincez les oignons, coupez les carottes en petits cubes (d'abord en julienne, puis en petits morceaux) et égouttez les petits pois. Préparez le bouillon en mettant au micro-onde un bouillon cube dans un bol d'eau.
\etape Faites revenir les lardons puis réservez-les. Faites dorer les saucisses (percez les un peu avec une fourchette pour que le gras puisse s'échapper) puis réservez-les aussi.
\etape Faites revenir les oignons et rajoutez un peu de beurre dans les oignons.
\etape Rajoutez alors les carottes, faites revenir quelques minutes.
\etape Rajoutez alors les lardons, une cuillère à soupe de farine, et mélangez.
\etape Rajoutez le bouillon, les petits pois, les saucisses, mélangez et faites mijoter à feu moyen pendant 30 minutes.
\end{preparation}

\end{recette}

\begin{recette}{Pizza (pâte)}{3}{30 min+1h}{15 min}\index{pizza}\index{pâte à pizza}
\begin{ingredients}[Pour une grande pizza]
\ingredient 250 g de farine de blé
\ingredient 1 sachet de levure de boulanger déshydratée
\ingredient 3 cuillères à soupe d’huile d’olive
\ingredient 1 cuillère à café rase de sel
\ingredient 15 cl d’eau tiède ($\sim\unit{40\degres C}$) (si 500g de farine, c'est 25cl)

\end{ingredients}
\begin{remarque}
Il faut de la levure de boulanger, et non de la levure chimique. La levure chimique va agir pendant la cuisson alors que la levure de boulanger agit essentiellement avant la cuisson. 

Les poudres à lever (levure chimique) conviennent aux pâtes génoises mais ne peuvent être utilisées dans des recettes de pains, pizzas, brioches\dots
\end{remarque}

\begin{preparation}
\etape Mélangez le sel dans l'eau tiède. Saupoudrez la levure petit à petit et tournez pour mélanger et éviter les grumeaux.
\etape Creusez un puit dans la farine mise dans un récipient et versez-y l'eau tiède et l'huile. 
\etape Mélangez en incorporant la farine au centre peu à peu à l'aide d'un batteur (et des ustensiles pour pétrir les pâtes, et non les fouets). Continuez jusqu'à ce que la pâte soit trop épaisse pour pouvoir être pétrie de cette façon.
\etape Saupoudrez un peu de farine restant sur le dessus, et finissez de pétrir à la main, en ajoutant la farine petit à petit, en utilisant le fait que la pâte colle tant qu'il n'y a pas assez de farine.
\etape Couvrez la d'un linge et laissez-la reposer pendant une heure. 
\begin{remarque}
Inutile de s'embêter avec la température, il ne faut pas que ce soit trop froid mais 20°C est suffisant. C'est le fait de mettre la levure dans l'eau tiède qui l'active.
\end{remarque}

\etape Étalez un peu de farine sur du papier cuisson (de préférence aux dimensions de votre four)
\etape Farinez un rouleau à patisserie et étalez la pâte préalablement farinée sur le papier cuisson. Afin de mieux étaler, il ne faut pas chercher à rouler entièrement la pâte en un coup mais plutôt faire des aller-retour sur une petite zone, afin que la pâte garde la forme (sinon elle a tendance à revenir sur elle même). 
\begin{remarque}
Il ne faut pas trop étaler la pâte, sinon elle ne montera absolument pas durant la cuisson.
\end{remarque}

\etape Garnissez la pizza selon votre goût et enfournez.
\end{preparation}

\begin{remarque}
Il est possible de congeler la pâte une fois la levée effectuée, en boule. Il suffit ensuite de la laisser décongeler et de la rouler comme une pâte à pizza normale. Vous pouvez même la pétrir brièvement pour qu'elle ait une meilleure texture si ça ne vous convient pas.

Avec 250g de farine et mon petit four, je fais deux pizzas, donc j'en congèle la moitié à chaque fois.
\end{remarque}


\begin{cuisson}
Faites cuire la pizza environ 15 minutes à 210\degres C au four.

\begin{attention}
Ne faites pas cuire avec le four en mode grill !
\end{attention}

\end{cuisson}


\end{recette}

\begin{recette}{Pizza à la crème}{3}{10 min + 20 min}{20 min}\index{pizza}\index{crème}
\begin{ingredients}
\ingredient Pâte à pizza
\ingredient 20cl de crème fraîche liquide (surtout pas de l'épaisse)
\ingredient Lardons (de préférence des allumettes, un demi-paquet suffit --- 50g)
\ingredient Fromage rapé
\ingredient Olive noire
\ingredient poivre, herbes, fond de veau
\end{ingredients}

\begin{preparation}
\etape Sortez la pâte du frigo 10 minutes avant afin qu'elle soit à température ambiante ou faites une pâte vous même.
\etape Étalez là dans un plat pour aller au four directement avec le papier sulfurisé fourni.
\etape Faites réduire la crème fraiche liquide dans une petite casserole avec une cuillère à café de fond de veau et un peu de romarin. Portez à ébullition ou un peu en dessous et faites bien réduire, il y en a pour environ 10 minutes je dirais. Quand ça commence à mousser, c'est que ça doit pas être loin. Afin de s'en assurer, éloigner la casserole du feu et regardez la consistance. Gardez à l'esprit que c'est toujours plus liquide à chaud que ça ne le sera à froid.
\etape Une fois fait, étalez ça sur la pâte à pizza. Ajoutez le fromage rapé, les lardons par dessus, et enfin les olives noires.
\end{preparation}

\begin{cuisson}
Faites cuire la pizza environ 20 minutes à 210\degres C.
\end{cuisson}


\end{recette}

\begin{recette}{Pizza au magret}{3}{}{20 min}\index{pizza}\index{magret}\index{canard}

\begin{ingredients}
\ingredient[Pour la sauce tomate]
\ingredient Petite boite de concentré de tomate
\ingredient huile d'olive
\ingredient Origan, Poivre

\ingredient[Pour la pizza]
\ingredient Pâte à pizza
\ingredient un magret de canard
\ingredient Lardons (de préférence des allumettes, un demi-paquet suffit --- 50g)
\ingredient Fromage rapé (environ 200g)
\ingredient Une boite d'olives noires
\end{ingredients}

\begin{preparation}
\etape Faites préchauffer le four à 210\degres C.
\etape Sortez la pâte du frigo et étalez là dans un plat pour aller au four.
\etape Mélangez le concentré de tomate, de l'origan, un peu de poivre et de l'huile d'olive.
\etape Étalez le avec une cuillère à café. Il ne dois pas y en avoir beaucoup (pas besoin que la couche de coulis soit uniforme).
\etape saupoudrez abondamment de gruyère rapé. Il faut généralement au moins un paquet de 200g. Saupoudrez de lardons.
\etape Coupez en tranche semi-épaisse (3 à 4mm) le magret de canard et dispersez les tranches sur la pizza.
\etape Ajoutez les olives noires.
\end{preparation}

\begin{cuisson}
Faites cuire la pizza environ 20 minutes à 210\degres C.
\end{cuisson}
\end{recette}

\begin{recette}{Pizza aux boulettes de bœuf}{3}{}{}\index{pizza}\index{boulettes de bœuf}\index{bœuf}

\begin{ingredients}
\ingredient[Pour la sauce tomate]
\ingredient Petite boite de concentré de tomate
\ingredient huile d'olive
\ingredient Origan, Poivre

\ingredient[Pour la pizza]
\ingredient Pâte à pizza
\ingredient 5 boulettes de boeufs
\ingredient Fromage rapé (environ 200g)
\ingredient Une boite d'olives noires
\end{ingredients}

\begin{preparation}
\etape Faites préchauffer le four à 270\degres C.
\etape Sortez la pâte du frigo et étalez là dans un plat pour aller au four.
\etape Mélangez le concentré de tomate, de l'origan, un peu de poivre et de l'huile d'olive.
\etape Étalez le avec une cuillère à café. Il ne dois pas y en avoir beaucoup (pas besoin que la couche de coulis soit uniforme).
\etape saupoudrez abondamment de gruyère rapé. Il faut généralement au moins un paquet de 200g.
\etape Coupez en 4 les boulettes de boeufs (si congelées, il faut les décongeler avant) et les répartir sur la pâte.
\etape Ajoutez les olives noires.
\end{preparation}

\begin{cuisson}
Faites cuire la pizza environ 13 minutes à 270\degres C.
\end{cuisson}
\end{recette}

\begin{recette}{Porc au caramel}{3}{1h30}{40min}\index{porc}\index{caramel}

\begin{ingredients}
\ingredient[Pour la sauce]
\ingredient un roti dans l'échine (1.5kg environ)
\ingredient 2 oignons
\ingredient 2 gousses d'ail
\ingredient Un morceau de gingembre frais ($\sim 50$ g)
\ingredient 12.5cl de sauce soja

\ingredient[Pour le caramel]
\ingredient 150g de sucre
\ingredient 5cl d'eau
\end{ingredients}

\begin{preparation}
\etape couper le porc en fins morceaux (je fais des tranches d'un demi centimètre environ, puis des lamelles d'1cm de large, et
je coupe ces lamelles en morceaux d'1 cm de long).
\etape pelez puis mixez l'ail, le gingembre et l'oignon que vous réservez dans un grand saladier (qui contiendra aussi le porc)
\etape Faire revenir les morceaux de porc à feu très vif pendant 3 minutes environ (pour 2kg je fais environ 4 fournées). Il
faut juste faire blanchir la viande et légèrement dorer. Réservez les morceaux cuits dans le saladier contenant gingembre et
oignon
\etape Dans la sauteuse, ajoutez enfin la totalité du porc et des légumes mixés. 
\etape Saupoudrez une cuillère à soupe de farine puis mélangez. 
\etape Ajoutez la sauce soja et laissez le tout réchauffer à feu doux pendant qu'on s'occupe du caramel
\etape Dans une casserole, versez le sucre et le fond d'eau. Mettez à feu vif et attendez que le caramel prenne une coloration
brune.
\etape Nappez alors le caramel obtenu sur le porc dans la sauteuse puis homogénéisez la sauce.
\end{preparation}

\begin{cuisson}
Couvrez et laissez mijoter à feux doux pendant 40 minutes environ.
\end{cuisson}
\end{recette}

\begin{recette}{Poulet au curry}{3}{45min}{1h}\index{poulet au curry}\index{poulet}\index{curry}
% (Excellent)

\begin{ingredients}
\ingredient 4 cuisses de poulet
\ingredient une pomme
\ingredient 2 oignons
\ingredient une gousse d'ail
\ingredient une boite (400g) de tomate en dés 
\ingredient 25cl de bouillon de volaille
\ingredient 25cl de crème fraiche
\ingredient 2 cuillères à soupe de curry
\ingredient un peu de jus de citron
\ingredient un peu de muscade rapé, un peu de canelle, sel, poivre, sucre si besoin
\end{ingredients}

\begin{preparation}
\etape Émincez l'oignon, rapez la pomme dans le même récipient
\etape Saisissez les morceaux de poulet dans une sauteuse
\etape Réservez les morceaux puis faites blondir l'oignon et la pomme dans les sucs.
\etape Pendant ce temps, préparez le bouillon de volaille, et écrasez la gousse d'ail (à la fourchette) pour l'incorporer au 
bouillon.
\etape Ajoutez alors le curry, la canelle, la muscade avec les oignons/pomme. Mélangez, puis ajoutez le bouillon et les dés de 
tomate (et un peu de sucre si besoin pour corriger l'acidité, notez que la pomme est déjà sucrée). 
\end{preparation}

\begin{cuisson}
Faites cuire pendant 45 minutes environ, à couvert et à feu doux. Ajoutez alors la crème fraiche et un peu de jus 
de citron, puis laissez mijoter encore 15 minutes environ. 

Servez avec un riz bazmati.
\end{cuisson}
\end{recette}

\begin{recette}{Poulet aux pruneaux et à la crème de whisky}{3}{1h30}{}\index{poulet}\index{pruneau}\index{crème de 
whisky}\index{bailey's}
\begin{ingredients}
\ingredient 8 morceaux de poulet
\ingredient 25 cl de crème de whisky (Bailey's)
\ingredient 3 échalotes
\ingredient 1 gousse d'ail
\ingredient 100g de pruneaux
\ingredient 15cl d'eau et un cube de bouillon de volaille
\ingredient 1 cuillère à soupe de farine
\ingredient huile, sel, poivre, jus de citron
\end{ingredients}

\begin{preparation}
\etape Émincez l'échalotte et l'ail très finement. Mettez les pruneaux dans 15cl d'eau chaude
\etape Saisissez les morceaux de poulets dans la sauteuse puis réservez-les
\etape Dans le jus du poulet, faites revenir les échalottes et la gousse d'ail écrasée.
\etape Quand les échalottes sont prêtes, réservez les pruneaux et préparez le bouillon cube dans le jus des pruneaux. Saupoudrez la farine et mélangez afin que la farine entoure les morceaux d'échalotte
\etape Ajoutez le bouillon, mélangez puis ajoutez la crème de whisky. Ajoutez ensuite les pruneaux et un peu de jus de citron. Mélangez puis ajoutez les morceaux de poulet. Salez et poivrez.
\etape Laissez mijoter une heure environ, à couvert et à feux doux.
\end{preparation}

\end{recette}

\begin{recette}{Poulet Chasseur}{4}{1h}{1h}\index{poulet}\index{poulet chasseur}
\begin{ingredients}
\ingredient 8 morceaux de poulet
\ingredient $250\unit{g}$ de champignons
\ingredient 3 échalotes
\ingredient $4\unit{cl}$ de Cognac
\ingredient $4\unit{cl}$ de vin blanc
\ingredient Un bol de bouillon de volaille
\ingredient farine, beurre, huile, sel, poivre
\ingredient estragon, cerfeuil
\end{ingredients}


\begin{preparation}
\etape Découpez et dégraissez les morceaux de poulet. 
\etape Épluchez, lavez et émincez les champignons. Épluchez et ciselez les échalotes.
\etape Saisissez dans du beurre ou de l'huile les morceaux de poulet à feu vif puis réservez-les.
\etape Faites revenir les échalottes, réservez-les dans une assiette (pas avec les morceaux de poulet).
\etape Faites revenir les champignons dans la sauteuse. 
\etape Rajoutez alors les échalottes. Ajoutez le cognac, sortez la sauteuse de feu et faites flamber. 
\etape Ajoutez une cuillère à soupe rase de farine, mélangez.
\etape Ajoutez le vin blanc, le bouillon de volaille, le cerfeuil et l'estragon. 
\end{preparation}

\begin{cuisson}
Laissez alors mijoter à couvert et à feu doux une heure environ. À la fin, contrôlez l'assaisonnement et la liaison.
\end{cuisson}
\end{recette}

\begin{recette}{Poulet gascon}{4}{45min}{1h}\index{poulet gascon}\index{poulet}\index{floc de gascogne}
% (Excellent)

\begin{ingredients}
\ingredient 4 cuisses de poulet
\ingredient $200$ g de champignons
\ingredient $2$ oignons (fenouils?)
\ingredient $2$ gousses d'ail
\ingredient $15$ cl de floc de gascogne
\ingredient 15cl de bouillon de volaille
\ingredient 1 cuillère à soupe rase de farine
\ingredient sel, poivre du moulin
\end{ingredients}

\begin{preparation}
\etape Écrasez les gousses d'ails, émincez l'oignon et coupez la carotte en petits cubes (en fine lamelle que vous coupez en tranche) et placez les dans des récipients séparés.
\etape Faites fondre le beurre dans une sauteuse et faites revenir les cuisses à feu vif. Pas besoin que la viande soit cuite à l'intérieur, c'est juste pour faire dorer.
\etape Réservez les morceaux puis faites revenir successivement et séparément l'oignon, les cubes de carotte et les champignons.
\etape Remettez dans la sauteuse oignon, carotte et champignon, mélangez.
\etape Ajoutez alors la farine, remuez jusqu'à l'incorporer autour des légumes.
\etape Ajoutez le bouillon de volaille afin d'homogénéiser la farine entourant les légumes et le bouillon.
\etape Ajoutez enfin le floc de gascogne, l'ail écrasé et les morceaux de poulets.
\end{preparation}

\begin{cuisson}
Faites cuire pendant une heure environ à feu doux et à couvert.
\end{cuisson}

\end{recette}

\begin{recette}{Ratatouille confite}{4}{1h30}{10h}\index{ratatouille}\index{roti}\index{porc}
\begin{ingredients}
\ingredient 200g de lardons
\ingredient 4 gros oignons
\ingredient 2 belles aubergines
\ingredient 2 poivrons verts
\ingredient 1 poivron rouge
\ingredient 4 courgettes
\ingredient 10 tomates
\ingredient 6 c à s de concentré de tomates
\ingredient 5 gousses d'ail
\ingredient 4 morceaux de sucre
\ingredient huile d'olive, sel, poivre, herbes de provence
\end{ingredients}

\begin{preparation}
\etape Peler les tomates (voir \refsec{sec:peler_tomate}), puis les écraser à la main dans une marmite. Ajoutez les 5 gousses d'ail écrasées, le sucre, le laurier et les herbes. 
\etape Faire réduire jusqu'à la consistance d'une purée, il faut que le liquide ait quasiment disparu. Normalement, en faisant réduire à feu moyen à moyen-vif pendant que vous faites revenir les autres légumes ça sera prêt. 
\begin{remarque}
Les autres légumes rendront de l'eau, ce n'est pas grave si la purée est bien réduite.
\end{remarque}
\etape Pendant ce temps, préparez les légumes : 
\etape Émincer les oignons épluchés et les faire fondre à feu doux dans une poêle avec du poivre. Réservez dans un saladier
\etape Émincer les poivrons et les faire fondre avec du sel jusqu'à ce qu'ils soient mous. Les mettre dans le saladier
\etape Laver les courgettes, les couper en petits cubes et les faire dorer à la poêle. Les mettre dans le saladier.
\etape Laver les aubergines, les couper en petits cubes et les faire dorer avec du poivre. Les réserver avec oignons, poivrons et courgettes.
\etape Dans la purée de tomate, rajouter le concentré de tomate, les lardons crus et les légumes revenus (oignon, poivron, courgette et aubergine).
\end{preparation}

\begin{cuisson}
Laisser mijoter à couvert pendant 7 à 10h environ à feu très doux (au minimum). 

Le confit de ratatouille est prêt quand il change de couleur et devient foncé. Goûter et rectifier l'assaisonnement si nécessaire. Si, à ce moment-là, il remonte un peu d'huile à la surface ou de jus, le retirer et arroser la ratatouille au moment du service avec un filet d'huile d'olive.

%TODO pour cuire le roti, il faut environ 4h dans la ratatouille, après (ou à ce moment là) il a commencé à cramer. Il faut laisser la ficelle sinon il risque de se désagréger.
\end{cuisson}
\end{recette}

\begin{recette}{Risotto aux champignons}{3}{1h}{}\index{risotto}
\begin{ingredients}
\ingredient 500g de riz (cuisson longue 20 min)
\ingredient 200g de champignons
\ingredient 2 oignons
\ingredient 10cl de vin blanc (pas trop, ou pas du tout de vin blanc, sinon ce n'est pas évaporé au bout des 20 minutes de cuisson)
\ingredient 25cl de crème fraiche liquide
\ingredient 65cl de bouillon de volaille (à 500g de riz correspond 1L de liquide (vin + crème fraiche liquide + bouillon)
\ingredient sel, poivre, herbes de provence
\end{ingredients}

\begin{preparation}
\etape Faites blondir l'oignon émincé dans une sauteuse
\etape Réservez-le et faites revenir les champignons
\etape Réservez les champignons, ajoutez de l'huile et faites-y rissolez le riz jusqu'à ce qu'il devienne translucide.
\etape Ajoutez l'oignon, les champignons et mélangez.
\etape Ajoutez le vin blanc et remuez quelques instants 
\etape Ajoutez alors le bouillon de volaille et la crème fraiche liquide. 
\etape Ajoutez les herbes et rectifiez l'assaisonnement.
\etape Laissez cuire 25 minutes à feu moyen et à couvert, le temps que le riz soit cuit. 
\begin{remarque}
Il faut plus de temps que le temps normal indiqué pour le riz
\end{remarque}
\end{preparation}

\end{recette}

\begin{recette}{Roti de Veau en cocotte aux champignons}{0}{}{}\index{roti}\index{veau}
\begin{ingredients}
\ingredient 1 rôti de veau de 1 à 1,2 kg
\ingredient 40 g de beurre
\ingredient 1 oignon
\ingredient sel et poivre
\ingredient 6 cuillères à café de fond de veau
\ingredient 1 verre de vin blanc (20 cl environ)
\ingredient 1 cuillère à soupe d'herbes de Provence
\ingredient 1 cuillère à soupe de basilic
\ingredient 1 grosse boîte de champignons de Paris
\end{ingredients}

\begin{preparation}
\etape Faire fondre le beurre dans une cocotte minute, y faire dorer le rôti de toutes parts, puis ajouter l'oignon émincé, le sel et le poivre.
\etape Délayer dans un bol le fond de veau, les herbes, le basilic, le vin blanc et 1 verre d'eau. Verser ce mélange dans la cocotte, ainsi que les champignons égouttés et remuer.
\etape Fermer la cocotte et laisser chuchoter environ 30 mn par kg.
\end{preparation}

\begin{remarque}
Pour accompagner la recette, des pommes de terres au four vont très bien, voir \refsec{sec:accompagnement}.
\end{remarque}
\end{recette}

\begin{recette}{Salade bigourdane}{5}{30 min}{}\index{salade bigourdane}\index{salade landaise}
\begin{ingredients}
\ingredient salade verte
\ingredient 10 à 15 noix
\ingredient 100g de lardons
\ingredient 100g de gésiers de canard confits
\begin{remarque}
Le plus pratique, ce sont les paquets de gésiers au rayon lardons ou canard. Les gésiers de volailes sont émincés et absolument pas gras. Beaucoup moins embêtant que des gésiers entiers, conservés dans la graisse.
\end{remarque}
\ingredient 50g de gruyère non rapé
\ingredient 1/4 de baguette de pain frais
\ingredient une gousse d'ail (ou ail semoule)
\ingredient vinaigrette (voir \refsec{sec:vinaigrette} ou \refsec{sec:vinaigrette-moutarde})

\end{ingredients}

\begin{preparation}
\etape Préparez les noix, que vous laissez dans un saladier. Ajoutez la salade, la vinaigrette et remuez.
\etape Si vous avez une gousse d'ail fraiche, frottez le pain avec. Puis coupez le pain en deux dans le sens de la longueur, et faites 3 à 4 lamelles dans chacun des deux morceaux. Coupez ensuite ces lamettes en petits cube d'environ un centimètre de long.
\etape Coupez le gruyère en cubes d'un demi centimètre de coté environ.
\etape Recoupez les gésiers pour faire des cubes d'un peu moins d'1 cm de coté.
\etape Faites revenir les lardons à la poële. 
\etape Réservez-les puis faites revenir les gésiers, et réservez-les avec les lardons.
\etape Faites alors dorer le pain dans la graisse ainsi rendue. Si vous avez de l'ail semoule à la place de la gousse d'ail, ajoutez le à ce moment là dans la poële afin de mélanger au pain. Il faut que le pain soit légèrement croustillant au bord, mais moelleux à l'intérieur. Il faut donc le surveiller, le tourner de temps en temps, et ne pas mettre à feu trop vif. Le pain ne va pas forcément dorer, il se peut que vous vous retrouviez avec des biscottes si vous cherchez absolument à ce qu'il colore. 
\etape Une fois le pain presque prêt, ajoutez les lardons et gésiers, remuez avec de faire réchauffer le tout, et de re-graisser le pain afin qu'il finisse de dorer. 
\etape Une fois chaud et prêt, éteignez le feu. Ajoutez alors le fromage en cube dans la poële en dehors du feu, puis versez immédiatement sur la salade, et mangez de suite. Ainsi, le fromage sera légèrement fondant, sans faire de filaments pour autant.
\end{preparation}
\end{recette}

\begin{recette}{Tagliatelles aux Noix de St Jacques}{4}{}{}\index{pâtes}\index{tagliatelle}\index{noix st Jacques}
\begin{ingredients}
\ingredient 500g de noix St Jacques
\ingredient 5 gousses d'ail
\ingredient 6 champignons
\ingredient 15cl de vin blanc
\ingredient 25cl de crème liquide
\ingredient 4 cuillères à soupe rase de sauce tomate (ne surtout pas en mettre plus)
\ingredient sel, poivre, persil
\end{ingredients}

\begin{preparation}
\etape Faire revenir les noix St Jacques, les gousses d'ail et le persil finement hachés avec une noix de beurre pendant 2 à 3 minutes.
\etape Rajouter les champignons et laisser cuire quelques minutes. N'attendez pas que les champignons soient cuits, laissez simplement fondre un peu puis ajoutez le vin blanc et laissez réduire jusqu'à ce que l'odeur de vin disparaisse presque complètement.
\etape Ajoutez alors la crème liquide et la sauce tomate.
\begin{remarque}
Il ne faut surtout pas mettre plus de tomate que les 4 cuillères à soupe. On peut rajouter un soupçon de sucre pour corriger un peu la tomate ou le vin blanc.
\end{remarque}
\etape Laissez mijoter 5 minutes puis servir sur une assiette les tagliatelles cuites puis disposez les noix en sauce par dessus
\end{preparation}

\end{recette}

\begin{recette}{Tajine d'agneau aux pruneaux}{0}{1h}{4h}\index{tajine}\index{agneau}\index{pruneaux}
\begin{ingredients}
\ingredient 500g de pruneaux
\ingredient 1kg d’oignon (à peu près)
\ingredient 1,5kg d’épaule d’agneau
\ingredient 2 gousses d'ail
\ingredient 50cl de bouillon de volaille (ou d'eau)
\ingredient une cuillère à café de cannelle
\ingredient un petit morceau de gingembre frais
\ingredient une dosette de safran (1g)
\ingredient quelques grains de coriandre écrasé
\ingredient sel, poivre
\end{ingredients}

\begin{preparation}
\etape Faites tremper les 500 g pruneaux dans le bouillon (ou l'eau) chaud. 
\etape Faites revenir les oignons émincés dans l'huile d'olive. Quand ils sont dorés, retirez-les et remplacez-les par la viande.
\etape Lorsque les morceaux sont bien dorés, mettez-les dans une cocotte avec les oignons. Salez et poivrez. Ajoutez l'ail et le gingembre écrasé, 
la cannelle, la safran et les grains de coriandre.
\etape Rajoutez le bouillon dans lequel ont trempé les pruneaux.
\begin{remarque}
Servez accompagné de semoule de blé.
\end{remarque}
\end{preparation}

\begin{cuisson}
Faites cuire à feux doux et à couvert pendant 5 heures environ. Ajoutez les pruneaux égouttés 30 minutes avant la fin de la
cuisson.
\end{cuisson}

\end{recette}

\begin{recette}{Tartiflette}{3}{1h}{30 min}\index{tartiflette}\index{roblochon}

\begin{ingredients}
\ingredient $1,5\unit{kg}$ de pommes de terre à chair ferme
\ingredient $200\unit{g}$ de lardons
\ingredient 2 oignons
\ingredient $1$ reblochon fermier
\ingredient 20-25cl de crème fraiche
\ingredient 5cl de vin blanc sec (facultatif)
\end{ingredients}

\begin{preparation}
\etape Éplucher les pommes de terre. Faites les cuire à l'autocuiseur 15 minutes (à partir du moment où ça siffle).
\etape Au terme de la cuisson, égoutter et laisser tiédir. (ne pas rafraichir !!!)
\etape Faites revenir les lardons quelques minutes puis réservez les
\etape Émincez l'oignon et faites le suer à la poêle dans la graisse des lardons.
\etape Coupez en cubes grossiers les pomme de terre. Mélangez ces cubes avec les oignons, les lardons, la crème et le petit verre de vin blanc sec et étalez ça dans le plat à gratin. 
\etape Découpez le reblochon en deux dans le sens de l'épaisseur (pour plus de facilité, on peut le découper alors qu'il est encore dans l'emballage, le fromage se tient mieux) et le déposer sur vos pommes de terre, croûte vers le bas.
\end{preparation}

\begin{cuisson}
Enfourner à four très chaud ($220-250\degres C$). Jusqu'à ce que le reblochon fonde et gratine en surface.
\end{cuisson}
\end{recette}

\begin{recette}{Tourin à l'ail}{3}{}{}\index{tourin à l'ail}\index{ail}
\begin{ingredients}
\ingredient 20 gousses d'ail
\ingredient 2 gros oignons
\ingredient 2 cuil. à soupe de farine
\ingredient 3 oeufs
\ingredient 1 cuil. à soupe de vinaigre
\ingredient huile d'olive, sel, poivre
\end{ingredients}

\begin{preparation}
\etape Faire bouillir 2 l d'eau avec 20 gousses d'ail épluchées.
\etape Dans un faitout, faire revenir deux gros oignons émincés dans de l'huile d'olive jusqu'à ce qu'ils deviennent translucides, sans les faire brunir.
\etape Ajouter deux cuil. à soupe de farine, mélanger et mouiller avec les 2 l d'eau et l'ail.
\etape Faire bouillir, ajouter une bonne pincée de sel et deux pincées de poivre.
\etape Couvrir et laisser mijoter à feux doux pendant une petite heure.
\etape Pendant ce temps, casser trois oeufs en séparant les blancs des jaunes.
\etape Ajouter dans les jaunes une cuil. à soupe de vinaigre de vin rouge, et diluer avec un peu de bouillon. Réserver.
\etape Au bout de 1 h de cuisson, ajouter les blancs d'oeuf au bouillon en agitant continuellement avec une cuillère en bois : ils formeront de longs filaments blancs.
\begin{remarque}
Pour cela, il faut commencer à remuer le bouillon pour lui donner une rotation relativement importante, puis verser lentement le blanc d'œuf tout en continuant de remuer.
\end{remarque}
\etape Hors du feu, ajouter les jaunes en les mélangeant d'un mouvement large et ferme.
\etape Remettre à feu très doux, sans laisser bouillir, une dizaine de minutes.
\end{preparation}

\begin{remarque}
Au moment de servir le tourin, vous pouvez disposer dans chaque assiette une tartine de pain de campagne arrosée d'huile d'olive, et assaisonner de poivre suivant votre goût : pour le tourin, soyez avare de sel et prodigue de poivre !
\end{remarque}
\end{recette}

\begin{recette}{Velouté de champignons}{0}{1h}{}\index{champignons}\index{velouté de champignons}
\begin{ingredients}
\ingredient 500g de champignons
\ingredient 20cl de crème fraiche liquide
\ingredient 1L de bouillon de volaille
\ingredient 3 échalottes
\ingredient 2 gousses d'ail
\ingredient 1 cuillère à soupe rase de farine
\ingredient sel, poivre, céleri, persil
\end{ingredients}

\begin{preparation}
\etape Faire suer l'échalotte dans un peu de graisse
\etape Ajoutez les champignons émincés et faites les revenir un peu
\etape Ajoutez alors une cuillère à soupe de farine puis mélangez
\etape Ajoutez alors le bouillon, l'ail émincé, sel, poivre, céleri et persil
\etape Couvrez et laissez cuire à feu doux pendant 45 minutes
\etape Ajoutez la crème fraiche, puis mixez le tout
\end{preparation}
\end{recette}

\begin{recette}{Velouté de courgette}{4}{1h}{}\index{velouté}\index{courgette}\index{velouté de courgettes}
\begin{ingredients}
\ingredient 1kg de courgettes
\ingredient bouillon de volaille
\ingredient 125g de boursin (ou équivalent ail et fines herbes)
\ingredient sel, poivre
\end{ingredients}

\begin{preparation}
\etape Coupez les courgettes en morceaux 
\etape Dans la marmite, ajoutez de l'eau jusqu'à couvrir les courgettes et faites bouillir pendant 45 minutes dans le bouillon
de volaille. 
\etape Égouttez alors mixez les courgettes ainsi cuites avec le fromage
\etape Ajoutez sel, poivre. Vous pouvez aussi rajouter un soupçon de céleri.
\end{preparation}
\end{recette}
