\begin{recette}{Brownie au Chocolat}{2}{15min}{20 min}
\begin{ingredients}
\ingredient 200g de chocolat patissier
\ingredient 100g de beurre
\ingredient 150g de sucre
\ingredient 4 œufs
\ingredient 75g de farine
\end{ingredients}

\begin{preparation}
\etape Commencez par mettre du papier sulfurisé au fond du moule afin de ne pas s'embêter à démouler ou faire la vaisselle. 
\etape Dans un saladier, mettez le chocolat et le beurre fondu et mélangez bien afin que ce soit homogène.
\begin{remarque}
On peut faire fondre au micro-onde (environ 1 min 30) ou à la casserole. Pensez à couvrir au micro-onde, le beurre explose.
\end{remarque}
\etape Ajoutez ensuite le sucre et mélangez.
\etape Incorporez un à un les œufs en mélangeant.
\etape Ajoutez les 3 cuillères à soupe de farine et mélangez jusqu'à ce que ce soit homogène. 
\end{preparation}

\begin{cuisson}
Allumez le four à 180°C (thermostat 6), chaleur tournante si possible, pas besoin de préchauffer. On enfourne 20 minutes très précisément, et on sort du four.

Ce gâteau n'est pas aéré, pas monté, d'où la dénomination de brownie que je lui ai donné. Il est donc assez compact, se mange très bien avec une crème anglaise par exemple.
\end{cuisson}
\end{recette}

\begin{recette}{Charlotte Aux Fraises}{0}{}{}
\begin{ingredients}[6 pers.]
\ingredient $500$ g de fraises (goûteuses et bien tendres), équeutées et coupées en morceaux
\ingredient $100$ g de fraises (les plus belles), pour la décoration
\ingredient $25$ cl de lait
\ingredient $35$ cl de crème fraîche liquide
\ingredient $1$ gousse de vanille
\ingredient 4 œufs
\ingredient $6$ cuil. à soupe de sucre
\ingredient $1$ cuil. à café de maïzena (ou de fécule)
\ingredient $4$ feuilles de gélatine (soit 8 g), mises à tremper dans de l'eau froide
\ingredient $1$ boîte de biscuits à la cuillère
\ingredient Chantilly en bombe pour la décoration
\end{ingredients}

\begin{preparation}
\etape[Commencez par confectionner la garniture de la charlotte]
\etape Portez à ébullition le lait et la gousse fendue, couvrez et laissez infuser.
\etape Fouettez les jaunes d'œufs avec 3 cuil. à soupe de sucre pour obtenir une mousse blanche.
\etape Incorporez la maïzena, puis le lait chaud.
\etape Remettez ensuite le tout dans la casserole et faites épaissir, sur feu doux, sans laisser bouillir.
\etape Retirez du feu, ajoutez la gélatine essorée, en mélangeant pour qu'elle se dissolve parfaitement, puis laissez légèrement tiédir.
\etape Fouettez la crème fraîche en chantilly, en ajoutant 3 cuil. à soupe de sucre en cours d'opération.
\etape Incorporez cette chantilly dans la crème à la vanille tièdie.

\etape[Enfin, procédez au montage de la charlotte]
\etape Tapissez un moule à charlotte de film plastique (ou d'aluminium).
\etape Mélangez les morceaux de fraises à la moitié de la crème de garniture.
\etape Versez la moitié de la garniture sans fraises dans le moule, couvrez d'une couche de biscuits, mettez la garniture aux fraises, puis des biscuits, le reste de garniture nature et terminez par une couche de biscuits.
\etape Posez une assiette sur la charlotte aux fraises et tassez légèrement.
\etape Laissez reposer la charlotte au réfrigérateur pendant au moins 4 heures.
\etape Au moment de servir, démoulez la charlotte aux fraises et décorez-la avec les fraises restantes et de la crème chantilly.
\end{preparation}

\end{recette}

\begin{recette}{Crêpes}{3}{10 min + 1h}{1h}
% (recette de marie)
\begin{ingredients}
\ingredient 250g farine
\ingredient 4 œufs
\ingredient $\sfrac{1}{2}$ lait
\ingredient 1 cuillère a soupe de sucre vanillé
\ingredient 1 pincée sel
\ingredient 50g beurre
\ingredient Rhum, ricard, arôme de vanille et extrait d'orange selon convenance pour parfumer.
\end{ingredients}


\begin{preparation}
\etape Dans un saladier, verser la farine et le sel.
\etape Y faire un puit et incorporer les œufs un par un.
\etape Verser petit à petit le lait et le mélanger à l'aide d'un fouet jusqu'à l'obtention d'une pâte liquide puis y rajouter l'huile.
\etape Rajouter les arômes souhaités
\etape Laisser reposer la pâte une heure au frais.
\end{preparation}

\begin{remarque}
Pour des crêpes plus légères, mettre moitié de lait et moitié d'eau
\end{remarque}

\begin{cuisson}
Faire cuire les crêpes dans une poêle très chaude légèrement huilée.
\end{cuisson}
\end{recette}

\begin{recette}{Crumble aux pommes}{0}{}{30 min}
\begin{ingredients}
\ingredient 5 pommes
\ingredient des framboises (ou du jus de citron, faute de framboises)
\ingredient $150$g de cassonnade
\ingredient $150$g de farine
\ingredient $125$g de beurre ramolli (pas fondu)
\ingredient une cuillère à soupe de cannelle
\end{ingredients}

\begin{preparation}
\etape Coupez les pommes en dés et disposez-les au fond du moule.
\etape Dans un saladier, mettez la farine, le beurre, le sucre et la canelle et malaxez le tout avec les mains. Mélangez jusqu'à obtenir quelque chose d'homogène et de friable.
\etape Répartissez le mélange sur les pommes sans tasser
\end{preparation}

\begin{cuisson}
Enfourner une demi-heure à $180$\degres C. Servir chaud ou tiède dans le plat de cuisson.
\end{cuisson}
\end{recette}


\begin{recette}{Flan}{3}{}{1h}
\begin{ingredients}
\ingredient 1 litre de lait
\ingredient 180g de sucre (dont la moitié pour le caramel)
\ingredient 7 œufs
\end{ingredients}

\begin{preparation}
\etape Faire préchauffer le four à $180\degres C$
\etape Dans une casserole, faites brunir 100g de sucre. Vous pouvez ajouter une goutte d'eau pour que le caramel se fasse plus vite.
\etape Pendant ce temps, mélangez les œufs, le lait et 80g de sucre dans un récipient. Ajoutez une gousse de vanille fendue.
\etape Versez le caramel au fond du moule, puis ajoutez la préparation avec le lait.
\end{preparation}

\begin{cuisson}
Mettre le moule dans un autre récipient plus grand contenant de l'eau, et faites cuire au bain marie pendant 1h (la surface doit être roussie).
\end{cuisson}
\end{recette}

\begin{recette}{Gâteau À l'ananas}{4}{}{1h}
\begin{ingredients}
\ingredient Rhum ou armagnac
\ingredient $150$ g de sucre
\ingredient $150$ g de beurre fondu
\ingredient $150$ g de farine
\ingredient $1$ paquet de levure si celle-ci n'est pas inclue dans la farine
\ingredient $4$ œufs
\ingredient $1$ paquet de caramel
\ingredient $1$ boite d'ananas ($6$ tranches environ)
\end{ingredients}

\begin{preparation}
\etape Beurrez le moule. Mettez le caramel, puis les tranches d'ananas.
\etape Mélangez le sucre, les œufs, la farine et ajoutez en dernier le beurre.
\etape Mettre ce mélange au dessus des ananas.
\end{preparation}

\begin{cuisson}
Faites cuire environ une heure à thermostat 5.

Démoulez chaud et ajoutez le jus d'ananas mélangé à du rhum et du sucre. (le jus froid)
\end{cuisson}

\end{recette}

\begin{recette}{Gâteau Au Yahourt}{4}{}{30 min}

\begin{ingredients}
\ingredient $1$ yahourt nature
\ingredient $2$ pots de yahourt de farine
\ingredient $2$ pots de yahourt de sucre
\ingredient $1$ pot de yahourt d'huile
\ingredient $1$ paquet de levure
\ingredient $2$ pomme
\ingredient $2$ œufs
\ingredient $1$ citron ou orange rapé
\ingredient ricard (pour parfumer)
\end{ingredients}

\begin{remarque}
On peut remplacer les pommes par des poires.
\end{remarque}

\begin{preparation}
\etape Préchauffer le four.
\etape Mélanger tout les ingrédients
\etape Ajouter les pommes coupées en tranches
\etape Beurrer le moule, fariner, puis verser la préparation.
\end{preparation}

\begin{cuisson}
Mettre au four 30 minutes, thermostat 5 (175\degres C)
\end{cuisson}
\end{recette}

\begin{recette}{Gateau brigitte}{0}{}{}
\begin{ingredients}
\ingredient $70$ g de beurre
\ingredient $250$ g de gâteaux types palets breton (2 paquets )
\ingredient $30$ g de sucre
\ingredient $400$ g de lait concentré sucré (une boite moyenne)
\ingredient un peu de lait
\end{ingredients}

\begin{preparation}
\etape Piler les gâteaux pour faire de la chapelure avec des morceaux moyens.

\begin{remarque}
Pour ma part, je tape dans le paquet de gâteaux sans même l'ouvrir pour économiser un torchon et mettre directement le concassé dans le plat. Ceci marche très bien pour des palets bretons par exemple.
\end{remarque}

\etape Le mélanger au sucre et y ajouter le beurre fondu\footnote{on peut faire fondre le beurre au micro onde}. Rajoutez un peu de lait (un fond de verre) pour lier.

\etape Ensuite, on étale le mélange au fond d'un plat et on met au frigo une demi journée (que ce soit froid et que ça durcisse en fait).

\etape Pour le dessus, on met à cuire au bain marie pendant 3h une boite de lait concentré sucré qui va devenir un peu caramélisé.

\etape On dispose des fruits au dessus de ce caramel, tranche de bananes ou selon le gout.

\etape On recouvre de chantilly.
\end{preparation}

\end{recette}

\begin{recette}{Gateau Roulé}{4}{}{}
\begin{ingredients}
\ingredient 3 œufs
\ingredient 75 g de sucre
\ingredient 75 g de farine
\ingredient 25 g de beurre
\ingredient arôme vanille, confiture
\end{ingredients}

\begin{preparation}
\etape Préchauffer le four à thermostat 4 (150°C) et beurrer le moule (un moule relativement grand, et rectangulaire de préférence.
\etape Casser les œufs et séparez le blanc du jaune.
\etape Ajouter le sucre aux jaunes et travailler au fouet jusqu'à ce que la pâte fasse un ruban. Parfumer à la vanille de préférence.
\begin{remarque}
Il y a peu de volume donc le batteur est peu adapté ici.
\end{remarque}
\etape Battre les blancs en neige ferme (ce coup ci, on peut utiliser le batteur). Faire fondre le beurre.
\etape Mélanger la farine aux jaunes. Ajouter maintenant les blancs, mais en remuant avec une cuillère à soupe (surtout pas au batteur). \og Entourez\fg la pâte pour ne pas chasser l'air contenu dans les blancs\footnote{En gros, il faut faire le tour du saladier, le dessous, avec des mouvements amples, sans chercher à exploser l'aglomérat de blanc}.
\etape Terminer en mélangeant le beurre fondu.
\begin{remarque}
Les 25g de beurre mis dans la pâte lui donnent du mœlleux et facilitent le roulage. Si on désire un biscuit plus léger, supprimer le beurre, mais procéder aux opérations du démoulage et du roulage très rapidement affin que le gateau n'ai pas le temps de sécher.
\end{remarque}
\etape Verser la pâte dans le plat et cuire au four pendant 7 à 8 minutes.
\etape Pendant la cuisson, préparer un torchon humide dans lequel vous mettrez le gâteau à sa sortie du four.
\begin{remarque}
À défaut de torchon humide, on peut le démouler normalement et l'humidifier à l'aide d'un pinceau, soit avec de l'eau, soit avec un sirop quelconque. Comptez environ 25cl, la pâte est très absorbante, et plus elle sera humidifiée, plus facile ça sera pour la rouler.
\end{remarque}
\etape À la sortie du four, démoulez le gâteau et roulez le dans le torchon. Laissez reposer.
\etape Déroulez ensuite, puis badigeonnez-le de confiture
\end{preparation}

Personnellement, je pense qu'il serait possible de se passer du torchon. À la place, on humidifie abondamment la pâte avec un sirop composé d'un jus de fruit et de rhum (le jus de fruit se mariant avec la confiture). Laissez un peu absorber le jus, il faut qu'il absorbe suffisamment pour qu'on puisse le rouler ensuite. Si on ne peux pas le rouler, coupez le en 4 et faites des couches, un peu comme un mille feuille.

Ce que j'ai fait, ne pouvant pas le rouler (sans le beurre, il est moins souple et/ou s'il est trop cuit), je le badigeonnais d'eau, puis confiture, puis eau sur la couche suivante que je mettais sur la confiture, eau sur l'autre coté de la dernière couche, puis confiture, et ainsi de suite.

\end{recette}

\begin{recette}{Poires pochées au vin rouge}{0}{1h30}{}
\begin{ingredients}
\ingredient 4 moyennes poires assez fermes
\ingredient 40cl de vin rouge
\ingredient 200g de sucre
\ingredient 1 cuillère à soupe d'extrait de vanille
\ingredient 1 cuillère à soupe d'extrait d'orange (ou zeste non traité)
\ingredient un peu de canelle
\end{ingredients}

\begin{preparation}
\etape Faites chauffer le vin, le sucre et les arômes. Portez à ébullition
\etape rajoutez les poires coupées en morceaux grossiers (typiquement 8 morceaux par poire) et laissez cuire une heure et demi environ jusqu'à ce que le jus devienne un peu plus épais, et les poires moelleuses
\etape Laissez refroidir et dégustez les poires froides avec un peu de chantilly.
\end{preparation}

\end{recette}

\begin{recette}{Tarte Tatin}{3}{}{}
\begin{ingredients}
\ingredient 100g de beurre
\ingredient 185g de sucre
\ingredient 6 pommes
\ingredient 1 pâte brisée
\ingredient cannelle, jus de citron
\end{ingredients}

\begin{preparation}
\etape Pelez et épépinez les pommes puis coupez les en 8. Arrosez-les de jus de citron pour éviter qu'elles noircissent.
\etape Faites fondre le beurre dans une poële, ajoutez le sucre et laissez cuire 5 à 10 minutes à feu moyen pour faire caraméliser ce mélange. Laissez sur le feu jusqu'à ce qu'il ait une belle couleur brune, pas besoin d'attendre que le mélange soit homogène sinon ça sera trop cuit. 
\etape Ajoutez les pommes et faites-les revenir 20 à 25 minutes à feu doux en les retournant de temps en temps pour qu'elles soient caramélisées et dorées uniformément.
\begin{remarque}
Au besoin, augmentez le feu en fin de cuisson pour que l'eau s'évapore (si plus aucune vapeur d'eau ne s'échappe au dessus de la poële, pas la peine de chercher à le faire). Faites attention à ce que le caramel ne brûle pas, il doit juste être un peu épais. 
\end{remarque}

\etape Beurrez un moule à tarte et disposez-y les pommes en cercles sans que les pommes ne touchent le bord du moule. Aidez vous d'une pince ou de deux fourchettes, serrez pour ne pas laisser d'espaces vides. Nappez-les de caramel. 
\etape Placez la pâte sur les pommes en fermant soigneusement autour des pommes et le long du moule. Faire quelques trous pour laisser échapper la vapeur.
\end{preparation}

\begin{cuisson}
Faites cuire 30 minutes à 220\degres C puis laissez reposer 15 minutes avant de retourner la tarte sur un plat de service. Saupoudrez de cannelle et servez tiède ou froid.

\begin{remarque}
Astuce pour démouler sans attacher, replacez à feu vif pour la détacher plus facilement du moule. Dès que le caramel fond, retournez la tarte sur un plat de service.
\end{remarque}
\end{cuisson}
\end{recette}

\begin{recette}{Tiramisu}{4}{}{}
\begin{ingredients}
\ingredient Une boite de mascarpone
\ingredient 400 g de boudoirs
\ingredient 5 œufs
\ingredient Café fort
\ingredient 40g de sucre en poudre
\ingredient Rhum, chocolat Van Houten (cacao pur)
\end{ingredients}

\begin{preparation}
\etape Séparez le blanc du jaune d'œufs
\etape Mélanger le Mascarpone avec le sucre et les jaunes d'œufs
\etape Battez les blancs en neige (avec une pincée de sel) et ajoutez le sucre quand les blancs sont presque prêts\footnote{Les blancs sont prêts quand ils ne tombent pas en retournant le plat.}.
\etape Mélangez ensuite la préparation du mascarpone avec les blancs en neige délicatement. Enveloppez le tout de mouvement circulaires, on longeant les bords et le dessous du récipient avec de ne pas casser les blancs en neige.
\etape Prendre un moule (un plat à gratin ou quelque chose du genre) et saupoudrez le fond de Van Houten
\etape Trempez les boudoirs dans le café fort et le rhum (le mélange doit être froid) puis étalez-les sur le plat.
\begin{remarque}
Les boudoirs ne doivent pas être totalement imbibés, juste l'extérieur, donc ne les attardez pas trop dans le café.
\end{remarque}
\etape Étalez de la crème sur les boudoirs puis saupoudrez de Van Houten
\etape Répétez les deux dernières opérations jusqu'à épuisement des ingrédients (typiquement 2 couches)
\end{preparation}

\begin{remarque}
Préparez le Tiramisu la veille afin de le faire reposer au frigo au moins quelques heures.
\end{remarque}
\end{recette}
