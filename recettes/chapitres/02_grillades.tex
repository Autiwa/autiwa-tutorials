% En règle générale, il vaut mieux y laisser s'imprégner au moins une heure une viande, et parfois, plusieurs heures sont même recommandées. Attention aux poissons, qui \og cuisent \fg sous l'effet du citron notamment. Il est donc recommandé de ne pas les laisser baigner trop longtemps.


\begin{recette}{Marinade Aigre-douce}{0}{10 min + 6h}{}
\begin{ingredients}
\ingredient 2 cuillères à soupe d'huile d'olive
\ingredient 2 cuillères à soupe de vinaigre (de vin ou balsamique, au choix)
\ingredient 2 cuillères à soupe de miel liquide
\ingredient 1 cuillère à soupe de moutarde
\ingredient 1 gousse d'ail hachée
\ingredient une pincée de sel, et un peu de poivre
\end{ingredients}

\begin{preparation}
\etape Mettez les ingrédients dans une poche de congélation
\etape Fermez la poche de manière grossière (en entortillant l'ouverture par exemple) puis secouez jusqu'à ce que la marinade soit homogène
\etape mettez la viande dans la poche, de préférence du porc qui va très bien avec, et laissez reposer au frigo quelques heures, une nuit typiquement
\etape Il ne reste plus qu'à ouvrir la poche et faire griller les morceaux marinés comme de la viande normale.
\end{preparation}

\end{recette}

\begin{recette}{Marinades au vin blanc}{0}{5 min + 6h}{}
\begin{ingredients}
\ingredient $\sfrac{1}{2}$ litre de vin blanc sec
\ingredient $1$ verre d'eau
\ingredient $1$ grosse cuillère à soupe de moutarde forte
\ingredient $1$ grosse cuillère à soupe de moutarde à l'ancienne
\ingredient $1$ grosse cuillère à soupe de thym
\end{ingredients}

\begin{preparation}
\etape Mettez les ingrédients dans une poche de congélation
\etape Fermez la poche de manière grossière (en entortillant l'ouverture par exemple) puis secouez jusqu'à ce que la marinade soit homogène
\etape mettez la viande dans la poche, de préférence du porc qui va très bien avec, et laissez reposer au frigo quelques heures, une nuit typiquement
\etape Il ne reste plus qu'à ouvrir la poche et faire griller les morceaux marinés comme de la viande normale.
\end{preparation}

\begin{remarque}
On peut ajouter du poivre et des oignons à cette marinade \dots  et aussi d'autres herbes\dots
\end{remarque}
\end{recette}

\begin{recette}{Marinade de Porc au paprika}{3}{5 min + 6h}{}
\begin{ingredients}
\ingredient 2 cuillères à soupe d'huile d'olive
\ingredient 1 cuillère à café de paprika
\ingredient herbe de provence, poivre
\end{ingredients}

\begin{preparation}
\etape Mettez les ingrédients dans une poche de congélation
\etape Fermez la poche de manière grossière (en entortillant l'ouverture par exemple) puis secouez jusqu'à ce que la marinade soit homogène
\etape mettez la viande dans la poche, de préférence du porc qui va très bien avec, et laissez reposer au frigo quelques heures, une nuit typiquement
\etape Il ne reste plus qu'à ouvrir la poche et faire griller les morceaux marinés comme de la viande normale.
\end{preparation}

\end{recette}

\begin{recette}{Marinade Tandoori}{3}{5 min + 6h}{}
\begin{ingredients}
\ingredient $2$ cuillères à soupe de Tandoori
\ingredient $2$ cuillères à soupe de jus de citron (ou de vinaigre)
\ingredient $2$ cuillères à soupe d'huile d'olive
\ingredient $1$ ou $2$ yahourt nature
\end{ingredients}

\begin{preparation}
\etape Mettez les ingrédients dans une poche de congélation
\etape Fermez la poche de manière grossière (en entortillant l'ouverture par exemple) puis secouez jusqu'à ce que la marinade soit homogène
\etape mettez la viande dans la poche, de préférence du porc qui va très bien avec, et laissez reposer au frigo quelques heures, une nuit typiquement
\etape Il ne reste plus qu'à ouvrir la poche et faire griller les morceaux marinés comme de la viande normale.
\end{preparation}

\begin{remarque}
Vous pouvez également mariner le poulet dans un peu de yaourt additionné de citron vert, de l'ail écrasé, et du \textbf{curry}. Ces marinades doivent imprégner assez longtemps le poulet.
\end{remarque}

\end{recette}

\begin{recette}{Marinade texane}{4}{15 min + 6h}{}
\begin{remarque}
Normalement, c'est fait avec des coustilles (travers de porc). La viande doit mariner au moins 6 heures.
\end{remarque}
\begin{ingredients}
\ingredient 1 oignon
\ingredient 3 gousses d'ail
\ingredient 2 cuillères à soupe de miel
\ingredient 3 cuillères à soupe de sauce soja
\ingredient 1 cuillère à soupe d'huile d'olive
\ingredient 1 cuillère à soupe de Ketchup
\ingredient 2 cuillères à soupe de vinaigre
\ingredient tabasco, thym, laurier, sel, poivre
\end{ingredients}

\begin{preparation}
\item Je met l'oignon et les gousses d'ails coupées grossièrement dans un mixeur avec le vinaigre et l'huile (ceci permet de mieux couper les morceaux)
\item Je prépare dans un bol le reste de la marinade, puis j'inclue le contenu du mixeur
\item Je mélange puis étale la mixture sur la viande que je laisse mariner quelques heures (environ une nuit)
\end{preparation}

\begin{remarque}
La marinade est un peu épaisse, et il faut l'étaler et non la verser (vu que j'en fais pas beaucoup dans ces cas là, je met la viande dans un plat et étale à la cuillère sur chaque face).
\end{remarque}
\end{recette}
