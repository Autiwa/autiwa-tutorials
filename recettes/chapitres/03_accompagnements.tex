\begin{recette}{Aubergines à la poële}{4}{45 minutes}{}\index{aubergines}
\begin{ingredients}
\ingredient 1kg d'aubergines
\ingredient 2 cuillères à soupe d'huile d'olive
\end{ingredients}

\begin{preparation}
\etape Pelez et coupez en cube les aubergines (environ $1\unit{cm}^3$).
\etape Faites chauffer l'huile puis mettez les aubergines en remuant pour bien répartir l'huile. 
\etape Remuez régulièrement jusqu'à ce que les aubergines soit dorées et moelleuses.
\end{preparation}

Le goût peut faire penser à des champignons, c'est très fin et je n'ai pas encore trouvé comment utiliser ce goût.

\begin{remarque}
La quantité d'huile est un point crucial. Trop, et ça sera trop gras, mais pas assez, et ça cramera au lieu de dorer. Notez qu'au début, tant que les aubergines ne sont pas réduites, il n'est pas bon de rajouter de l'huile tant que ça absorbe ou tant qu'on voit que ça ne dore pas. En effet, ce qui compte c'est à la fin. Car quand les aubergines vont réduire, l'huile qu'elles ont absorbé va ressortir et ça ne sert donc à rien d'en mettre trop. 

C'est quelque chose qui se voit à l'œil à force de faire la recette.
\end{remarque}

\end{recette}

\begin{recette}{Carotte à la crème}{3}{20 min}{1h}\index{carottes}\index{crème}
\begin{ingredients}
\ingredient 1kg de carottes
\ingredient 2 échalotes
\ingredient 2 oignons
\ingredient 1 gousse d'ail
\ingredient 20g de beurre
\ingredient 20cl de bouillon (eau + bouillon cube de boeuf ou de volaille)
\ingredient 20 cl de crème fraîche épaisse
\ingredient sel, poivre, cerfeuil (frais ou pas)
\end{ingredients}

\begin{preparation}
\etape Éplucher et couper en rondelles les carottes. Peler et émincer les échalotes, les oignons et l’ail. Hacher le cerfeuil.

\etape Dans une cocotte, faire suer dans le beurre les oignons, les échalotes et l’ail pendant 4 minutes sur feu doux.

\etape Ajouter les rondelles de carottes et prolonger la cuisson 5 minutes en remuant de temps en temps. Il faut que les carottes commencent à rendre un peu de jus.

\etape Verser le bouillon, saler et poivrer. Couvrir et laisser cuire sur feu moyen pendant 25 minutes (ajouter de l'eau s'il n'en reste plus, les carottes doivent tout absorber).

\begin{remarque}
Les carottes doivent être cuites avant d'ajouter la crème. Une fois la crème ajoutée, ça cuit beaucoup moins vite.
\end{remarque}

\etape Ajouter la crème fraîche et le cerfeuil, bien mélanger. Goûter et rectifier l’assaisonnement si nécessaire. Laisser cuire encore 10 minutes sur feu doux.

\etape Servir aussitôt pour accompagner un rôti de porc ou des escalopes .
\end{preparation}

\end{recette}

\begin{recette}{Poëlée forestière}{4}{}{}\index{pomme de terre}
\begin{ingredients}
\ingredient $500\unit{g}$ de pommes de terre coupées en dé
\ingredient $4$ ou $5$ oignons
\ingredient $200\unit{g}$ de lardons
\end{ingredients}

\begin{preparation}
\etape Faites cuire les lardons
\etape Une fois cuits, sortez les et faites revenir les oignons à feux doux dans la graisse des lardons en en rajoutant au besoin. Tournez les de temps en temps jusqu'à ce qu'ils soit dorés.
\etape sortez les et mettez les avec les lardons. Maintenant mettez les pommes de terre, surgelés ou coupées préalablement, à cuire à feux doux jusqu'à ce qu'elles soit cuites, et dorées. Il est important de les laisser cuire à feux doux, et de ne pas changer augmenter le feu pendant la cuisson.
\etape Une fois les pommes de terres cuites, ajoutez les oignons et les lardons, remuez de sorte à obtenir un mélange homogène et laissez le temps que les oignons et lardons se réchauffent, remuez et servez.
\end{preparation}

\end{recette}

\begin{recette}{Pommes de terre marinées au four}{2}{}{}\index{pomme de terre}
\begin{ingredients}
\ingredient Environ 3.5 pommes de terre par personne.
\ingredient paprika, herbes de provence, huile d'olive
\ingredient un sac de congélation
\end{ingredients}

\begin{preparation}
\etape Dans le sac de congélation mettez un peu d'huile d'olive, une cuillère à café de paprika et un peu d'herbe de provence. Fermer le fond du sac en spiralant la poche, puis mélangez en la secouant.
\etape Lavez les pommes de terre et fendez les en deux dans l'épaisseur puis encore en deux comme des grosses frites.
\etape Mettez les dans la marinade.
\end{preparation}

\begin{cuisson}
Faites préchauffer le four à 200\degres C 10 minutes environ, puis enfournez les 30 minutes en les disposant dans un grand plat à tarte.
\end{cuisson}
\end{recette}

\begin{recette}{Pommes de terre au four}{2}{}{}\label{sec:pomme-de-terre-four}\index{pomme de terre}
\begin{ingredients}
\ingredient Environ 3.5 pommes de terre par personne.
\ingredient sel, poivre, huile d'olive
\ingredient un sac de congélation
\end{ingredients}

\begin{preparation}
\etape Dans le sac de congélation mettez un peu d'huile d'olive, du sel et du poivre. Fermer le fond du sac en spiralant la poche, puis mélangez en la secouant.
\etape Lavez les pommes de terre et fendez les en deux dans l'épaisseur puis encore en deux comme des grosses frites.
\etape Mettez les dans la marinade.
\end{preparation}

\begin{cuisson}
Faites préchauffer le four à 200\degres C 10 minutes environ, puis enfournez les 30 minutes en les disposant dans un grand plat à tarte.
\end{cuisson}

\end{recette}

\begin{recette}{Pommes de terre vinaigrette}{2}{}{}\index{pomme de terre}
\begin{ingredients}
\ingredient 2 grosses pommes de terre par personne
\ingredient 1 échalote hachée
\ingredient 2 cuillères à soupe de moutarde
\ingredient 4 cuillères à soupe d'huile d'olive
\ingredient 2 cuillères à soupe de vinaigre de vin
\ingredient ciboulette
\ingredient sel, poivre
\end{ingredients}

\begin{preparation}
\etape Faire bouillir une grande casserole d'eau. Eplucher les pommes de terre et les couper en morceaux. Jeter les pommes de terre dans l'eau bouillante et les faire pendant au moins 25mn (plus selon la taille des morceaux). Bien vérifier que les morceaux soient cuits au centre.

\etape Égoutter et faire refroidir les pommes de terre. Préparer la vinaigrette en mélangeant l'échalote hachée, la moutarde, l'huile d'olive et le vinaigre. Saler, poivrer.

\etape Dans un saladier, mélanger la sauce et les pommes de terre et rectifier l'assaisonnement si nécessaire.

\begin{remarque}
Mettre au frais si vous préférez la salade de pommes de terre froide que chaude.
\end{remarque}

\etape Au moment de servir parsemer de ciboulette.
\end{preparation}
\end{recette}

\begin{recette}{Semoule}{2}{10 min}{}\index{semoule}
\begin{ingredients}
\ingredient semoule de blé dur (compter 60g par personne si accompagnement, sinon 100g)
\ingredient eau (même volume que la semoule (pas encore cuite)
\ingredient huile d'olive, sel
\end{ingredients}

\begin{preparation}
\etape À l'aide d'un verre doseur, choisissez une quantité de semoule (60g par exemple)
\etape Ajoutez un filet d'huile à la semoule et mélangez bien afin que l'huile soit répartie autour des grains
\etape Faites bouillir une quantité d'eau égale au volume de la semoule que vous souhaitez faire cuire (par exemple, pour 100g de semoule, c'est environ $12.5\unit{cl}$. 
\begin{remarque}
Je fais chauffer l'eau au micro onde pour ma part. C'est rapide et même si je ne vois pas les bulles, l'eau est quand même bien chaude. 
\end{remarque}
\etape Ajoutez alors l'eau avec la semoule, et remuez bien avec une fourchette afin que ce soit homogène. Laissez à couvert (dans un bol avec une assiette par exemple) pendant 5 à 10 minutes.
\etape Égrénez enfin la semoule avec une fourchette afin de bien séparer les grains.
\end{preparation}
\end{recette}
