\documentclass[a4paper,twoside]{article}
\usepackage{autiwa}
\usepackage{listings}
\lstset{language=IDL,
basicstyle=\ttfamily\small,
columns=flexible,
escapechar=+}
\lstset{keywordstyle=\color{blue}\bfseries}

\title{Aide mémoire Subversion (svn)}
\author{Autiwa}

\newcommand{\raccourci}[1]{{\bfseries #1}}

\makeindex
\begin{document}

\tableofcontents

\clearpage

\section{Préambule}
Subversion permet de gérer un projet (de programmation généralement) et de garder en mémoire l'historique de toutes les versions d'un ensemble de fichiers. Il permet de gérer un projet à plusieurs, de programmer afin de pouvoir revenir en arrière, comparer avec d'anciennes versions et cie. 

Le principe est d'avoir un serveur subversion (un seul possible) qui va garder en mémoire l'historique de toutes les versions et un client subversion (plusieurs possibles) qui vont se connecter au serveur pour mettre à jour la version des fichiers ou en récupérer les dernières versions.

\begin{remarque}
Il est possible que le serveur soit lui aussi client, dans le cas où il n'y aurait qu'un seul développeur et qu'on ne souhaite pas passer par internet.
\end{remarque}

\section{Subversion sur internet}
Je vais prendre l'exemple de google code, qui est celui que j'ai choisi et que je suis en train d'apprendre.

\subsection{Récupérer le contenu du projet}\label{sec:checkout}
Une fois le projet créé (sur la page \url{http://code.google.com/hosting/createProject}), il faut faire : 
\begin{verbatim}
svn checkout --username autiwa@gmail.com --password "passwd" \
https://autiwa-tutorials.googlecode.com/svn/trunk/ Formulaires
\end{verbatim}

Cette commande permet de récupérer le contenu du projet et de le copier dans un dossier \texttt{Formulaires} qui sera créé dans le dossier courant.

\begin{definition}[Checkout]
Opération d'extraction d'une version d'un projet du repository vers un répertoire de travail local.
\end{definition}


\subsection{Appliquer les changements qu'on vient de faire localement : commit}\label{sec:commit}\index{commit}\index{mise à jour local -> serveur}

Pour mettre à jour les versions sur serveur à partir des modifications effectuées localement, il faut : 
\begin{verbatim}
svn commit -m "initialisation" --username autiwa@gmail.com --password passwd
\end{verbatim}
où \texttt{"initialisation"} est le commentaire qui décrit la mise à jour et les modifications effectuées, \texttt{passwd} étant le mot de passe associé à votre identifiant (ici \texttt{autiwa@gmail.com} pour moi).

\begin{remarque}

\end{remarque}



\subsection{Ajouter des fichiers ou dossiers au projet}\index{add}
Pour ajouter des fichiers il faut faire :
\begin{verbatim}
svn add latex/ vim/ 
\end{verbatim}
où \texttt{latex/} et \texttt{vim/} sont deux dossiers existant dans le dossier local de référence

\begin{remarque}
Au cas où ça serait pas clair. J'ai créé un dossier \texttt{/home/autiwa/Formulaires} grâce à \refsec{sec:checkout}. Dans ce dossier, j'ai créé et rempli à la main les sous-dossiers \texttt{latex/} et \texttt{vim/}. Maintenant, grâce à la commande ci-dessus, je définis ces sous-dossiers comme étant rattachés au projet. En faisant ainsi le contenu est rajouté récursivement.
\end{remarque}

\begin{attention}
Cette commande n'agit que sur le répertoire local (la \emph{working copy}). Il faut ensuite faire un \texttt{commit} (voir \refsec{sec:commit}) pour valider les changements sur le serveur.
\end{attention}

\subsection{Supprimer des fichiers ou dossiers au projet}\index{delete}
Pour supprimer des fichiers il faut faire :
\begin{verbatim}
svn delete latex/ vim/ 
\end{verbatim}
où \texttt{latex/} et \texttt{vim/} sont deux dossiers existant dans le dossier local de référence. 

\begin{attention}
Cette commande n'agit que sur le répertoire local (la \emph{working copy}). Il faut ensuite faire un \texttt{commit} (voir \refsec{sec:commit}) pour valider les changements sur le serveur.
\end{attention}



\end{document}