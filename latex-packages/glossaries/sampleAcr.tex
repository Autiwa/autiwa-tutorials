\documentclass[a4paper]{report}

\usepackage[colorlinks,plainpages=false]{hyperref}
\usepackage[style=long,toc]{glossaries}

\makeglossaries

\newacronym{svm}% label
{svm}% abbreviation
{support vector machine}% long form

\newacronym{ksvm}{ksvm}{kernel support vector machine}

\newacronym{rna}{rna}{ribonukleins\"aure}

\begin{document}
\tableofcontents

\chapter{Support Vector Machines}

\Glspl{svm} are used widely in the area of pattern recognition.

This is the text produced without a link: \glsentrytext{svm}.
This is the text produced on first use without a link:
\glsentryfirst{svm}. This is the entry's description without
a link: \glsentrydesc{svm}.

This is the entry in uppercase: \GLS{svm}.

\chapter{Kernel Support Vector Machines}

The \gls{ksvm} is \ifglsused{svm}{an}{a} \gls{svm} that uses
the so called ``kernel trick''. Plural: \glspl{ksvm}.

\glsresetall
Possessive: \gls{ksvm}['s].
Make the glossary entry number bold for this
one \gls[format=hyperbf]{svm}.

\chapter{Another chapter}

You don't need to worry about makeindex's special characters:
\gls{rna}.

\printglossary[title={Acronyms}]

\end{document}
