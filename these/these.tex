\documentclass[a4paper,twoside]{book}
\usepackage[french]{autiwa}
\title{Migration et accrétion d'embryons planétaires dans un disque radiatif}
\author{C. Cossou, S. Raymond \and A. Pierens}

\begin{document}
\tableofcontents
\newpage



\chapter{Introduction}
%TODO parler de formation planétaire (en citant des papiers, sans rentrer dans les détails. notamment pollack, alibert)

%TODO une des grandes questions c'est  : comment on forme des noyaux de jupiter et Kepler 11?



\chapter{Physique des disques}
\section{Les disques protoplanétaire}
%TODO voir thèse mordasini, et les articles de mordasini, alibert, ida et lin, histoire de voir ce qu'ils font)
\subsection{Formation et évolution}
\subsection{Propriétés}
\subsubsection{Profil de densité}
\subsubsection{Profil de température}
%TODO parler de la température du disque (et les phénomènes principaux qui ont un effet sur la température, chauffage visqueux, irradiation de l'étoile, irradiation externe. Parler dans cette partie de l'opacité, des transitions et à quoi c'est dû, des modèles had oc pour l'opacité et des incertitudes qui en découlent

\subsection{Les bords du disque}
%TODO parler des bords du disque et de tous les problèmes que ça pose

\section{Interaction disque-planète}
\subsection{Migration planétaire}
\paragraph{Type I}
\paragraph{Type II}
\paragraph{Type III}

\subsection{L'amortissement de l'excentricité}%circularisation
%TODO parler des autres phénomènes importants dans le disque, comme l'amortissement de l'excentricité

\subsection{L'amortissement de l'inclinaison}%coplanarisation
%TODO parler de l'amortissement de l'inclinaison, 

\subsection{L'accrétion du gaz}
%TODO parler de l'accrétion, et du fait que ça va créer des planètes géantes notamment

\chapter{Le Code N-Corps}
\section{Présentation de mercury}
\section{Disque 1D}
\subsection{Profil de densité de surface}
\subsection{Table d'opacité}
\subsection{Profil de température}
\section{Migration type I}
%TODO paardekooper
\section{Amortissement de e et I}
\section{Effet de l'excentricité sur le couple de corotation}
%TODO du couple de corotation quand l'excentricité augmente, mais je sais pas encore où le placer, pas dans la partie intro je suppose

\section{Outils d'analyse}
\subsection{Diagrammes de couple a-m}

\chapter{Mécanismes individuels}
\section{Les Résonances de Moyen Mouvement (MMR)}
\subsection{Résonances et excentricité}
\subsection{Stabilité et ordre des résonances}

\section{Les Zones de Convergence}
\subsection{Existence et intérêt}
\subsection{Les différents types}
\subsection{Résonances et Accrétions}

\chapter{Mécanismes de formation}
\section{Décalage de la Zone de Convergence}
\section{Formation des super terre chaude}
\section{Effets des paramètres du disque}
\subsection{Viscosité du disque}
\subsection{Profil de densité de surface}
\subsection{Profil de température}
\subsection{Masse du disque}
\subsection{Table d'opacité}

\chapter{Discussion et limite du modèle}
\section{Étude de sensibilité}
\subsection{Le choix de la table d'opacité et son implémentation}
%TODO le choix de la table, mais aussi le fait qu'on a besoin d'une densité volumique, ou qu'on a besoin de la masse moléculaire moyenne. 

\subsection{La modélisation des bords du disque}

\subsection{Pas d'effet indirect des ondes de densité sur les autres planètes}

\subsection{Modélisation de la viscosité}

\subsection{Profil de densité du gaz en 2D}

\chapter{Conclusion}

\end{document}
