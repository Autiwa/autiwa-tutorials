\documentclass[logos,parttoc]{bordeaux-thesis}

%########################################################################
% Extensions
%########################################################################

%Text encoding and fonts
\usepackage[utf8]{inputenc}
\usepackage[T1]{fontenc}
\usepackage{autiwa-macros}
%\usepackage{graphicx}

% for the french text adaptation (babel is already in the class)
\FrenchFootnotes % pour les notes de bas de page
\AddThinSpaceBeforeFootnotes % pour les puristes (donc tout le monde!) qui veulent une espace fine entre le mot et l'appel de note


\renewcommand\theequation{\thechapter.\arabic{equation}}
\makeatletter
\@addtoreset{equation}{chapter}
\makeatother

\usepackage{natbib}%améliore la bibliographie, surtout pour citer les publications scientifiques.
\usepackage{frbib}%pour mettre la bibliographie en français
\usepackage{aas_macros}% contient les macros pour la définition des noms de journaux des publications scientifiques utilses pour les enterées bibtex de ADS.
\usepackage[nottoc]{tocbibind}%pour que la bibliographie  et l'index apparaissent dans la table des matières. l'option permet de ne pas afficher la table des matières. (notlof, notlot pour table des figures et ou des tableaux)

\usepackage{xspace}%Permet à babel d'utiliser la macro xspace  partout où c'est nécessaire. (Voir la doc de babel pour de plus amples explications.)

%########################################################################
% Title page
%########################################################################

%Thesis author
\author{Christophe \textsc{Cossou}}

%Title for main language (french)
\title{Migration et accrétion d'embryons planétaires dans un disque radiatif}
%Titles for other languages
\title[english]{My English thesis title}


%Keywords for main language (french)
\keywords{Blabla, blabla, blabla, blabla, blabla, blabla, blabla}
%Keywords for other languages languages
\keywords[english]{Blabla, blabla, blabla, blabla, blabla, blabla, blabla}

%Order number of the thesis
\ordernumber{1234}

%Date of defense
\date{xx Xxxxxxxxx xxxx}

%You define the commission member list using \addcommissionmember (mandatory) with an optional role (eg: president, supervisor, etc...)
\addcommissionmember{M.}{Aaaaa}{Bbbbbbbb}
\addcommissionmember{M.}{Ccccccccc}{Dddddddddd}
\addcommissionmember[Directeur de thèse]{M.}{Eee}{Fffff}
\addcommissionmember{M.}{Ggggggggg}{Hhhhhhh}
\addcommissionmember[Président du jury]{M.}{Iiii}{Jjjjjjjjjjjj}
\addcommissionmember{M.}{Kkkkkkkkkkkk}{Llllll}

%If some referees are not part of the commission, you can add them in a separate list with \addreferee (optional)
\addreferee{M.}{Mmmmmmmm}{Nnnnnnnn}
\addreferee{M.}{Oooooooooo}{Pppppppp}

\newcommand{\dummytext}{
Lorem ipsum dolor sit amet, consectetuer adipiscing elit. Phasellus blandit massa non tellus. Pellentesque blandit. Etiam sapien. Quisque sed massa ac tortor accumsan bibendum. Donec et orci quis mi sollicitudin consectetuer. Donec malesuada. Pellentesque bibendum pellentesque elit. Morbi et diam ac wisi auctor fringilla. Cras nec arcu sed velit dapibus blandit. Maecenas mollis aliquet quam. In eget sem nec orci fringilla sagittis. Suspendisse cursus placerat massa. Pellentesque non metus. Morbi congue tellus eget tellus. Suspendisse justo. Suspendisse potenti. Praesent interdum lorem in velit. Nullam sit amet nisl eget wisi consectetuer consequat. Mauris vel felis. Nulla sed neque.

Nulla facilisi. Maecenas accumsan gravida wisi. Maecenas sodales gravida neque. Mauris in est a ante molestie gravida. In id neque. Ut augue. Duis fringilla ullamcorper risus. Nullam at lorem. Quisque consequat turpis ac libero. Ut auctor ante commodo magna. Donec in magna. Integer sodales. Donec ac nibh eu felis suscipit elementum.

Fusce convallis dolor sit amet dolor. Nulla sit amet pede. Maecenas et ante vitae risus tempus facilisis. Nullam ut tellus et lacus sollicitudin condimentum. Maecenas vitae lorem. Quisque nec leo varius est euismod posuere. Integer ac diam in enim pellentesque pulvinar. Etiam sodales tristique eros. Curabitur non magna. Suspendisse blandit metus vitae purus. Phasellus nec sem vitae arcu consequat auctor. Donec nec dui. Donec sit amet lorem vel erat tristique laoreet. Duis ac felis tincidunt arcu consequat faucibus. Vestibulum ultrices porttitor purus. In semper consequat dolor. Nunc porta. Vestibulum nisl ipsum, rhoncus quis, adipiscing sed, sollicitudin ut, quam.
}


%########################################################################
% Document start
%########################################################################

\begin{document}

%Print title NOW
\maketitle%

%Disable page numbering
\pagestyle{empty}

%########################################################################
% Multilingual abstracts
%########################################################################

%French abstract:
\begin{abstract}
\dummytext
\end{abstract}

%Horizontal rule
\noindent\hspace*{0.35\textwidth}\hrulefill\hspace*{0.35\textwidth}\\[-\bigskipamount]

%English abstract:
\begin{abstract}[english]
\dummytext
\end{abstract}

%########################################################################
% Acknowledgments
%########################################################################

\pagebreak\strut\newpage

\chapter*{Remerciements}
%Put the text vertically centered
\vfill
\dummytext
\vfill

\newpage

%########################################################################
% Contents
%########################################################################

\strut\newpage

\tableofcontents

\newpage

%########################################################################
% Introduction
%########################################################################

%Enable page numbering
\pagestyle{fancy}

\chapter*{Introduction}
\addstarredchapter{Introduction}% To be used instead of addcontentsline in order to have the good minitoc. If not, the starred chapter create a shift in the minitocs.
%%\addcontentsline{toc}{chapter}{Introduction}

%TODO parler de formation planétaire (en citant des papiers, sans rentrer dans les détails. notamment pollack, alibert)

%TODO une des grandes questions c'est  : comment on forme des noyaux de jupiter et Kepler 11?



\chapter{Physique des disques}
\section{Les disques protoplanétaire}
%TODO voir thèse mordasini, et les articles de mordasini, alibert, ida et lin, histoire de voir ce qu'ils font)
\subsection{Formation et évolution}
%TODO 
\subsection{Propriétés}
%TODO 
\subsubsection{Profil de densité}
%TODO 
\subsubsection{Profil de température}
%TODO parler de la température du disque (et les phénomènes principaux qui ont un effet sur la température, chauffage visqueux, irradiation de l'étoile, irradiation externe. Parler dans cette partie de l'opacité, des transitions et à quoi c'est dû, des modèles had oc pour l'opacité et des incertitudes qui en découlent

\subsection{Les bords du disque}
%TODO parler des bords du disque et de tous les problèmes que ça pose

\section{Interaction disque-planète}
\subsection{Migration planétaire}
%TODO 
\paragraph{Type I}
%TODO 
\paragraph{Type II}
%TODO 
\paragraph{Type III}
%TODO 

\subsection{L'amortissement de l'excentricité}%circularisation
%TODO parler des autres phénomènes importants dans le disque, comme l'amortissement de l'excentricité

\subsection{L'amortissement de l'inclinaison}%coplanarisation
%TODO parler de l'amortissement de l'inclinaison, 

\subsection{L'accrétion du gaz}
%TODO parler de l'accrétion, et du fait que ça va créer des planètes géantes notamment

\chapter{Le Code N-Corps}
Afin d'étudier la formation planétaire et les interactions avec le disque de gaz, j'ai utilisé un code de simulation N-corps, qui permet de regarder l'évolution d'un nombre arbitraire de corps orbitant autour d'un astre central. 

Ce choix est apparu naturellement. Au début de ma thèse j'ai fait quelques simulations hydrodynamiques avec le code Genesis développé par Arnaud Pierens. J'ai rapidement constaté que ce genre de simulations, bien que modélisant de manière poussée le disque, ne permettait pas d'étudier de manière approfondie la dynamique planétaire. Le temps de calcul nécessaire pour une simulation limite en effet grandement le nombre de corps ainsi que la durée d'intégration. J'ai donc souhaité me tourner vers un code N-corps, afin de privilégier la dynamique planétaire, et de modifier ce programme afin d'y inclure les effets d'un disque de gas sur la dynamique planétaire. 

J'ai ainsi gagné en temps de calcul, et j'ai ouvert un vaste champ d'investigation sur les paramètres du disques, le nombre de corps en interaction, me permettant de faire des systèmes planétaires très divers, parfois avoir plusieurs centaines d'embryons pour plusieurs millions d'années, chose impossible dans les simulations hydrodynamiques du début de ma thèse où 20 corps pendant quelques dizaines de milliers d'années était un maximum. 

Ce choix a bien entendu introduit son lot d'incertitudes et d'approximations qui sont discutés dans la partie \refsec{sec:discussion}. La présente section a pour but de présenter le code N-corps que j'ai utilisé ainsi que les différents effets du disque que j'ai modélisé. J'ai avant tout souhaité présenter les parties qui ont des conséquences sur la physique du disque, que ce soit en terme de choix d'un modèle particulier, ou de limitations numériques qu'il est bien de garder à l'esprit quand on interprète les résultats.

\section{Présentation de mercury}
Le code N-corps choisi est le code \textbf{mercury} \citep{chambers1999hybrid}. Ce code offre la possibilité de choisir un algorithme parmi 5 différents (BS, BS2, RADAU, MVS et HYBRID), ayant des propriétés diverses. Dans le cadre de ma thèse, je n'ai utilisé que l'algorithme HYBRID, qui utilise l'algorithme MVS la plupart du temps, mais change pour l'algorithme BS2 lors de rencontres proches. Il est possible de déterminer à quel moment on considère qu'une rencontre est "proche" dans le fichier de paramètre de programme, j'ai laissé le paramètre par défaut. 

La raison de ce changement est assez simple. MVS est un algorithme symplectique, c'est à dire à pas de temps constant, dans lequel on défini un hamiltonien que l'on résout pour faire évoluer les orbites. La conservation de l'énergie est moins bonne que pour un algorithme à pas de temps adaptatif, mais le point très important est que cette conservation de l'énergie est bien meilleure au cours du temps. C'est à dire que là où les algorithmes tels que BS, BS2 et RADAU verront leur erreur sur l'énergie augmenter au cours du temps, les algorithmes symplectiques vont eux voir leur erreur rester plus ou moins constante au cours du temps. 

Dans le cadre de mes simulations, j'ai accordé une importance limitée aux variations d'énergie, étant donné que les couples que l'on rajoute pour simuler la présence du disque de gaz font que l'énergie n'est pas conservée pour une planète donnée. Cependant, il est important de bien résoudre les orbites et c'est ce point qui est le plus crucial ici. En effet, quelques tests ont permis de contraindre le pas de temps minimal qu'il est nécessaire d'avoir en fonction de la distance orbitale d'une planète. La contrainte de pas de temps dans mes simulations vient donc d'une distance minimale en dessous de laquelle les orbites ne sont pas correctement calculées. Cette limite, afin d'éviter tout problème, est choisie pour être en dessous du bords interne du disque de gaz que je défini.

%TODO Simulation lancées dans le dossier $sse/test_mercury, sur une même simulation mais avec différents algorithmes pour voir l'évolution de la conservation de l'énergie au cours du temps

%TODO parler de la précision de conservation de l'énergie
%TODO parler du pas de temps qui a une influence sur les orbites, et des tests effectués pour contraindre le pas de temps par rapport à l'orbite minimale accessible.

\section{Disque 1D}
Afin de calculer les effets d'un disque de gaz, une modélisation de ce dernier est nécessaire. Le but étant d'avoir une grande souplesse, le disque implémenté est bien entendu très simplifié. Toutes les quantités sont intégrées et invariantes selon la hauteur z et la position azimutale $\theta$ dans le disque, résultant en un modèle radial de toutes les quantités. 

Dans la mesure du possible, les quantités du disque ont été calculées de manière consistante. Je vais présenter dans la suite de manière chronologique comment sont calculées les grandeurs physiques du disque.
%TODO 
\subsection{Profil de densité de surface}
Le profil de densité de surface est défini au début de la simulation comme une loi de puissance de la forme :
\begin{align}
\Sigma(R) &= \Sigma_0 \times R^{-\alpha}
\end{align}
où $\Sigma_0$ est la densité de surface à $1\unit{AU}$ et $\alpha$ l'indice de la loi de puissance. 

Ce profil de densité de surface est défini pour une certaine étendue radiale. On défini donc un bord interne $R_\text{in}$ et un bord externe $R_\text{out}$. Le bord interne est généralement à $0,1\unit{AU}$ et le bord externe à $100\unit{AU}$. 

Afin de calculer les valeurs suivantes, ce disque est échantillonné et toutes les valeurs nécessaires sont ensuite calculées à chacun de ces points. 

\subsection{Table d'opacité}
%TODO 
\subsection{Profil de température}
%TODO 
\section{Migration type I}
%TODO 
%TODO paardekooper
\section{Amortissement de e et I}
%TODO 
\section{Effet de l'excentricité sur le couple de corotation}
%TODO du couple de corotation quand l'excentricité augmente, mais je sais pas encore où le placer, pas dans la partie intro je suppose

\chapter{Mécanismes individuels}
\section{Les Résonances de Moyen Mouvement (MMR)}
%TODO 
\subsection{Résonances et excentricité}
%TODO 
\subsection{Stabilité et ordre des résonances}
%TODO 

\section{Les Zones de Convergence}
%TODO 
\subsection{Existence et intérêt}
%TODO 
\subsection{Les différents types}
%TODO 
\subsection{Diagrammes de couple a-m}
%TODO parler des raisons pour lesquelles la zone de convergence dépend de la masse et de la distance parfois, avec les comparaisons des temps (dynamique, de U-turn and de diffusion)
\subsection{Résonances et Accrétions}
%TODO 

\chapter{Mécanismes de formation}
\section{Décalage de la Zone de Convergence}
%TODO 
\section{Formation des super terre chaude}
%TODO 
\section{Effets des paramètres du disque}
%TODO 
\subsection{Viscosité du disque}
%TODO 
\subsection{Profil de densité de surface}
%TODO 
\subsection{Profil de température}
%TODO 
\subsection{Masse du disque}
%TODO 
\subsection{Table d'opacité}
%TODO 

\chapter{Discussion et limite du modèle}\label{sec:discussion}
\section{Étude de sensibilité}
%TODO 
\subsection{Le choix de la table d'opacité et son implémentation}
%TODO le choix de la table, mais aussi le fait qu'on a besoin d'une densité volumique, ou qu'on a besoin de la masse moléculaire moyenne. 

\subsection{Modélisation de la viscosité}
%TODO 



\section{Approximations}
%TODO 
\subsection{Profil de densité du gaz en 2D}
%TODO 
\subsection{La modélisation des bords du disque}
%TODO 
\subsection{Pas d'effet indirect des ondes de densité sur les autres planètes}
%TODO 
\subsection{Auto-gravité}
%TODO 

\chapter*{Conclusion}
\addstarredchapter{Conclusion} % to be used in place of addcontentsline to avoid problems with minitoc
%%\addcontentsline{toc}{chapter}{Conclusion}

%TODO 

%\thispagestyle{empty}
%\strut\newpage

\bibliographystyle{plainnat}
\bibliography{these}%.bib

\end{document}
