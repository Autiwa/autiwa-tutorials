\documentclass[a4paper, 12pt]{report}

%twoside : mode recto-verso
%openright : Pour que les chapitres apparaissent en page impaire

\usepackage[utf8]{inputenc}
\usepackage[frenchb, english]{babel}                        % Document en fran�ais
\usepackage[T1]{fontenc}                             % Utilisation du codage de caract�res T1, nouvelle norme LaTeX 


%% r�glage des marges
%\setlength{\oddsidemargin}{0.46cm}
%\setlength{\evensidemargin}{-0.46cm}
%\setlength{\topmargin}{-0.6cm}
%\setlength{\textheight}{24cm}  % Hauteur de la page
%\setlength{\textwidth}{16cm}   % Largeur de la page

\begin{document}

\thispagestyle{empty}

\vfill

\begin{center}
\textsc{Effet de la structure du disque sur la formation et la migration des planètes}
\end{center}

\begin{center}
\textbf{Résumé}
\end{center}

Au delà du système Solaire et de ses planètes, nous avons maintenant un catalogue de quasiment 1000 exoplanètes qui illustrent la grande diversité des planètes et des systèmes qu'il est possible de former. Cette diversité est un défi que les modèles de formation planétaire tentent de relever. La migration de type 1 est un des mécanismes pour y parvenir. En fonction des propriétés du disque protoplanétaire, les planètes peuvent s'approcher ou s'éloigner de leur étoile. La grande variété des modèles de disques protoplanétaires permet d'obtenir une grande variété de systèmes planétaires, en accord avec la grande diversité que nous observons déjà pour l'échantillon limité qui nous est accessible. Grâce à des simulations numériques, j'ai pu montrer qu'au sein d'un même disque, il est possible de former des super-Terres ou des noyaux de planètes géantes selon l'histoire de migration d'une population d'embryons.

\bigskip

\noindent\textbf{Mots-clefs : } Formation planétaire, migration, Disques protoplanétaires, Interactions Disque-Planète, Systèmes Planétaires, Simulations numériques

\bigskip

%Horizontal rule
\noindent\hspace*{0.25\textwidth}\hrulefill\hspace*{0.25\textwidth}

\vfill

\selectlanguage{english}

\begin{center}
\textsc{Effect of the disc structure on planets formation and migration}
\end{center}

\begin{center}
\textbf{Abstract}
\end{center}

In addition to the Solar System and its planets, we now have a database of nearly 1000 planets that emphasize the huge diversity of planets and systems that can be formed. This diversity is a challenge for planetary formation models. Type I migration is one of the mechanisms possible to explain this diversity. Depending on disc properties, planets can migrate inward or outward with respect to their host star. The huge parameter space of protoplanetary disc models can form a huge diversity of planetary systems, in agreement with the diversity observed in the nonetheless small sample accessible to us. Thanks to numerical simulations, I showed that within the same disc, it is possible to form super-Earths or giant planet cores, depending on the migration history of an initial population of embryos.

\bigskip

\noindent\textbf{Keywords: } Planets and satellites: formation, Protoplanetary disks, Planet-disk interactions, planetary systems, Methods: numerical

\vfill

\end{document}