Afin de conclure ce travail, nous allons dans un premier temps récapituler les résultats principaux de cette thèse. Nous nous attarderons ensuite sur certaines limitations importantes des modèles utilisés pour ensuite parler des perspectives futures qui découlent de ce présent travail. 

\section{Résultats}
\subsection{Cartes de migration}\index{carte de migration}
Nous avons vus dans le chapitre \refsec{sec:chap3} que les paramètres du disque avaient une influence sur la carte de migration de ce dernier. 

La viscosité et la densité de surface ont une influence très importante sur la carte de migration. Ces grandeurs, bien que dépendant intrinsèquement du disque considéré, vont aussi varier au cours de la dissipation du disque. Comprendre l'évolution de la migration en fonction de ces paramètres nous permet donc de mieux comprendre comment va se comporter une planète au cours de la vie du disque. Au cours de la dissipation du disque, la viscosité va, au même titre que la densité de surface, diminuer au cours du temps \citep[Fig. 16]{guilloteau2011dual}. La décroissance de ces deux paramètres, prise séparément, va dans le même sens, c'est à dire le déplacement des zones de convergence vers les parties internes du disque. De plus, au fur et à mesure de la dissipation du disque, les zones de convergences disparaissent.

Il est à noter cependant que les zones de convergences ont tendance à disparaître rapidement pour les planètes massives. Mais au fur et à mesure de la dissipation du disque, des planètes peu massives peuvent migrer vers l'extérieur alors qu'elles ne pouvaient pas le faire quand le disque était plus massif. Tard dans la vie du disque, des zones de convergences pour les planètes de faible masse, de l'ordre d'une masse terrestre pourraient jouer un rôle important dans la formation des planètes telluriques.

\subsection{Formation de super-Terres chaudes et de noyaux de planètes géantes}\index{super-Terre}\index{noyau de planète géante}
Dans le chapitre \refsec{sec:chap4}, j'ai montré qu'une même population statistique d'embryons pouvait aboutir à la formation d'un système compact de super-Terres chaudes ou à la formation d'un ou plusieurs embryons de planètes géantes. Dans le modèle, un noyau de planète géante est un embryon planétaire qui devient massif ($m>5\mearth$) suffisamment rapidement pour inverser sa migration et qu'il reparte vers l'extérieur, vers une zone de convergence située autour de $15\unit{UA}$ dans le disque présenté. Dans ce même modèle, nous obtenons des super-Terres si les embryons migrent vers l'intérieur plus rapidement qu'ils ne grossissent en masse. Ainsi, cette masse est transportée dans les régions internes où un système compacts d'embryons en résonance de moyen mouvement apparaît, maintenu dans le disque grâce au fort couple de corotation qui apparaît au bord interne.

Pour un même disque de gaz, si on diminue la quantité de masse disponible pour les embryons planétaires, alors nous formons toujours des systèmes compacts de super-Terres, mais nous formons pas ou peu de cœurs de planètes géantes. Nous observons donc que la métallicité d'une étoile n'a pas ou peu d'influence sur la probabilité de formation d'une super-Terres, tandis qu'elle influe grandement la formation de planètes géantes, en accord avec les observations. 

Le principal défaut de ce modèle de formation est pour l'instant son incapacité à former des noyaux de planètes géantes de manière isolée, c'est à dire sans la présence d'un système compact de super-Terres chaudes dans les parties internes du disque. 

\section{Limitations des modèles}
\subsection{Modèle d'opacité}\index{opacité}
L'opacité joue un rôle prépondérant dans la migration planétaire. Cette dernière a une influence majeure sur le profil de température et sur le couple de Corotation. 

En fonction du modèle d'opacité choisi, la migration d'une planète dans un disque donné peut être très différente \refsec{sec:influence_opacity_table}. 

De plus, le cadre dans lequel l'opacité est calculé est souvent négligé. Afin de calculer l'opacité, il est nécessaire de connaître les propriétés des poussières, tant leur distribution que leur composition. L'abondance de métaux est susceptible de varier d'une étoile à l'autre et avant d'avoir un effet sur la quantité de masse disponible pour la formation planétaire, ce paramètre aura un impact direct sur l'opacité dans le disque. 

Ensuite, les propriétés de la poussière sont susceptibles de varier au cours de la dissipation du disque et de la formation des planètes, même si les collisions peuvent régénérer la population de poussière.

L'abondance de planètes semble dépendre de la métallicité des étoiles autour desquelles elles orbitent \citep{fischer2005planet}. Une étude sur le sujet devrait aussi s'intéresser à l'influence de la métallicité sur l'opacité dans le disque, compte tenu du fait que ça aura une incidence sur la carte de migration qui en découle.

\subsection{Longueur de lissage}\index{longueur de lissage}
La migration de Type I que nous utilisons est issue des formules de \cite{paardekooper2011torque}. Même si ces formules sont utilisées ici dans un disque 1D modélisé où le profil de température, densité de surface, rapport d'aspect change en fonction de la distance, la formule originelle est calculée pour un disque où le rapport d'aspect est constant et indépendant de la température et où les profils de densité de surface et de température ont un indice constant. 

Ces formules analytiques ont été calculées puis ajustées à des simulations hydrodynamiques 2D. Une comparaison indépendante des formules avec d'autres simulations hydrodynamiques 2D trouve un bon accord avec ces formules, excepté le fait que la migration est plus lente dans les simulations hydrodynamiques 2D \citep{pierens2013makingaccepted}. 

Le biais le plus important introduit par cette formule est qu'elle dépend fortement de la longueur de lissage $b/h$. Le problème est qu'il n'existe pas de valeur optimale de la longueur de lissage à la fois pour les couples de Lindblad et de corotation \citep{masset2002coorbital}. En effet, le couple de corotation provient de régions au plus proche de la planète, qui sont totalement modifiées par la taille de la zone lissée autour de la planète. La longueur de lissage tend à être plus petite si on veut modéliser correctement le couple de corotation, alors qu'elle doit être légèrement supérieure si on veut modéliser correctement le couple de Lindblad. 

Il semble impossible de modéliser une migration doublement lissée dans un code hydrodynamique qui modélise le disque et la migration avec précision. Une solution approchée serait de modéliser artificiellement deux longueurs de lissage dans les codes N-corps, une pour le couple de Corotation et une pour le couple de Lindblad. Il faut toutefois vérifier que le résultat permet de contourner une limitation numérique en n'introduisant pas de biais ou d'erreur plus importante que le gain que l'on cherche à obtenir. Une étude des simulations 3D, où l'introduction d'une longueur de lissage n'est pas nécessaire, et des simulations N-corps semble donc nécessaire pour s'en assurer. 

\subsection{Modélisation de la densité, température et viscosité dans un disque}
La formule du couple de migration dépend des paramètres du disque et de leur évolution au sein de ce dernier. 

Notre modèle consiste en un disque avec un profil de densité de surface en loi de puissance fixe, puis nous calculons le profil de température $T$, de rapport d'aspect $h=H/R$, de diffusivité thermique $\chi$, d'opacité $\kappa$ et de profondeur optique $\tau$ en fonction du rayon. Nous avons donc un disque 1D. Nous n'avons pas de dépendance explicite en fonction de la hauteur du disque. 

La plus grand incertitude de notre modèle est l'approximation qui consiste à dire que depuis le bord interne jusqu'au bord externe, nous avons une évolution en loi de puissance de la densité de surface $\Sigma\propto R^{-d}$ dont l'indice $d$ est fixe. Quand $d>1$, pour une masse de disque constante, la masse se trouve préférentiellement au bord interne résultant en des températures $T>100 000\unit{K}$ dans certaines conditions. Dans de tels disques, la modélisation atteint ses limites.

D'autres modèles fixent au contraire le profil de température pour calculer le profil de densité de surface. Enfin, dans les simulations hydrodynamiques, une viscosité $\nu$ constante est souvent choisie afin de simplifier la modélisation. Aucune de ces approximations n'est réaliste pour des distances allant de $0.1$ à $100\unit{UA}$. 
%TODO ref !!!!!

Une étude intéressante serait de voir si un modèle simplifié 1D ne pourrait pas modéliser de manière cohérente le profil de densité de surface, de température et de viscosité. La migration est maintenant disponible pour les modèles N-corps à l'aide de formules analytiques. Peut-être que la modélisation des rétroactions entre densité de surface, viscosité et température peut être effectuée à l'aide d'un disque simplifié, sans dépendance ni azimutale ni verticale. En effet, à l'heure actuelle, quel que soit le type de simulation considéré, l'incertitude la plus élevée réside toujours dans la valeur que l'on attribue à l'un des trois profils que l'on prend pour paramètre libre, que ce soit la température, la densité de surface ou la viscosité.

\subsection{Accrétion de gaz et dissipation du disque}\index{accrétion de gaz}\index{dissipation du disque}
Lors de la dissipation du disque, en particulier dans la deuxième phase où la photo-évaporation joue un rôle majeur, le bord interne va être fortement modifié. En conséquence, la stabilité des systèmes compacts de super-Terres sera affectée. La manière dont le bord interne sera modifié avec le temps ainsi que le profil de densité de surface dans son voisinage sont des paramètres clés dont les effets restent à quantifier. 

De plus, dans notre modèle nous n'avons pas pris en compte l'accrétion de gaz. Or une partie au moins des planètes que nous
formons ont des masses à partir desquelles l'accrétion de gaz commence. L'accrétion de gaz, en accélération la croissance en
masse des embryons pourrait augmenter la probabilité de former des cœurs de planète géante dans les parties externes du
disque, ces derniers pourraient en effet accéder à la zone de migration vers l'extérieur plus facilement. Dans le même temps, la
proximité des super-Terres chaudes avec leur étoile doit aussi avoir une influence sur leur atmosphère. 

Une étude ultérieure est nécessaire afin de comprendre les effets de la dissipation du disque et de l'accrétion de gaz sur la formation des super-Terres qui sont dans des conditions (masse et distance) où ces deux effets peuvent êtres importants.


\section{Perspectives}
\subsection{Formation de super-Terres ou planète géantes}\index{super-Terre}\index{noyau de planète géante}
Nous avons vu précédemment qu'il était possible de former des systèmes compacts de super-Terres chaudes et des noyaux de planètes géantes. Il reste à étudier la stabilité de ces systèmes durant la dissipation du disque. Il reste aussi à voir si les noyaux de planètes géantes que nous formons peuvent effectivement donner des planètes géantes si on inclut dans le modèle l'accrétion de gaz et la migration de Type II. 

Enfin, une étude importante doit être poursuivie afin de voir s'il est possible de former des noyaux de planètes géantes sans former de systèmes compacts au bord interne. Il existe deux manières principales pour y parvenir. C'est possible si la quasi totalité de la masse initiale contenue dans les embryons est nécessaire pour former le ou les noyaux de planètes géantes. Ainsi il ne reste pas suffisamment de masse pour former un système compact au bord interne. L'autre manière d'y parvenir, c'est que le système compact ne survive pas au bord interne. Un phénomène intéressant serait que la présence d'une planète géante dans les parties externes d'un disque rende un système compact interne instable durant la dissipation du disque. 

\subsection{Self-shadowing}\index{self-shadowing}
Nous avons vu \refsec{sec:shadow} que l'ombre du disque, modélisée de façon cohérente, n'était susceptible d'avoir une influence sur la carte de migration que dans une région où l'irradiation domine le profil de température. Or, typiquement, la zone d'ombre d'un disque se situe justement dans les parties internes du disque, là où le chauffage visqueux domine. 

Cependant, la zone morte modifie le disque de telle sorte qu'une légère surdensité se produit au bord interne de cette dernière. De plus, dû à la soudaine chute de la viscosité, le chauffage visqueux est beaucoup moins important et la zone morte est une région où l'irradiation peut dominer, bien que nous soyons dans les parties internes du disque. 

La zone morte pourrait donc être une région privilégiée du disque où l'irradiation et le \og self-shadowing\fg ont une grande importance. C'est de plus une zone d'intérêt pour la formation planétaire par le piège à planète qu'elle peut constituer \citep{hasegawa2011origin}.

Il serait intéressant d'étudier l'effet conjoint de l'ombre du disque et de la zone morte sur la carte de migration et les conséquences que cela a sur la migration et formation planétaire. En effet, il semble que les deux effets résultent en la création d'une zone de convergence pour les masses très faibles \reffig{fig:dz_shadow_map} qui pourrait constituer un piège à planète de faible masse au cours de la vie du disque. Ce piège à planète se situe au niveau du bord interne de la zone morte, c'est à dire de l'ordre de $1\unit{UA}$.

\subsection{Effet indirect des ondes de densité sur les autres planètes}
Le couple de migration est calculé dans des simulations où la planète est isolée. Mais comment réagi une planète aux ondes de densité d'une autre planète? 

A priori, une planète n'est pas sensible à l'onde de densité de Lindblad d'une autre planète car elle n'est pas résonante avec la planète. En effet, ce qui génère le couple de Lindblad est d'une part la présence de l'onde, mais aussi et surtout le fait que l'onde tourne dans le disque à la fréquence épicyclique de la planète considérée. Dans le cas de deux planètes possédant deux périodes orbitales différentes, l'onde de la deuxième planète n'est donc pas en interaction résonante avec la première. 

La question reste cependant posée dans le cas où les planètes sont en résonances. Dans ce cas il existe un rapport entier entre leurs deux périodes orbitales et ce même rapport entier existe entre les vitesses angulaires. Le cas particulier des résonances co-orbitales est intéressant car les planètes ont le même demi-grand axe et la même vitesse angulaire en première approximation. \cite{podlewska2012outward, baruteau2013disk} ont étudié cet aspect pour des planètes perturbées par l'onde de densité d'une planète massive. Dans un tel système, la réponse du disque n'est pas linéaire pour la deuxième planète. Aucune étude n'a été faite à ce jour sur cet effet dans le cas de deux planètes de masses comparables. 

\section{Conclusion}
J'ai cherché tout au long de ma thèse à développer un code numérique simple et modulaire avec une idée en tête, pouvoir tester les interactions et/ou différences entre différents modèles. L'idée était de profiter des libertés offertes par un code N-corps plus rapide pour tester des domaines de l'espace des paramètres qui ne sont pas accessibles aux codes hydrodynamiques. 

\bigskip

Dans le chapitre \refsec{sec:chap3}, j'ai étudié la migration dans les disques à l'aide de cartes de migration qui permettent de comprendre le comportement de la migration dans un disque donnée de manière assez intuitive. Dans un premier temps j'ai cherché à comprendre la forme de ces cartes de migration. Puis, j'ai cherché à comprendre l'influence des paramètres du disque sur la migration, toujours à l'aide des cartes de migration. Le code N-corps que j'ai développé m'a donné une grande liberté quant à l'espace des paramètres accessible. 

Il m'a de plus permis de tester la migration dans des conditions totalement inaccessibles aux simulations hydrodynamiques, typiquement une centaine d'embryons planétaires évoluant pendant plusieurs millions d'années. Ces simulations m'ont permis d'observer des phénomènes liés à l'interaction entre plusieurs effets isolés. L'interaction entre résonance et amortissement du couple de corotation en est un exemple. La conséquence directe est la possibilité pour un système de stopper sa migration autour d'une zone d'équilibre qui ne correspond à aucune zone de couple nul dans le disque. 

\bigskip

La grande diversité des disques et migrations montre que les populations synthétiques de planètes ne peuvent reproduire la population de planètes extrasolaires avec un seul type de disque. Les propriétés de l'étoile, la masse du disque, la quantité de poussière, la dissipation du disque et l'évolution subséquente de la densité de surface, l'opacité et la température doivent être prises en compte. 

Cela souligne que nous manquons cruellement de contraintes observationnelles. En particulier sur la densité de surface. Le profil de la nébuleuse solaire minimale en $R^{-\frac{3}{2}}$ est largement utilisé, mais ne correspond pas aux observations qui trouvent un profil moyen en $R^{-1}$. De plus, il est peu probable qu'une seule loi de puissance décrive correctement la totalité d'un disque. 

ALMA devrait amener des observations de très haute résolution de disques protoplanétaires, et un gain de précision sur les propriétés des disques. Ces contraintes devraient nous permettre de limiter l'espace des paramètres et de trouver quelques disques dits \og classiques\fg sur lesquels la formation planétaire pourra se concentrer. 

Il n'est pas impossible que dans le futur, certaines sous-populations particulières d'exoplanètes puissent être expliquées par un type particulier de disque. En effet, si le nombre de planètes est faible par rapport au nombre total, il est possible que ces planètes nécessitent des conditions particulières pour être formées. Dans certaines de mes simulations, j'ai pu observer la formation conjointe d'un cœur de planète géante dans les parties externes, autour de $10\unit{UA}$ et d'un système compact au bord interne ($a\sim 0.1\unit{UA}$). Aucun système extra-solaire de ce type n'a encore été trouvé. Peut-être que dans quelques années de tels systèmes seront détectés. Ce serait en tout cas un argument fort en faveur du modèle que j'ai présenté ici. 

\bigskip

Avec des observations toujours plus nombreuses, que ce soient des disques protoplanétaires ou des exoplanètes, nous avons de plus en plus de données à confronter à nos modèles de formation et d'évolution des systèmes planétaires. 

Les prochaines années seront le berceau de la révolution ALMA, en particulier dans le domaine des disques protoplanétaires. Si nous avons de plus en plus de contraintes sur l'état final des systèmes planétaires, nous avons encore beaucoup de libertés sur les conditions initiales pour les former. 

Le plus grand défi de la formation planétaire est de faire co-évoluer le disque et les planètes. En effet, ils interagissent au cours de leur vie, mais il est souvent difficile de tenir compte des deux évolutions en même temps. Pour ce faire, un seul type de simulation ne suffit pas. Afin d'étudier le problème dans toute sa complexité, il est nécessaire de calibrer, puis modéliser, depuis les plus petites échelles jusqu'aux plus larges. Les simulations N-corps s'inscrivent en bout de cette chaîne, utilisant les modèles mis au point dans des simulations complexes mais coûteuses en temps, afin d'accéder à l'immense espace des paramètres qui s'offre à nous.
