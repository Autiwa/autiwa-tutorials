\section{Les Résonances de Moyen Mouvement (MMR)}
\subsection{Définition}\index{résonance!de Moyen Mouvement}\index{Mean Motion Resonance|see{résonance}}
Les \gras[resonance@résonance]{résonances de moyen mouvement} sont des configurations orbitales particulières de deux planètes dans lesquelles il existe un lien entre les périodes orbitales des planètes. Exemple, si deux planètes sont en résonance $3:2$, ça signifie que la planète interne effectuera 3 orbites pendant que la planète externe en effectuera 2.

Ces configurations particulières confèrent une stabilité accrue aux planètes. Plus la résonance est forte et plus il sera difficile pour les planètes d'en sortir.

\bigskip

On met généralement une résonance sous la forme $(p+q):p$ où $p$ et $q$ sont des entiers. Cette forme permet de mettre en évidence un des paramètres qui permet de rendre compte de la force de la résonance. En effet, plus $q$ est petit et plus la résonance est force. Ainsi, les résonances avec $q=1$ sont les plus fortes. On dit que $q$ est l'ordre de la résonance (plus l'ordre est petit et plus la résonance est forte).

\begin{attention}
Mais ce n'est pas le seul paramètre à prendre en compte pour évaluer la force d'une résonance et je suis bien incapable de tous les décrire.
\end{attention}

Pour une résonance $(p+q):p$ on définit un certain nombre d'angles $\theta_i$ dits \gras[angle de résonance]{angles de résonance} de la forme :
\begin{align}
\theta_{i+1} &=(p+q)\lambda_2 -p\lambda_1 - \left[i\varpi_{1} + (q-i)\varpi_2\right]
\end{align}
avec $i$ allant de $0$ à $q$ ; où $\lambda$ sont les longitudes moyennes, $\varpi$ les longitudes du péricentre et les indices $1$ et $2$ se réfèrent respectivement à la planète interne et externe. Pour une résonance $(p+q):p$ on a donc $q+1$ angles de résonance.

Les angles de résonances mesurent l'angle entre les deux planètes au point de conjonction. Si un seul de ces angles est en libration (oscillation autour d'une valeur moyenne) au lieu de circuler librement de $0$ à $2\pi$ alors on dit que les planètes sont en résonances. Le nombre d'angles en libration permettra aussi d'avoir une idée de la force de la résonance.

\begin{exemple}
Soit une résonance $7:2$, les angles de résonances sont :
\begin{align*}
\theta_1 &= 7 \lambda_2 -2\lambda_1 - 5 \varpi_1\\
\theta_2 &= 7 \lambda_2 -2\lambda_1 - \left( 4 \varpi_1 + 1\varpi_2 \right)\\
\theta_3 &= 7 \lambda_2 -2\lambda_1 - \left( 3 \varpi_1 + 2\varpi_2 \right)\\
\theta_4 &= 7 \lambda_2 -2\lambda_1 - \left( 2 \varpi_1 + 3\varpi_2 \right)\\
\theta_5 &= 7 \lambda_2 -2\lambda_1 - \left( 1 \varpi_1 + 4\varpi_2 \right)\\
\theta_6 &= 7 \lambda_2 -2\lambda_1 - 5 \varpi_2
\end{align*}
\end{exemple}

\begin{remarque}
Les \gras[kirkwood@Kirkwood!lacunes de]{lacunes de Kirkwood} font elles aussi intervenir des résonnances mais contrairement à ce qu'on pourrait penser, ces résonances avec Jupiter sont des zones déplétées en astéroïdes. La raison profonde n'est pas parfaitement connue mais il semblerait que ce soit dû au chaos. Je ne saurais pas expliquer exactement ce que ça veut dire par contre.

La résonance imposte une valeur de $a$, mais des échanges sont possibles entre les deux corps en résonance (je ne sais pas bien de quelles valeurs par contre), et il est possible que par ce biais l'eccentricité puisse augmenter, et ainsi dépléter la lacune de kirkwood en favorisant les collisions entre les objets en résonance et les autres qui sont dans la ceinture.
\end{remarque}
%TODO MMR : lire a thorough analysis of the dunamics involved the reader should consult (Murray & Dermott, 1999)
%TODO 
\subsection{Résonances et excentricité}
%TODO 
\subsection{Stabilité et ordre des résonances}
%TODO 

\section{Les Zones de Convergence}
%TODO 
\subsection{Existence et intérêt}
%TODO 
\subsection{Les différents types}
%TODO 
\subsection{Diagrammes de couple a-m}
%TODO parler des raisons pour lesquelles la zone de convergence dépend de la masse et de la distance parfois, avec les comparaisons des temps (dynamique, de U-turn and de diffusion)
\subsection{Résonances et Accrétions}
%TODO 