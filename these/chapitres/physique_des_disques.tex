\chapter{Évolution visqueuse du disque : démonstration}\label{app:equation_angular_momentum}
Durant ma thèse, j'ai été amené à redémontrer l'équation régissant l'évolution visqueuse du disque, à la fois spatialement et temporellement. Lors de cette démonstration, j'ai eu besoin de bien plus d'étapes détaillées de calcul que ce que j'ai pu trouver dans les papiers. J'ai donc entrepris de le refaire en détail, et je regroupe ici les étapes de calcul et les astuces nécessaires pour arriver à l'équation finale, disponible notamment dans \cite{pringle1981accretion}.

\bigskip

On cherche à faire le bilan de moment cinétique sur l'anneau décrit \reffig{fig:disk_ring}. Son moment cinétique est défini par :
\begin{align}
\vect{J_a} &= \vect{R} \wedge (m_a\vect{v(R)}) \nonumber\\
&= m_a \cdot R \cdot R\Omega(R)\hat{e}_z\nonumber\\
&= 2\pi R \Delta R \Sigma(R)\cdot R \cdot R\Omega(R)\hat{e}_z\nonumber\\
\vect{J_a} &= 2\pi R^3 \Delta R \Sigma(R)\Omega(R)\hat{e}_z\label{eq:J_a2}
\end{align}
où $\Sigma$ et $\Omega$ sont la densité de surface et la vitesse angulaire du gaz à la position $R$ dans le disque.

Le flux de moment cinétique est simplement défini comme la quantité de moment cinétique emportée ou apportée par le flux de masse défini précédemment \refeq{eq:dif_F_M} :
\begin{subequations}
\begin{align}
\dif J(R) &= \vect{r} \wedge \left(\dif F_M(R) \vect{v}(R)\right)\nonumber\\
 &= \dif F_M(R) \cdot R^2\Omega(R)\hat{e}_z\nonumber\\
\dif J(R) &= 2\pi v_r(R) \Sigma(R)\cdot R^3\Omega(R)\hat{e}_z\label{eq:dJ_in2}\\
\dif J(R+\Delta R) &= \vect{r} \wedge \left(\dif M(R+\Delta R) \vect{v}(R+\Delta R)\right)\nonumber\\
 &= \dif F_M(R+\Delta R) \cdot \left(R+\Delta R\right)^2\Omega(R+\Delta R)\hat{e}_z\nonumber\\
\dif J(R+\Delta R) &= -2\pi v_r(R+\Delta R) \Sigma(R+\Delta R)\cdot \left(R+\Delta R\right)^3\Omega(R+\Delta R)\hat{e}_z\label{eq:dJ_out2}
\end{align}\label{eq:dJ2}
\end{subequations}

\bigskip

À ceci s'ajoute la variation de moment cinétique induite par la friction entre anneaux concentriques, en d'autres termes, dus à la viscosité du disque. Cette variation de moment cinétique est représentée sous la forme d'un couple exercé par les anneaux internes et externes à celui considéré. 

Le taux de cisaillement $A$ est donné par : 
\begin{align}
A &= r \dod{\Omega}{r}
\end{align}
et représente les frottements induits par la rotation différentielle.

La force visqueuse par unité de longueur est définie par :
\begin{align}
\dif F_\text{vis} &= \nu \Sigma A = \nu \Sigma r \dod{\Omega}{r}
\end{align}

La force visqueuse induite par les anneaux entourant l'anneau considéré est alors : 
\begin{subequations}
\begin{align}
\vect{F_\text{in}}(R) &= 2\pi R \times \dif F_\text{vis}(R) \nonumber\\
\vect{F_\text{in}}(R)&= 2\pi\nu \Sigma R^2 \dod{\Omega}{r}(R) \hat{e}_\theta\\
\vect{F_\text{out}}(R+\Delta R) &= 2\pi (R+\Delta R) \times \dif F_\text{vis}(R+\Delta R) \nonumber\\
\vect{F_\text{out}}(R+\Delta R)&= 2\pi\nu \Sigma (R+\Delta R)^2 \dod{\Omega}{r}(R+\Delta R) \cdot \hat{e}_\theta
\end{align}
\end{subequations}
L'anneau interne tournant plus vite, la force est dirigée dans le sens de rotation $\hat{e}_\theta$. À l'inverse, l'anneau externe tourne moins vite, il tend à freiner l'anneau de référence et s'oppose à son mouvement. La force est donc opposée au sens de rotation.

\bigskip

Ainsi, le couple $\vect{\Gamma}=\vect{r}\wedge\vect{F}$ issu de chacun des anneaux entourant celui de référence s'écrit :
\begin{subequations}
\begin{align}
\vect{\Gamma_\text{in}} &= R\hat{e}_r\wedge\vect{F_\text{in}}\nonumber\\
\vect{\Gamma_\text{in}} &= 2\pi\nu \Sigma R^3 \dod{\Omega}{r}(R) \hat{e}_z\label{eq:G_in2}\\
\vect{\Gamma_\text{out}} &= (R+\Delta R)\hat{e}_r\wedge\vect{F_\text{out}}\nonumber\\
\vect{\Gamma_\text{out}} &= 2\pi\nu \Sigma (R+\Delta R)^3 \dod{\Omega}{r}(R+\Delta R) \hat{e}_z\label{eq:G_out2}
\end{align}\label{eq:J_torques2}
\end{subequations}

\bigskip

On fait maintenant un bilan des variations de moment cinétique pour l'anneau de gaz. Pour cela on dit que la variation de moment cinétique (que l'on écrit en dérivant $J_a(t)$) est égale aux variations de moment cinétiques induites aux bords de l'anneau par échange de masse à laquelle s'ajoute la différence entre les deux couples visqueux qui s'appliquent au bord externe et interne. Ce qui donne : 
\begin{align}
\dod{J_a}{t} &= \dif J(R+\Delta R) + \dif J(R) + \Gamma_\text{out} - \Gamma_\text{in}\label{eq:cons_J_a2}
\end{align}

En utilisant \refeq{eq:J_a2}, \refeq{eq:dJ2}, \refeq{eq:J_torques2}, dans \refeq{eq:cons_J_a2}
\begin{align*}
\begin{split}
\dpd{}{t}\left(2\pi R^3 \Delta R \Sigma(R)\Omega(R)\right) &= -\left(R+\Delta R\right)^3v_R(R+\Delta R) \Sigma(R+\Delta R) \Omega(R+\Delta R)\\
& + R^3 v_R(R) \Sigma(R) \Omega(R) + \left[\nu(R+\Delta R)^3\Sigma(R+\Delta R)\right.\\
&\left. \dod{\Omega}{r}(R+\Delta R)-\nu \Sigma(R) R^3 \dod{\Omega}{r}(R)\right]
\end{split}
\end{align*}



On fait tendre $\Delta R$ vers 0, et de manière similaire au bilan de masse obtenu précédemment, il vient alors 
\begin{align*}
\dpd{}{t}\left(R^3 \Sigma\Omega\right) &= -\dpd{}{r}\left(R^3 v_R \Sigma \Omega\right) + \dpd{}{r}\left(\nu \Sigma R^3 \dod{\Omega}{r}\right)\\
\intertext{$R$ et $t$ sont des variables indépendantes, on peut donc sortir $R$ de la dérivée partielle temporelle afin de faire apparaître une forme qui fait penser à une équation de continuité.}
R\dpd{}{t}\left(R^2 \Sigma\Omega\right) &= -\dpd{}{r}\left(R^3 v_R \Sigma \Omega\right) + \dpd{}{r}\left(\nu \Sigma R^3 \dod{\Omega}{r}\right)
\end{align*}

$R$ et $t$ étant des variables indépendantes, on peut écrire :
\begin{align}
R\dpd{}{t}\left(\Sigma R^2\Omega\right) + \dpd{}{r}\left(R^3 v_R \Sigma \Omega\right) &= \dpd{}{r}\left(\nu \Sigma R^3 \dod{\Omega}{r}\right)\label{eq:ang_mom_01}
\end{align}

\bigskip

On suppose que $\dpd{\Omega}{t}=0$ vu que le potentiel gravitationnel est indépendant du temps (on ne considère pas une masse variable de l'étoile due à l'accrétion), et sachant que $R$ ne dépend pas explicitement de $t$, en utilisant la formule : 
\begin{align*}
\dpd{uv}{x} &= \dpd{u}{x}v + u\dpd{v}{x}
\end{align*}
on peut écrire :
\begin{align}
\dpd{}{t}\left(\Sigma\cdot R^2\Omega\right) &= \left(R^2\Omega\right)\dpd{\Sigma}{t} + \Sigma\cancelto{0}{\dpd{R^2\Omega}{t}}\label{eq:ang_mom_tmp_01}
\end{align}

De même : 
\begin{align}
\dpd{}{r}\left(R^3 v_R \Sigma \Omega\right) &= \dpd{}{r}\left(R v_R \Sigma \cdot R^2\Omega\right)\nonumber\\
&= \left(R^2\Omega\right)\dpd{}{r}\left(R v_R \Sigma\right) + R \Sigma v_R \dpd{}{r}\left(R^2\Omega\right)\label{eq:ang_mom_tmp_02}
\end{align}

En utilisant \refeq{eq:ang_mom_tmp_01} et \refeq{eq:ang_mom_tmp_02} dans \refeq{eq:ang_mom_01}, on fait alors apparaître \refeq{eq:conservation_masse}, ce qui donne : 
\begin{align}
R\left(R^2\Omega\right)\dpd{\Sigma}{t} + \left(R^2\Omega\right)\dpd{}{r}\left(R v_R \Sigma\right) + R \Sigma v_R \dpd{}{r}\left(R^2\Omega\right) &= \dpd{}{r}\left(\nu \Sigma R^3 \dod{\Omega}{r}\right)\nonumber\\
\left(R^2\Omega\right)\cancelto{0}{\left[R\dpd{\Sigma}{t} + \dpd{}{r}\left(R v_R \Sigma\right)\right]} + R \Sigma v_R \dpd{}{r}\left(R^2\Omega\right) &= \dpd{}{r}\left(\nu \Sigma R^3 \dod{\Omega}{r}\right)\nonumber\\
R \Sigma v_R \dpd{}{r}\left(R^2\Omega\right) &= \dpd{}{r}\left(\nu \Sigma R^3 \dod{\Omega}{r}\right)\nonumber\\
R \Sigma v_R &= \inv{\dpd{}{r}\left(r^2\Omega\right)} \dpd{}{r}\left(\nu \Sigma R^3 \dod{\Omega}{r}\right)\label{eq:r_sigma_vr}
\end{align}

\bigskip

On injecte alors \refeq{eq:r_sigma_vr} dans \refeq{eq:conservation_masse} afin de supprimer $v_r$ de l'expression et obtenir finalement : 
\begin{align*}
\dpd{\Sigma}{t} &= -\inv{r}\dpd{}{r}\left[\inv{\dpd{}{r}\left(R^2\Omega\right)} \dpd{}{r}\left(\nu \Sigma r^3 \dod{\Omega}{r}\right)\right]
\end{align*}

On décale le signe moins au niveau de la dérivée de la vitesse angulaire, cette dernière étant généralement négative, ça permet d'avoir un terme positif :
\begin{important}
\begin{align}
\dpd{\Sigma}{t} &= \inv{r}\dpd{}{r}\left\{\inv{\dpd{}{r}\left(r^2\Omega\right)} \dpd{}{r}\left[\nu \Sigma r^3 \left(-\dod{\Omega}{r}\right)\right]\right\}
\end{align}
\end{important}

\bigskip

On fait maintenant l'approximation que le mouvement est képlerien, avec pour première conséquence que $\Omega = \sqrt{\frac{GM}{r^3}}$. On peut alors simplifier l'équation : 
\begin{align*}
\dpd{\Sigma}{t} &= \inv{r}\dpd{}{r}\left\{\inv{\inv{2}\sqrt{\frac{GM}{r}}} \dpd{}{r}\left[\nu \Sigma r^3 \left(\frac{3}{2}\sqrt{\frac{GM}{r^5}}\right)\right]\right\}\nonumber\\
&= \inv{R}\dpd{}{r}\left\{\bcancel{2}\sqrt{\frac{r}{\cancel{GM}}} \dpd{}{r}\left[\nu \Sigma \frac{3}{\bcancel{2}}\cancel{\sqrt{GM}}r^\sfrac{1}{2}\right]\right\}\nonumber\\
\end{align*}

On obtient alors l'équation suivante : 
\begin{important}
\begin{align}
\dpd{\Sigma}{t} &=\frac{3}{r}\dpd{}{r}\left[\sqrt{r} \dpd{}{r}\left(\nu \Sigma r^\sfrac{1}{2}\right)\right]
\end{align}
\end{important}
