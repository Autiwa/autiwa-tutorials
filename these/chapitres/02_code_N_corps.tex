Afin d'étudier la formation planétaire et les interactions avec le disque de gaz, j'ai utilisé un code de simulation N-corps, qui permet de regarder l'évolution d'un nombre arbitraire de corps orbitant autour d'un astre central. 

Ce choix est apparu naturellement. Au début de ma thèse j'ai fait quelques simulations hydrodynamiques avec le code Genesis développé par Arnaud Pierens. J'ai rapidement constaté que ce genre de simulations, bien que modélisant de manière poussée le disque, ne permettait pas d'étudier de manière approfondie la dynamique planétaire. Le temps de calcul nécessaire pour une simulation limite en effet grandement le nombre de corps ainsi que la durée d'intégration. J'ai donc souhaité me tourner vers un code N-corps, afin de privilégier la dynamique planétaire, et de modifier ce programme afin d'y inclure les effets d'un disque de gas sur la dynamique planétaire. 

J'ai ainsi gagné en temps de calcul, et j'ai ouvert un vaste champ d'investigation sur les paramètres du disques, le nombre de corps en interaction, me permettant de faire des systèmes planétaires très divers, parfois avoir plusieurs centaines d'embryons pour plusieurs millions d'années, chose impossible dans les simulations hydrodynamiques du début de ma thèse où 20 corps pendant quelques dizaines de milliers d'années était un maximum. 

Ce choix a bien entendu introduit son lot d'incertitudes et d'approximations qui sont discutés dans la partie \refsec{sec:discussion}. La présente section a pour but de présenter le code N-corps que j'ai utilisé ainsi que les différents effets du disque que j'ai modélisé. J'ai avant tout souhaité présenter les parties qui ont des conséquences sur la physique du disque, que ce soit en terme de choix d'un modèle particulier, ou de limitations numériques qu'il est bien de garder à l'esprit quand on interprète les résultats.

\section{Présentation de mercury}
Le code N-corps choisi est le code \textbf{mercury} \citep{chambers1999hybrid}. Ce code offre la possibilité de choisir un algorithme parmi 5 différents (BS, BS2, RADAU, MVS et HYBRID), ayant des propriétés diverses. Dans le cadre de ma thèse, je n'ai utilisé que l'algorithme HYBRID, qui utilise l'algorithme MVS la plupart du temps, mais change pour l'algorithme BS2 lors de rencontres proches. Il est possible de déterminer à quel moment on considère qu'une rencontre est "proche" dans le fichier de paramètre de programme, j'ai laissé le paramètre par défaut. 

La raison de ce changement est assez simple. MVS est un algorithme symplectique, c'est à dire à pas de temps constant, dans lequel on défini un hamiltonien que l'on résout pour faire évoluer les orbites. La conservation de l'énergie est moins bonne que pour un algorithme à pas de temps adaptatif, mais le point très important est que cette conservation de l'énergie est bien meilleure au cours du temps. C'est à dire que là où les algorithmes tels que BS, BS2 et RADAU verront leur erreur sur l'énergie augmenter au cours du temps, les algorithmes symplectiques vont eux voir leur erreur rester plus ou moins constante au cours du temps. 

Dans le cadre de mes simulations, j'ai accordé une importance limitée aux variations d'énergie, étant donné que les couples que l'on rajoute pour simuler la présence du disque de gaz font que l'énergie n'est pas conservée pour une planète donnée. Cependant, il est important de bien résoudre les orbites et c'est ce point qui est le plus crucial ici. En effet, quelques tests ont permis de contraindre le pas de temps minimal qu'il est nécessaire d'avoir en fonction de la distance orbitale d'une planète. La contrainte de pas de temps dans mes simulations vient donc d'une distance minimale en dessous de laquelle les orbites ne sont pas correctement calculées. Cette limite, afin d'éviter tout problème, est choisie pour être en dessous du bords interne du disque de gaz que je défini.

%TODO Simulation lancées dans le dossier $sse/test_mercury, sur une même simulation mais avec différents algorithmes pour voir l'évolution de la conservation de l'énergie au cours du temps

%TODO parler de la précision de conservation de l'énergie
%TODO parler du pas de temps qui a une influence sur les orbites, et des tests effectués pour contraindre le pas de temps par rapport à l'orbite minimale accessible.

\section{Disque 1D}
Afin de calculer les effets d'un disque de gaz, une modélisation de ce dernier est nécessaire. Le but étant d'avoir une grande souplesse, le disque implémenté est bien entendu très simplifié. Toutes les quantités sont intégrées et invariantes selon la hauteur z et la position azimutale $\theta$ dans le disque, résultant en un modèle radial de toutes les quantités. 

Dans la mesure du possible, les quantités du disque ont été calculées de manière consistante. Je vais présenter dans la suite de manière chronologique comment sont calculées les grandeurs physiques du disque.
%TODO 
\subsection{Profil de densité de surface}
Le profil de densité de surface est défini au début de la simulation comme une loi de puissance de la forme :
\begin{align}
\Sigma(R) &= \Sigma_0 \times R^{-\alpha}
\end{align}
où $\Sigma_0$ est la densité de surface à $1\unit{AU}$ et $\alpha$ l'indice de la loi de puissance. 

Ce profil de densité de surface est défini pour une certaine étendue radiale. On défini donc un bord interne $R_\text{in}$ et un bord externe $R_\text{out}$. Le bord interne est généralement à $0,1\unit{AU}$ et le bord externe à $100\unit{AU}$. 

Afin de calculer les valeurs suivantes, ce disque est échantillonné et toutes les valeurs nécessaires sont ensuite calculées à chacun de ces points. 

\subsection{Table d'opacité}
%TODO 
\subsection{Profil de température}
%TODO 
\section{Migration type I}
%TODO 
%TODO paardekooper
%TODO parler du papier de bitsch (annexe I) où ils font un correctif pour le facteur 4 qu'il manque dans paardekooper
\section{Amortissement de e et I}
%TODO 
\section{Effet de l'excentricité sur le couple de corotation}
%TODO du couple de corotation quand l'excentricité augmente, mais je sais pas encore où le placer, pas dans la partie intro je suppose