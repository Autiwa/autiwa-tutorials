\chapter{Formulaire}
Ici sont répertoriées bon nombre de formules que j'ai utilisé et qui relient des grandeurs physique entre elles. Dans la mesure du possible, une source est donnée où la formule est mentionnée. Ceci a pour but de centraliser ces formules, liées à la physique des disques, et que j'ai parfois eu du mal à retrouver parmis la quantité d'articles ou de livres traitant du sujet. 

\section{Propriétés du disque}

La prescription alpha pour la viscosité d'un disque est définie par :
\begin{align}
\nu &= \alpha c_s H
\end{align}

\begin{align}
c_s &= \sqrt{\frac{k_B T}{\mu m_H}}
\end{align}

\begin{align}
H &= \inv{\Omega}\sqrt{\frac{k_B T}{\mu m_H}}\\
&= \frac{c_s}{\Omega}
\end{align}

On considère que la densité de surface est égale à la densité volumique, intégrée sur la taille verticale $2H$ du disque. 
\begin{align}
\Sigma &= 2\rho H
\end{align}

\section{Propriétés des orbites képleriennes}
