\chapter{Formulaire}
Ici sont répertoriées bon nombre de formules que j'ai utilisé et qui relient des grandeurs physique entre elles. Dans la mesure du possible, une source est donnée où la formule est mentionnée. Ceci a pour but de centraliser ces formules, liées à la physique des disques, et que j'ai parfois eu du mal à retrouver parmis la quantité d'articles ou de livres traitant du sujet. 

\section{Variables usuelles}\label{sec:variables}
Dans la thèse, j'utilise couramment les mêmes notations pour une propriété physique donnée. Ici je fais un inventaire des notations, afin qu'on puisse s'y référer, et pour gagner en clarté dans le texte en m'évitant de redéfinir à chaque fois les mêmes unités. Les paramètres avec un $p$ en indice indiquent simplement que c'est la valeur du paramètre à la position orbitale de la planète.

\begin{table}[htbp]
\centering
\begin{tabular}{|>{$}c<{$}|p{7cm}|}
\hline
k_B & constante de Boltzmann \\\hline
\sigma & constante de Stefan-Boltzmann\\\hline
G & Constante de gravitation universelle\\\hline
b/h & Longueur de lissage du potentiel gravitationnel de la planète en unité de son rayon de Hill\\\hline
\mu & Masse moléculaire moyenne du gaz constituant principal du disque\\\hline
\unit{M_\odot} & Masse solaire (unité de masse)\\\hline
\unit{M_\oplus} & Masse terrestre\\\hline
\gamma & Indice adiabatique du gaz\\\hline
\end{tabular}
\caption{Liste des constantes et notations associées.}
\end{table}

\begin{table}[htbp]
\centering
\begin{tabular}{|>{$}c<{$}|p{7cm}|}
\hline
\alpha & paramètre adimensionné pour la prescription $\alpha$ du disque, permettant de définir une viscosité fonction de la vitesse du son\\\hline
H & Échelle de hauteur du disque\\\hline
h=H/R & rapport d'aspect du disque\\\hline
\nu & Viscosité cinématique du disque\\\hline
T & Température\\\hline
c_s & vitesse du son\\\hline
\Sigma & Densité de surface du disque de gaz\\\hline
\rho & Densité volumique du disque de gaz\\\hline
\Omega & vitesse angulaire d'une particule fluide ou d'une planète dans le disque\\\hline
\kappa & Fréquence épicyclique\\\hline
\chi & Diffusivité thermique\\\hline
d & Indice négatif du profil de densité de surface\\\hline
\beta & Indice négatif du profil de température\\\hline

\end{tabular}
\caption{Liste des paramètres du disque et notations associées.}
\end{table}

\begin{table}[htbp]
\centering
\begin{tabular}{|>{$}c<{$}|p{7cm}|}
\hline
M_p & Masse de la planète\\\hline
M_\star & Masse de l'étoile centrale\\\hline
q & $M_p/M_\star$\\\hline
\Gamma_\text{tot} & Couple total exercé par le disque sur la planète\\\hline
\Gamma_L & Couple différentiel de Lindblad\\\hline
\Gamma_C & Couple de corotation\\\hline
J & Moment cinétique\\\hline
t_\text{rad} & Temps de diffusion radiatif du disque\\\hline
t_\text{visc} & Temps de diffusion visqueux du disque\\\hline
t_\text{diff} & Temps de diffusion du disque (peut être le temps radiatif ou visqueux, ou les deux, selon le problème considéré)\\\hline
t_\text{lib} & Temps mis par une particule en corotation pour effectuer une orbite de corotation complète\\\hline
t_\text{U-turn} & Temps mis par une particule en corotation pour effectuer un demi tour devant ou derrière la planète\\\hline
\end{tabular}
\caption{Liste des variables et notations associées. Les paramètres avec un $p$ en indice indiquent simplement que c'est la valeur du paramètre à la position orbitale de la planète.}
\end{table}

\section{Propriétés du disque}

La prescription alpha pour la viscosité d'un disque est définie par :
\begin{align}
\nu &= \alpha c_s H
\end{align}

\begin{align}
c_s &= \sqrt{\frac{k_B T}{\mu m_H}}
\end{align}

\begin{align}
H &= \inv{\Omega}\sqrt{\frac{k_B T}{\mu m_H}}\\
&= \frac{c_s}{\Omega}
\end{align}
où $m_H$ est la masse d'un atome d'hydrogène.

La densité de surface $\Sigma$ est calculée en intégrant verticalement la densité volumique, $\rho_0$ étant la densité volumique dans le plan équatorial : 
\begin{align}
\Sigma &= \sqrt{2\pi}\rho_0 H
\end{align}

La diffusivité thermique $\chi$ est définie par : 
\begin{align}
\chi &= \frac{16\gamma (\gamma - 1) \sigma T^4}{3\kappa\rho^2H^2\Omega^2}
\end{align}

\section{Propriétés des orbites képleriennes}
Soit une planète de demi-grand axe $a$, d'excentricité $e$, de masse $m_p$, de période orbitale $T$ orbitant autour d'une étoile de masse $m_\star$. 

Il y a une relation entre sa période orbitale $T$ et son demi-grand axe :
\begin{align}
\frac{T^2}{a^3} &= \frac{4\pi}{G(m_\star + m_p)}
\end{align}
où $G$ est la constante de gravitation universelle.

On défini le périastre $q$ et l'apoastre $Q$ comme étant les distances minimales et maximales entre l'étoile et la planète : 
\begin{subequations}
\begin{align}
q &= a (1 - e)\\
Q &= a (1 + e)
\end{align}
\end{subequations}

La vitesse angulaire moyenne $\Omega$ (ou instantanée en supposant que $e\ll 1$) est définie par : 
\begin{align}
\Omega &= \sqrt{\frac{G(m_\star + m_p)}{a^3}}
\end{align}

La vitesse linéaire moyenne $v$ est définie par : 
\begin{align}
v &= \sqrt{\frac{G(m_\star + m_p)}{a}}
\end{align}

L'énergie $E$ et la norme du moment cinétique $J$ d'une orbite képlerienne de demi-grand axe $a$ et d'excentricité $e$ sont donnés par :
\begin{align}
E &= \inv{2}v^2 - \frac{G(m_\star + m_p)}{r} = \frac{G(m_\star + m_p)}{2a}\\
\norm{\vect{J}} &= G(m_\star + m_p) a (1-e^2)
\end{align}

\bigskip

Une astuce qui peut être utile. Si on souhaite connaître les coordonnées polaires $(r,\theta)$ d'une orbite à partir des coordonnées cartésiennes $(x,y)$, il y a les formules suivantes : 
\begin{subequations}
\begin{align}
r &= \sqrt{x^2 + y^2}\\
\theta &= 2 \arctan\left(\frac{y}{x} + r\right)
\end{align}
\end{subequations}
En particulier, la formule pour $\theta$ permet d'éviter les problèmes de validité des formules dans un certain domaine restreint d'angle. 