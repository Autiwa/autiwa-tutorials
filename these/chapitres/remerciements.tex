Cette section étant sans doute la seule que bon nombre de lecteurs parcourront, je vais donc faire ici la conclusion scientifique de ma thèse, point par point, dans un style aussi soporifique que possible, et en latin, juste par méchanceté gratuite, tout en me délectant du rictus qui doit déformer vos visages à cet instant. 

%Je vous ferai grâce des non remerciements de thèse, parait que ce n'est pas très protocolaire.

%TODO directeur, jury etc

%TODO autres contributeurs, dont franckH

Si le laboratoire était une voûte, Franck Hersant en serait assurément la clé. Je ne déroge donc pas à la règle, même si j'ai le plaisir de remercier Franck non seulement pour son aide numérique, mais aussi et surtout pour son aide pour des questions plus physiques. J'ai particulièrement apprécié sa capacité à faire passer les concepts les plus compliqués avec beaucoup d'humour.

Franck Hersant (FranckH hereafter)

%TODO paragraphe spécial franckS

Franck Selsis (FranckS hereafter)

%TODO les copains, et romuald en particulier, sergi, audrey, émeline

%TODO postérité de quelques clins d'oeil qui ne seront compris que par les personnes concernées, sans éveiller la curiosité des autres. en particulier un truc pour la bretagne, ou tout/tous ou des trucs comme ça.

La thèse, c'est aussi beaucoup de moments inoubliables. Le vol d'un portefeuille (et tous les papiers d'identités) à Madrid, le premier jour d'une semaine de colloque, être perdu à 30km de Munich, moi qui ne sais dire péniblement que "gutten morgen" en allemand. Je voudrais chaleureusement remercier les concepteurs de Fortran pour me procurer si souvent la joie de voir un "Segmentation Fault" en guise de seule indication de bug pour un programme. Merci à Apple pour son merveilleux OS qui a réussi à me convaincre, en 3 ans de thèse qu'en fait, le hardware est malheureusement lui aussi du niveau du software chez eux. 

La thèse, c'est aussi beaucoup d'informatique. Pour le débug et les soucis informatiques quotidiens, merci à FranckH, Romuald (dit le débugueur fou, capable de trouver un bug dans un programme qu'il ne comprend pas), Sergi, ainsi que Pierre Gay et Benoit Hiroux pour leur aide préciseuse lors de la création de mon code. 

%TODO remercier macOS et autre trucs à la con

%TODO famille en dernier


VRAC : 

La science, c'est un peu comme les feux de l'amour. Il y a des associations partout, des pièges. Chacun travaille sur une interaction, il émet des hypothèses. Les autres travaillent avec ses conclusions, et on avance ainsi sur un système bancal qu'on rafistole au fur et à mesure.

Et le Franck d'or est décerné à.... (franckH et franckS ex aequo?)

un morceau du CID remasterisé par mes soins? (à améliorer, c'est juste pour l'idée pour l'instant)


Ô rage, Ô désespoir Ô thèse ennemie
Et ne suis blanchi dans les travaux guerriers
que pour voir en un jour de soutenance tant de lauriers? 
Mes doigts qu'avec respect tout le labo admire, 
mes doigts qui tant de fois on codé du fortran
tant de fois renfrogné sur ma chaise d'ordi
trahit donc ma segmentation fault et ne fait rien pour moi? 
Ô cruel souvenir des exos que je résolvait !
Œuvre de tant de jours en un jour effacée !
Nouvelle page blanche fatale à mon bonheur !
Précipice élevé d'où tombe mes planètes !
Faut-il de votre éclat voir triompher l'astrologie, 
et mourir sans vengeance, en ce 21 décembre? 
fortran77, sois de mon prince à présent gouverneur ; 
ce haut rang n'admet point un scientifique sans bugs
et ton jaloux orgueil par cet affront insigne
malgré le choix du directeur m'en a su rendre indigne. 
Et toi de mes exploits glorieux instrument, 
Mais d'un code sans commentaires, inutile ornement, 
Fer jadis tant à craindre, et qui dans cette offense, 
m'as servi parade et non pas de défense. 
Va, quitte désormais le dernier des binaires,
passe pour me venger en de meilleurs serveurs.