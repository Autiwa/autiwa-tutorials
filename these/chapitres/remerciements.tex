\section{Introduction}
La thèse est un processus long, complexe, et il serait illusoire de faire croire, ou même tenter de faire croire, que je ne dois ce présent manuscrit qu'à moi même. Et comme vous le méritez bien, je prends la liberté de rédiger cette partie comme un article scientifique. 

\section{Le staff}
%Sean et arnaud

%les thésards

\section{Le Franck d'Or}
Durant ma thèse, j'ai eu le loisir de constater que les césards étaient un peu désuets. J'ai donc décidé de créer le Franck d'Or afin de remercier ici tout spécialement Franck Selsis (FranckS hereafter) et Franck Hersant (FranckH hereafter). Il a été impossible de les départager, alors les deux seuls candidats se retrouvent premier \textit{ex-aequo}. Point ici de Franck1 ou franck2, on ne dit plus thésard, on dit doctorant, j'applique ici le même principe aux deux francks. 

Merci à FranckH pour son aide, sa disponibilité, son humour, ses compétences aussi complètes que diverses, parait même qu'il fait de la mécanique\dots En fin de thèse, et pour certaines parties, et avec toute la retenue dont j'ai essayé de faire preuve, j'ai parfois pu abuser de sa gentillesse et de ses compétences. J'ai en tout cas la fierté de lui avoir posé plus de questions de physique que de questions informatique. Merci infiniment pour tout ce que tu as pu faire, pour le débug, les tests, les discussions et tout le reste. Sans ton aide ma thèse et surtout sa rédaction n'auraient absolument pas été les mêmes. 

Merci à FranckS, pour sa disponibilité, sa culture inépuisable sur la photo, la SF, la physique et tout le reste. Sans lui qui m'a fait réaliser que mon attrait pour l'astrophysique pouvait être bien plus réel et tangible que je le croyais, je n'aurais sans doute jamais fait cette thèse. C'est par un stage de licence sous sa direction que j'ai mis un pieds au labo, c'est grâce à lui que j'ai pris goût à la recherche. Pour la bonne humeur qu'il distille au labo, il a contribué largement à rendre ces 6 ans agréables (Licence, M1, M2 + thèse).

D'aucuns prétendent que la fusion des deux francks génèrerait un trou noir, vitesse de la lumière et transport de l'information m'voyez?

\section{Les gens que vous ne connaissez pas}
%famille, amis, dont romuald

\section{Conclusion}
La conclusion de cette étude, menée par (moi et al., 2013) se résume en un mot que je ne pourrais jamais assez appuyer : Merci. 

Merci à vous tous pour ces grands coups de main et ces petits rien qui ne se voient pas sur une page de texte mais qui insidieusement aide à sa réalisation.

\section{Aknowledgements}
Je ne pouvais pas finir cette section sans m'offrir le luxe d'une boucle infinie, une section remerciements dans les remerciements. Je remercie donc ici mes moments de craquage qui, tout au long de ma rédaction m'ont mené à ré-écrire le cid façon Fortran, entre autres choses. 

Merci aussi à Apple et quelques autres qui m'ont aussi fait penser à ce que je ne voudrais jamais faire figurer dans mes remerciements. Même au bout de 3 ans d'effort je ne vous aime toujours pas, mais vous me le rendez bien. 

%%%%%%%%%%%%%

Cette section étant sans doute la seule que bon nombre de lecteurs parcourront, je vais donc faire ici la conclusion scientifique de ma thèse, point par point, dans un style aussi soporifique que possible, et en latin, juste par méchanceté gratuite, tout en me délectant du rictus qui doit déformer vos visages à cet instant. 

%Je vous ferai grâce des non remerciements de thèse, parait que ce n'est pas très protocolaire.

%TODO directeur, jury etc

%TODO autres contributeurs, dont franckH

Si le laboratoire était une voûte, Franck Hersant en serait assurément la clé. Je ne déroge donc pas à la règle, même si j'ai le plaisir de remercier Franck non seulement pour son aide numérique, mais aussi et surtout pour son aide pour des questions plus physiques. J'ai particulièrement apprécié sa capacité à faire passer les concepts les plus compliqués avec beaucoup d'humour.

Franck Hersant (FranckH hereafter)

%TODO paragraphe spécial franckS

Franck Selsis (FranckS hereafter)

%TODO les copains, et romuald en particulier, sergi, audrey, émeline

À vous, mes amis ou copains de passages, confrères de thèse. À toi Romuald plus particulièrement. Je ne vous dois rien (de scientifique), pas une once d'équation ou de code informatique mais je vous dois bien plus que ça. Vous avez contribué à rendre ces 3 ans agréables, parfois même à surmonter les galères (plus ou moins) quotidiennes et pour ça vous avez toute ma gratitude. Même si je dois corriger mon propos quelque peu ; je dois à Romuald quelques lignes de codes et corrections de bug !

%TODO postérité de quelques clins d'oeil qui ne seront compris que par les personnes concernées, sans éveiller la curiosité des autres. en particulier un truc pour la bretagne, ou tout/tous ou des trucs comme ça.

La thèse, c'est aussi beaucoup de moments inoubliables. Le vol d'un portefeuille (et tous les papiers d'identités) à Madrid, le premier jour d'une semaine de colloque, être perdu à 30km de Munich, moi qui ne sais dire péniblement que "gutten morgen" en allemand. Je voudrais chaleureusement remercier les concepteurs de Fortran pour me procurer si souvent la joie de voir un "Segmentation Fault" en guise de seule indication de bug pour un programme. Merci à Apple pour son merveilleux OS qui a réussi à me convaincre, en 3 ans de thèse qu'en fait, le hardware est malheureusement lui aussi du niveau du software chez eux. 

La thèse, c'est aussi beaucoup d'informatique. Pour le débug et les soucis informatiques quotidiens, merci à FranckH, Romuald (dit le débugueur fou, capable de trouver un bug dans un programme qu'il ne comprend pas), Sergi, ainsi que Pierre Gay et Benoit Hiroux pour leur aide précieuse lors de la création de mon code. 

%TODO remercier macOS et autre trucs à la con

%TODO famille en dernier

Merci à tous ceux qui ont une place si spéciale à mes yeux qu'une pauvre phrase au détour d'un paragraphe, ou un bête nom dans une liste serait leur faire insulte. 


VRAC : 

La science, c'est un peu comme les feux de l'amour. Il y a des associations partout, des pièges. Chacun travaille sur une interaction, il émet des hypothèses. Les autres travaillent avec ses conclusions, et on avance ainsi sur un système bancal qu'on rafistole au fur et à mesure.

Et le Franck d'or est décerné à.... (franckH et franckS ex aequo?)
Le comité des Franck d'or s'est réunis mais n'a pas pu trancher. Un franck d'or est donc décerné conjointement à Franck Hersant et Franck Selsis comme étant le meilleur duo de scientifique. Il paraitrait qu'une discussion qui réunirait ces deux entitées franckesques créerait un vortex d'entropie qui pourrait jusqu'à contredire la relativité générale. En effet, la quantité d'information transitant dans l'air serait voisine de celle autour de l'horizon des évènements d'un trou noir (non en fait c'est naze, ne garder que le début)

Le franck d'or est décerné à franck Hersant (blablabla) et le franck d'Honneur est décerné à Franck Selsis.

La conclusion de la thèse est "la piscine est trop près de la maison", en vertue du fait que j'ai sorti la tête hors de l'eau au bout de 2 ans et demi, pour me retrouver au pieds du mur.

un morceau du CID remasterisé par mes soins? (à améliorer, c'est juste pour l'idée pour l'instant)


Ô rage, Ô désespoir Ô recherche ennemie
N'ai-je donc tant vécu que pour cette infamie?
Et ne suis-je blanchi dans les travaux codés
que pour voir en un talk flétrir tant de lauriers? 
Mes doigts tant vénérés par tous les doctorants, 
mes doigts qui tant de fois ont codé du fortran
tant de fois fatigués sur mon pauvre clavier
trahissent donc ma simu et me font tout planter? 
Ô cruel souvenir d'exos solutionnés !
Œuvre de tant de jours en un jour effacée !
Nouveau \og Not A Number\fg fatal à mes planètes !
Précipice élevé d'où toutes elles se jettent !
Faut-il de votre éclat voir gagner idl, 
et mourir ruiné par les frais annuels ? 
idl, sois de mon code a présent évincé ; 
ce haut rang n'admet point de code non commenté
et ta vile licence par cet affront insigne
malgré tes librairies t'en a su rendre indigne. 
Et toi F77 glorieux instrument, 
Mais d'un code actuel inutile ornement, 
Fer jadis tant à craindre, et qui dans cette offense, 
m'a servi de support, et non de récompense. 
Va, quitte désormais mes simulations,
passe, pour me venger à l'ultime version.

/!\ Les 4 et 3e avant dernier vers sont à changer si possible. Rajouter un truc sur python, les cartes perforées ou un truc du style?


Citations : 
On ne rédige jamais aussi bien que quand on a les paupières lourdes comme des bouteilles de butane (sic) et le cerveau comme une compression de césar. (mardi 6 août 2013 ; 19h23)
