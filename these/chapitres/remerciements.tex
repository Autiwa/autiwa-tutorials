La thèse est un processus long, complexe, et il serait illusoire de faire croire, ou même tenter de faire croire, que je ne dois
ce présent manuscrit qu'à moi même. Cette section étant sans doute la seule que bon nombre de lecteurs parcourront, je vais donc
faire ici la conclusion scientifique de ma thèse, point par point, dans un style aussi soporifique que possible, et en latin,
juste par méchanceté gratuite, tout en me délectant du rictus qui doit déformer vos visages à cet instant. Ou pas.

À toutes les personnes du laboratoire qui n'ont pas écrit une ligne de cette thèse, mais qui y ont contribué indirectement je
dis merci. En particulier Annick et Cécile à la préparation des missions, toujours avec le sourire et la bonne humeur malgré mes
questions stupides. Pour un handicapé des transports comme moi, je suis passé de la peur de prendre le bus à un transatlantique
jusqu'au canada, et c'est un peu grâce à vous. 

Merci bien évidemment à Sean et arnaud sans qui non seulement je n'aurais jamais fait de thèse, mais je ne serais pas arrivé au
bout de celle-ci non plus ! Ça tombe sous le sens, mais j'ai beaucoup appris à leur contact. Merci à Alessandro, Caroline,
Richard et Aurélien d'avoir accepté de faire partie de mon Jury. La thèse n'étant pas le travail ayant le plus d'impact dans la
carrière d'un chercheur, je suis heureux d'avoir eu le privilège de vous le présenter.

Mais j'ai beaucoup d'autres gens à remercier. La minute d'expression corporelle gasconne si vous préférez. 

\bigskip

Merci à FranckS, pour sa disponibilité, sa culture inépuisable sur la photo, la SF, la physique et tout le reste. Sans lui qui
m'a fait réaliser que mon attrait pour l'astrophysique pouvait être bien plus réel et tangible que je le croyais, je n'aurais
sans doute jamais fait cette thèse. C'est par un stage de licence sous sa direction que j'ai mis un pieds au labo, c'est grâce à
lui que j'ai pris goût à la recherche. Pour la bonne humeur qu'il distille au labo, il a contribué largement à rendre ces 6 ans
agréables (Licence, M1, M2 + thèse, oui j'ai bien compté).

Merci à FranckH pour son aide, sa disponibilité, son humour, ses compétences aussi complètes que diverses, parait même qu'il
fait de la mécanique\dots En fin de thèse, et pour certaines parties, et avec toute la retenue dont j'ai essayé de faire preuve,
j'ai parfois pu abuser de sa gentillesse et de ses compétences. J'ai en tout cas la fierté de lui avoir posé plus de questions
de physique que de questions informatiques. Merci infiniment pour tout ce que tu as pu faire, pour le débug, les tests, les
discussions et tout le reste. Sans ton aide ma thèse et surtout sa rédaction n'auraient absolument pas été les mêmes. 

\bigskip

Même si vous méritez plus que ça, merci à Audrey, Émeline, Clément, Fanny, Claire,
Alice, Jonathan, Bastien et
Sergi pour les bons moments passés à l'observatoire ou ailleurs. Merci à David, entre autre pour son aide envers un expatrié sans papiers à Madrid.

Un merci tout particulier à Romuald, déjà pour sa capacité à trouver les bugs dans des codes informatiques qu'il ne comprend
pas, mais aussi le soutien qu'il a pu m'apporter tout au long de cette thèse. 

Un grand merci aussi à Marianne, pour sa bonne humeur, son humour décapant, et bien que ça ne se limite pas à ça, merci pour nos
mails bombing. 

Enfin, merci à ma famille, et en particulier mes parents pour m'avoir permis de donner corps à ma curiosité sans limite. Un deuxième merci pour la lecture attentive de mon manuscrit !

En guise d'introduction à quelques vers du CID remaniés en préface de ma thèse, je voudrais chaleureusement remercier les
concepteurs de Fortran pour me procurer si souvent la joie de voir un "Segmentation Fault" en guise de seule indication de bug
pour un programme. Merci à Apple pour m'avoir permis de démontrer qu'un doctorant sans mac, c'est aussi inconcevable qu'un
poisson sans bicyclette (\textcopyright Desproges). 


\begin{quote}
\og Ô rage, Ô désespoir Ô recherche ennemie\\
N'ai-je donc tant vécu que pour cette infamie?\\
Et ne suis-je blanchi dans les sources bugués\\
que pour voir en un talk flétrir tant de lauriers? \\
Mes doigts tant vénérés par tous les doctorants, \\
mes doigts qui tant de fois ont codé du fortran\\
tant de fois fatigués sur mon pauvre clavier\\
trahissent donc ma simu et me font tout planter? \\
Ô cruel souvenir d'exos solutionnés !\\
Œuvre de tant de jours en un jour effacée !\\
Nouveau \og Not A Number\fg fatal à mes planètes !\\
Précipice élevé d'où toutes elles se jettent !\\
Faut-il de votre éclat voir gagner idl, \\
et mourir ruiné par les frais annuels ? \\
idl, sois de mon code a présent évincé ; \\
ce haut rang n'admet point de code non commenté\\
et ta vile licence par cet affront insigne\\
malgré tes librairies t'en a su rendre indigne. \\
Et toi F77 glorieux instrument, \\
sans carte perforée inutile ornement, \\
fini tous tes goto et mots clés désuets\\
place à la programmation orienté objet\\
Va, quitte désormais mes simulations,\\
passe, pour me venger à l'ultime version.\fg

Monologue de C. Cossou après un n\ieme \textbf{Not A Number}\\
  --- \textit{Le Cid}, Corneille, 1637 et des poussières, acte I, scène 4, p. 20 vers 237-260
\end{quote}
